\documentclass[../main.tex]{subfiles}

\begin{document}




\chapter{Starting at the Beginning: The Natural Numbers}
\begin{itemize}
    \item \marginnote{6/15:}This text will begin by reviewing high school level material, but as rigorously as possible.
    \begin{itemize}
        \item It will teach the skill of proving complicated properties from simpler ones, allowing you to understand why an "obvious" statement really is obvious.
        \item One particularly important skill is the use of \textbf{mathematical induction}.
        \item We will strive to eliminate \textbf{circularity}.
    \end{itemize}
    \textbf{Circularity}: \dq{Using an advanced fact to prove a more elementary fact, and then later using the elementary fact to prove the advanced fact}{14}
    \item The number systems used in real analysis, listed in order of increasing sophistication, are the \textbf{naturals} $\N$\footnote{Note that in this text, the natural numbers will include 0. The natural numbers without 0 will be called the \textbf{positive integers} $\Z^+$.}, the \textbf{integers} $\Z$, the \textbf{rationals} $\Q$, and the \textbf{reals} $\R$.
    \begin{itemize}
        \item \textbf{Complex numbers} $\C$ will only be used much later.
    \end{itemize}
    \item This chapter will answer the question, "How does one actually \emph{define} the natural numbers?"
\end{itemize}



\section{The Peano Axioms}
\begin{itemize}
    \item \textbf{Peano Axioms}: First laid out by Guiseppe Peano, these are a standard way to define the natural numbers.
    \item How do you define operations on the naturals?
    \begin{itemize}
        \item Complicated operations are defined in terms of simpler ones: Exponentiation is repeated multiplication, multiplication is repeated addition, and addition is repeated \textbf{incrementing}.
    \end{itemize}
    \item \textbf{Incrementing}: The most fundamental operation --- best thought of as counting forward by one number.
    \begin{itemize}
        \item Incrementing is one of the fundamental concepts that allows us to define the natural numbers.
        \item Let\footnote{This notation is pulled from some computer languages such as $C$.} $n\pplus$ denote the increment, or \textbf{successor}, of $n$.
        \begin{itemize}
            \item For example, $3\pplus=4$ and $(3\pplus)\pplus=5$.
        \end{itemize}
    \end{itemize}
    \item Let $x:=y$ denote the statement, "$x$ is defined to equal $y$."
    \item At this point, we can begin defining the natural numbers.
    \begin{axm}\label{axm:0inN}
        0 is a natural number.
    \end{axm}
    \begin{axm}\label{axm:npplus}
        If $n$ is a natural number, then $n\pplus$ is also a natural number.
    \end{axm}
    \item To avoid having to use incrementation notation for every number, we adopt a convention.
    \begin{defn}
        We define 1 to be the number $0\pplus$, 2 to be the number $(0\pplus)\pplus$, 3 to be the number $((0\pplus)\pplus)\pplus$, etc.
    \end{defn}
    \item From these axioms, we can already prove things.
    \begin{prop}
        3 is a natural number.
        \begin{proof}
            By Axiom \ref{axm:0inN}, 0 is a natural number. By Axiom \ref{axm:npplus}, $0\pplus=1$ is a natural number. By Axiom \ref{axm:npplus} again, $1\pplus=2$ is a natural number. By Axiom \ref{axm:npplus} again, $2\pplus=3$ is a natural number.
        \end{proof}
    \end{prop}
    \item It seems like Axioms \ref{axm:0inN} and \ref{axm:npplus} have us pretty well covered. However, what if the number system wraps around (e.g., if $3\pplus=0$)? We can fix this with the following.
    \begin{axm}
        0 is not the successor of any natural number; i.e., we have $n\pplus\neq 0$ for every natural number $n$.
    \end{axm}
    \item We can now prove that $4\neq 0$ (because $4=3\pplus$, $3\in\N$, and $n\pplus\neq 0$).
    \item However, there are still issues --- what if the number system hits a ceiling at 4, e.g., $4\pplus=4$?
    \item A good way to prevent this kind of behavior is via the following.
    \begin{axm}\label{axm:successorDistinctness}
        Different natural numbers must have different successors, i.e., if $n,m\in\N$ and $n\neq m$, then $n\pplus\neq m\pplus$. Equivalently\footnote{This is an example of reformulating an implication using its \textbf{contrapositive}. In the converse direction, it is the \textbf{axiom of substitution}.}, if $n\pplus=m\pplus$, then $n=m$.
    \end{axm}
    \item We can now prove propositions like the following, extending our anti-wrap around proving ability.
    \begin{prop}
        6 is not equal to 2.
        \begin{proof}
            Suppose $6=2$. Then $5\pplus=1\pplus$, so by Axiom \ref{axm:successorDistinctness}, $5=1$. Then $4\pplus=0\pplus$, so by Axiom \ref{axm:successorDistinctness}, $4=0$, which contradicts our proof that $4\neq 0$.
        \end{proof}
    \end{prop}
\end{itemize}




\end{document}