\documentclass[../main.tex]{subfiles}

\pagestyle{main}
\renewcommand{\chaptermark}[1]{\markboth{\chaptername\ \thechapter: #1}{}}
\setcounter{chapter}{2}
\externaldocument{../main}

\begin{document}




\chapter{Set Theory}
\begin{itemize}
    \item \marginnote{7/4:}Set theory will be frequently used in the subsequent chapters; it is part of the foundation of almost every other branch of mathematics.
    \begin{itemize}
        \item Note that Euclidean geometry will not be defined --- we will use the Cartesian coordinate system's parallel with the real numbers instead.
    \end{itemize}
    \item This chapter covers the elementary aspects, Chapter \ref{chr:8} covers more advanced topics, and the finer subtleties are well beyond the scope of this text.
\end{itemize}



\section{Fundamentals}\label{sse:3.1}
\begin{itemize}
    \item We define sets axiomatically, as we did with the natural numbers\footnote{Note that the following list of axioms will be somewhat overcomplete, as some axioms may be derived from others. However, this is helpful for pedagogical reasons, and there is no real harm being done.}.
    \begin{axm}[Sets are objects]\label{axm:setsAreObjects}
        If $A$ is a set, then $A$ is also an object. In particular, given two sets $A$ and $B$, it is meaningful to ask whether $A$ is also an element of $B$.
    \end{axm}
    \item Note that while all sets are objects, not all objects are sets.
    \begin{itemize}
        \item For example, $1$ is not a set while $\{1\}$ is.
        \item Note, though, that \textbf{pure set theory} considers all objects to be sets. However, impure set theory (where some objects are not sets) is conceptually easier to deal with.
        \begin{itemize}
            \item Since both types are equal for the purposes of mathematics, we will take a middle-ground approach.
        \end{itemize}
    \end{itemize}
    \item If $x,y$ are objects and $A$ a set, then the statement $x\in A$ is either true or false. Note that $x\in y$ is neither true nor false, simply meaningless.
    \item We now define equality for sets.
    \begin{dfn}[Equality of sets]\label{dfn:setEquality}
        Two sets $A$ and $B$ are equal, $A=B$, iff every element of $A$ is an element of $B$ and vice versa. To put it another way, $A=B$ if and only if every element $x$ of $A$ belongs also to $B$, and every element $y$ of $B$ belongs also to $A$.
    \end{dfn}
    \item Note that this implies that repetition of elements does not effect equality ($\{3,3\}=\{3\}$, for example).
    \item It can be proven that this notion of equality is reflexive, symmetric, and transitive (see Exercise \ref{exr:3.1.1}).
    \item \marginnote{7/14:}Since $x\in A$ and $A=B$ implies $x\in B$, the $\in$ relation obeys the axiom of substitution as well.
    \begin{itemize}
        \item Thus, any operation defined in terms of the $\in$ relation obeys the axiom of substitution.
    \end{itemize}
    \item We define sets in an analogous way to how we defined natural numbers from 0, onward.
    \begin{axm}[Empty set]\label{axm:emptyset}
        There exists a set $\emptyset$ \textup{(}also denoted $\{\}$\textup{)}, known as the empty set, which contains no elements, i.e., for every object $x$, we have $x\notin\emptyset$.
    \end{axm}
    \begin{itemize}
        \item Note that there is only one empty set --- if $\emptyset$ and $\emptyset'$ were supposedly distinct empty sets, then Definition \ref{dfn:setEquality} would prove that $\emptyset=\emptyset'$.
    \end{itemize}
    \item \textbf{Non-empty set}: A set that \dq{is not equal to the empty set}{36}
    \item Non-empty sets must contain at least one object.
    \begin{lem}[Single choice]\label{lem:singleChoice}
        Let $A$ be a non-empty set. Then there exists an object $x$ such that $x\in A$.
        \begin{proof}
            Suppose for the sake of contradiction that no object $x$ exists such that $x\in A$, i.e., for all objects $x$, we have $x\notin A$. By Axiom \ref{axm:emptyset}, we have $x\notin\emptyset$ either. Thus, $x\in A \Longleftrightarrow x\in\emptyset$ (denoting logical equivalence; both statements are equally false), so, by Definition \ref{dfn:setEquality}, $A=\emptyset$, a contradiction.
        \end{proof}
    \end{lem}
    \item There exist more sets than just the empty set.
    \begin{axm}[Singleton sets and pair sets]\label{axm:singletonPair}
        If $a$ is an object, then there exists a set $\{a\}$ whose only element is $a$, i.e., for every object $y$, we have $y\in\{a\}$ if and only if $y=a$; we refer to $\{a\}$ as the \textbf{singleton set} whose element is $a$. Furthermore, if $a$ and $b$ are objects, then there exists a set $\{a,b\}$ whose elements are $a$ and $b$; i.e., for every object $y$, we have $y\in\{a,b\}$ if and only if $y=a$ or $y=b$; we refer to this set as the \textbf{pair set} formed by $a$ and $b$.
    \end{axm}
    \item By Definition \ref{dfn:setEquality}, there exists only one singleton set for each object $a$ and only one pair set for any two objects $a,b$.
    \item Note that the singleton set axiom follows from the pair set axiom, and the pair set axiom follows from the singleton set axiom and the pairwise union axiom, below.
    \item As alluded to, the pairwise union axiom allows us to build sets with more than two elements.
    \begin{axm}[Pairwise union]\label{axm:pairwiseUnion}
        Given any two sets $A,B$, there exists a set $A\cup B$ called the \textbf{union} $A\cup B$ of $A$ and $B$, whose elements consist of all the elements which belong to $A$ or $B$ or both. In other words, for any object $x$,
        \begin{equation*}
            x\in A\cup B \Longleftrightarrow (x\in A\text{ or }x\in B)
        \end{equation*}
    \end{axm}
    \item The $\cup$ operation obeys the axiom of substitution (if $A=A'$, then $A\cup B=A'\cup B$).
    \item We now prove some basic properties of unions (one below and three in Exercise \ref{exr:3.1.3}).
    \begin{lem}\label{lem:associativitySetUnion}
        If $A,B,C$ are sets, then the union operation is associative, i.e., $(A\cup B)\cup C=A\cup(B\cup C)$.
        \begin{proof}
            By Definition \ref{dfn:setEquality}, showing that every element $x$ of $(A\cup B)\cup C$ is an element of $A\cup(B\cup C)$ and vice versa will suffice to prove this lemma. Suppose first that $x\in(A\cup B)\cup C$. By Axiom \ref{axm:pairwiseUnion}, this means that at least one of $x\in A\cup B$ or $x\in C$ is true. We now divide into two cases. If $x\in C$, then by Axiom \ref{axm:pairwiseUnion}, $x\in B\cup C$, and, so, by Axiom \ref{axm:pairwiseUnion} again, we have $x\in A\cup(B\cup C)$. Now suppose instead that $x\in A\cup B$. Then by Axiom \ref{axm:pairwiseUnion}, $x\in A$ or $x\in B$. If $x\in A$, then $x\in A\cup(B\cup C)$ by Axiom \ref{axm:pairwiseUnion}, while if $x\in B$, then by consecutive applications of Axiom \ref{axm:pairwiseUnion}, we have $x\in B\cup C$ and, hence, $x\in A\cup (B\cup C)$. A similar argument shows that every element of $A\cup(B\cup C)$ lies in $(A\cup B)\cup C$, and so $(A\cup B)\cup C=A\cup(B\cup C)$, as desired.
        \end{proof}
    \end{lem}
    \item As a consequence of the above, we are free to write $A\cup B\cup C\cup\cdots$ to denote repeated unions without having to use parentheses.
    \item We can also now define triplet sets ($\{a,b,c\}:=\{a\}\cup\{b\}\cup\{c\}$), quadruplet sets, and so forth.
    \begin{itemize}
        \item However, we cannot yet define a set of $n$ objects or an infinite set.
    \end{itemize}
    \item Note that addition and union are analogous, but importantly \emph{not} identical.
    \item Some sets are "larger" than others; hence, subsets.
    \begin{dfn}[Subsets]\label{dfn:subsets}
        Let $A,B$ be sets. We say that $A$ is a \textbf{subset} of $B$, denoted $A\subseteq B$, iff every element of $A$ is also an element of $B$, i.e., for any object $x$, $x\in A \Longrightarrow x\in B$. We say that $A$ is a \textbf{proper subset} of $B$, denoted $A\subsetneq B$ if $A\subseteq B$ and $A\neq B$.
    \end{dfn}
    \item The $\subseteq$ and $\subsetneq$ operations obey the axiom of substitution (since both $=$ and $\in$, the two component operations of $\subseteq$ and $\subsetneq$, obey it).
    \item Note that $A\subseteq A$ and $\emptyset\subseteq A$ for any set $A$.
    \item Note that less than or equal to and subset are analogous, but not identical, either (see below for one related property and Exercise \ref{exr:3.1.4} for two more).
    \begin{prp}[Sets are partially ordered by set inclusion 1]\label{prp:subsetTransitive}
        Let $A,B,C$ be sets. If $A\subseteq B$ and $B\subseteq C$, then $A\subseteq C$.
        \begin{proof}
            Suppose that $A\subseteq B$ and $B\subseteq C$. To prove that $A\subseteq C$, we have to prove that every element of $A$ is an element of $C$. Let $x\in A$. Then $x\in B$ (since $A\subseteq B$), implying that $x\in C$ (since $B\subseteq C$).
        \end{proof}
    \end{prp}
    \item There exist relations between subsets and unions (see Exercise \ref{exr:3.1.7}).
    \item Note this difference between $\subsetneq$ and $<$: Given any two distinct natural numbers $n,m$, one is smaller than the other (Proposition \ref{prp:trichotomy}). However, given any two distinct sets, it is not in general true that one is a subset of the other. This is why we say that sets are \textbf{partially ordered} while the natural numbers (for example) are \textbf{totally ordered} (see Definitions \ref{dfn:partiallyOrderedSets} and \ref{dfn:totallyOrderedSet}, respectively).
    \item Note that $\in$ and $\subseteq$ are distinct ($2\in\{1,2,3\}$, but $2\nsubseteq\{1,2,3\}$; similarly, $\{2\}\subseteq\{1,2,3\}$, but $\{2\}\notin\{1,2,3\}$).
    \item It is important to distinguish sets from their elements, for they can have different properties ($\N$ is an infinite set of finite elements, and $\{\N,\Z,\Q,\R\}$ is a finite set of infinite objects).
    \item We now formally state that it is acceptable to create subsets.
    \begin{axm}[Axiom of specification]\label{axm:specification}
        Let $A$ be a set, and for each $x\in A$, let $P(x)$ be a property pertaining to $x$ \textup{(}i.e., $P(x)$ is either a true statement or a false statement\textup{)}. Then there exists a set, called $\{x\in A:P(x)\text{ is true}\}$ \textup{(}or simply $\{x\in A:P(x)\}$ for short\textup{)}, whose elements are precisely the elements $x$ in $A$ for which $P(x)$ is true. In other words, for any object $y$,
        \begin{equation*}
            y\in\{x\in A:P(x)\text{ is true}\} \Longleftrightarrow (y\in A\text{ and }P(y)\text{ is true})
        \end{equation*}
    \end{axm}
    \item \emph{Also known as} \textbf{axiom of separation}.
    \item Specification obeys the axiom of substitution (since $\in$ obeys it).
    \item Sometimes $\{x\in A:P(x)\}$ is denoted by $\{x\in A\Big|P(x)\}$ (useful when we need the colon for something else, e.g., $f:X\to Y$).
    \item We use Axiom \ref{axm:specification} to define intersections.
    \begin{dfn}[Intersections]\label{dfn:intersection}
        The \textbf{intersection} $S_1\cap S_2$ of two sets is defined to be the set
        \begin{equation*}
            S_1\cap S_2 := \{x\in S_1:x\in S_2\}
        \end{equation*}
        In other words, $S_1\cap S_2$ consists of all the elements which belong to both $S_1$ and $S_2$. Thus, for all objects $x$,
        \begin{equation*}
            x\in S_1\cap S_2 \Longleftrightarrow x\in S_1\text{ and }x\in S_2
        \end{equation*}
    \end{dfn}
    \item The $\cap$ operation obeys the axiom of substitution (since $\in$ obeys it).
    \begin{itemize}
        \item Note that since $\cap$ is defined in terms of more primitive operations, it is well-defined.
    \end{itemize}
    \item Problems with the English word, "and."
    \begin{itemize}
        \item It can mean union or intersection depending on the context.
        \begin{itemize}
            \item If $X,Y$ are sets, "the set of elements of $X$ and elements of $Y$" refers to $X\cup Y$, e.g., "the set of singles and males."
            \item If $X,Y$ are sets, "the set of objects that are elements of $X$ and elements of $Y$" refers to $X\cap Y$, e.g., "the set of people who are single and male."
        \end{itemize}
        \item It can also denote addition.
        \begin{itemize}
            \item "2 and 3 is 5" means $2+3=5$.
        \end{itemize}
        \item \dq{One reason we resort to mathematical symbols instead of English words such as `and' is that mathematical symbols always have a precise and unambiguous meaning, whereas one must often look very carefully at the context in order to work out what an English word means}{42}
    \end{itemize}
    \item \textbf{Disjoint} (sets): Two sets $A,B$ such that $A\cap B=\emptyset$.
    \item \textbf{Distinct} (sets): Two sets $A,B$ such that $A\neq B$.
    \begin{itemize}
        \item Note that $\emptyset$ and $\emptyset$ are disjoint but not distinct.
    \end{itemize}
    \item We also use Axiom \ref{axm:specification} to define difference sets.
    \begin{dfn}[Difference sets]\label{dfn:differenceSets}
        Given two sets $A$ and $B$, we define the set $A-B$ or $A\setminus B$ to be the set $A$ with any elements of $B$ removed, i.e.,
        \begin{equation*}
            A\setminus B := \{x\in A:x\notin B\}
        \end{equation*}
    \end{dfn}
    \item For example, $\{1,2,3,4\}\setminus\{2,4,6\}=\{1,3\}$ --- in many cases, $B\subseteq A$, but not necessarily.
    \item \marginnote{7/15:}See Exercise \ref{exr:3.1.6} for some basic properties of unions, intersections, and difference sets.
    \begin{itemize}
        \item The de Morgan laws are named after the logician Augustus De Morgan (1806-1871).
        \item The \textbf{laws of Boolean algebra} (the contents of Proposition \ref{prp:booleanAlgebra}) are named after the mathematician George Boole (1815-1864).
        \begin{itemize}
            \item They are applicable to a number of objects other than sets (e.g., laws of propositional logic).
        \end{itemize}
        \item Note the \textbf{duality} in Proposition \ref{prp:booleanAlgebra}, manifesting itself in the certain symmetry between $\cup$ and $\cap$, and $X$ and $\emptyset$.
    \end{itemize}
    \item \textbf{Duality}: \dq{Two distinct properties or objects being dual to each other}{43}
    \item We can do a lot, but we still wish to do more. For example, we'd like to transform sets (say, take $\{3,5,9\}$ and increment each object to yield $\{4,6,10\}$).
    \begin{axm}[Replacement]\label{axm:replacement}
        Let $A$ be a set. For any object $x\in A$, and any object $y$, suppose we have a statement $P(x,y)$ pertaining to $x$ and $y$ such that for each $x\in A$, there is at most one $y$ for which $P(x,y)$ is true. Then there exists a set $\{y:P(x,y)\text{ is true for some }x\in A\}$, such that for any object $z$,
        \begin{equation*}
            z\in\{y:P(x,y)\text{ is true for some }x\in A\} \Longleftrightarrow P(x,z)\text{ is true for some }x\in A
        \end{equation*}
    \end{axm}
    \item For example, let $A:=\{3,5,9\}$ and let $P(x,y)$ be the statement $y=x\pplus$. By Axiom \ref{axm:successorDistinctness}, for every $x\in A$, there is exactly one $y$ for which $P(x,y)$ is true (namely, the successor of $x$). Thus, by Axiom \ref{axm:replacement}, the set $\{y:y=x\pplus\text{ for some }x\in\{3,5,9\}\}$ exists. It is clearly the same set as $\{4,6,10\}$.
    % \begin{itemize}
    %     \item Let $B:=\{y:y=x\pplus\text{ for some }x\in\{3,5,9\}\}$ and $C:=\{4,6,10\}$. Show $B=C$.
    %     \begin{proof}
    %         By Definition \ref{dfn:setEquality}, it will suffice to show that every element of $B$ is an element of $C$ and vice versa. Suppose $z\in\{y:y=x\pplus\text{ for some }x\in\{3,5,9\}\}$. By Axiom \ref{axm:replacement}, $z=x\pplus$ for some $x\in\{3,5,9\}$.
    %     \end{proof}
    % \end{itemize}
    \item The set obtained can be smaller than the original set, e.g., $\{y:y=1\text{ for some }x\in\{3,5,9\}\}=\{1\}$.
    \item We often abbreviate the set specified in Axiom \ref{axm:replacement} to one of the following.
    \begin{gather*}
        \{y:y=f(x)\text{ for some }x\in A\}\\
        \{f(x):x\in A\}\\
        \{f(x)\Big|x\in A\}
    \end{gather*}
    \item We can combine Axioms \ref{axm:replacement} and \ref{axm:specification}, i.e., $\{f(x):x\in A;P(x)\text{ is true}\}$, e.g., $\{n\pplus:n\in\{3,5,9\};n<6\}=\{4,6\}$.
    \item Although we have assumed that natural numbers are objects in several examples up to this point, we must formalize this notion.
    \begin{axm}[Infinity]\label{axm:infinity}
        There exists a set $\N$, whose elements are called the natural numbers, as well as an object $0$ in $\N$, and an object $n\pplus$ assigned to every natural number $n\in N$, such that the Peano axioms hold.
    \end{axm}
    \item Axiom \ref{axm:infinity} is called the \textbf{axiom of infinity} because \dq{it introduces the most basic example of an infinite set, namely the set of natural numbers $\N$}{44}
\end{itemize}


\subsection*{Exercises}
\begin{enumerate}[ref={\thesection.\arabic*}]
    \item \label{exr:3.1.1}\marginnote{7/4:}Show that the definition of equality in Definition \ref{dfn:setEquality} is reflexive, symmetric, and transitive\footnote{Note that since Definition \ref{dfn:setEquality} should be an axiom (should axiomatize equality and all that that entails for sets), this exercise is silly (see \cite{bib:TaoErrata}).}.
    \begin{proof}
        Given a set $A$, suppose $A\neq A$. Then, by Definition \ref{dfn:setEquality}, every element of $A$ is not an element of $A$, a contradiction. Thus, $A=A$.\par
        Let sets $A=B$. Then, by Definition \ref{dfn:setEquality}, every element $x$ of $A$ belongs also to $B$, and every element $y$ of $B$ belongs also to $A$. Identically, every element $y$ of $B$ belongs also to $A$, and every element $X$ of $A$ belongs also to $B$. Thus, $B=A$.\par
        Let sets $A=B$ and $B=C$. Then, by Definition \ref{dfn:setEquality}, every element $x$ of $A$ belongs also to $B$, and every element $y$ of $B$ belongs also to $A$. Similarly, every element $y$ of $B$ belongs also to $C$, and every element $z$ of $C$ belongs also to $B$. Since $x\in A\Rightarrow x\in B\Rightarrow x\in C$, and $y\in C\Rightarrow y\in B\Rightarrow y\in A$, $A=C$.
    \end{proof}
    \item \label{exr:3.1.2}\marginnote{7/14:}Using only Definition \ref{dfn:setEquality}, Axiom \ref{axm:setsAreObjects}, Axiom \ref{axm:emptyset}, and Axiom \ref{axm:singletonPair}, prove that the sets $\emptyset$, $\{\emptyset\}$, $\{\{\emptyset\}\}$, and $\{\emptyset,\{\emptyset\}\}$ are all distinct (i.e., no two of them are equal to each other).
    \begin{proof}
        % Super rigorous: Suppose for the sake of contradiction that $\emptyset=\{\emptyset\}$. By Axiom \ref{axm:setsAreObjects}, $\emptyset$ is an object, implying by Axiom \ref{axm:singletonPair} that there exists a set $\{\emptyset\}$ (which happens to be the set on the right of the hypothesized equality). It follows that $\emptyset\in\{\emptyset\}$. Since $\emptyset=\{\emptyset\}$, by Definition \ref{dfn:setEquality}, $\emptyset\in\emptyset$ as well. But this contradicts Axiom \ref{axm:emptyset}, which asserts that $x\notin\emptyset$ for all objects $x$, including $\emptyset$. Therefore, $\emptyset\neq\{\emptyset\}$.
        First, we show that all sets are distinct from the empty set. Axiom \ref{axm:setsAreObjects} asserts that $\emptyset$ and $\{\emptyset\}$ are objects. By Axiom \ref{axm:singletonPair}, we have $\emptyset\in\{\emptyset\}$, $\emptyset\in\{\emptyset,\{\emptyset\}\}$, and $\{\emptyset\}\in\{\{\emptyset\}\}$. Since $x\notin\emptyset$ for all objects $x$ (Axiom \ref{axm:emptyset}), $\{\emptyset\}$, $\{\{\emptyset\}\}$, and $\{\emptyset,\{\emptyset\}\}$ all contain objects that $\emptyset$ does not (namely, $\emptyset$, $\{\emptyset\}$, and $\emptyset$, respectively). Thus, by Definition \ref{dfn:setEquality}, $\emptyset\neq\{\emptyset\}$, $\emptyset\neq\{\{\emptyset\}\}$, and $\emptyset\neq\{\emptyset,\{\emptyset\}\}$.\par
        Next, we show that $\{\emptyset\}\neq\{\emptyset,\{\emptyset\}\}$. By Axiom \ref{axm:singletonPair}, $\{\emptyset\}\in\{\emptyset,\{\emptyset\}\}$ and $y\in\{\emptyset\}$ iff $y=\emptyset$. Since $\{\emptyset\}\neq\emptyset$ (see above), $\{\emptyset\}\notin\{\emptyset\}$. Thus, $\{\emptyset,\{\emptyset\}\}$ contains an object that $\{\emptyset\}$ does not, implying by Definition \ref{dfn:setEquality} that $\{\emptyset\}\neq\{\emptyset,\{\emptyset\}\}$.\par
        Lastly, we show that $\{\emptyset\}\neq\{\{\emptyset\}\}$ and that $\{\emptyset,\{\emptyset\}\}\neq\{\{\emptyset\}\}$. We proceed in a similar manner to the above. By Axiom \ref{axm:singletonPair}, $\emptyset\in\{\emptyset\}$, $\emptyset\in\{\emptyset,\{\emptyset\}\}$, and $y\in\{\{\emptyset\}\}$ iff $y=\{\emptyset\}$. Since $\emptyset\neq\{\emptyset\}$ (see above), $\emptyset\notin\{\{\emptyset\}\}$. Thus, $\{\emptyset\}$ and $\{\emptyset,\{\emptyset\}\}$ both contain an object that $\{\{\emptyset\}\}$ does not, implying by Definition \ref{dfn:setEquality} that $\{\emptyset\}\neq\{\{\emptyset\}\}$ and that $\{\emptyset,\{\emptyset\}\}\neq\{\{\emptyset\}\}$.\par
        % Second, we show that $\{\emptyset\}\neq\{\{\emptyset\}\}$. By Axiom \ref{axm:singletonPair}, $\emptyset\in\{\emptyset\}$ and $y\in\{\{\emptyset\}\}$ iff $y=\{\emptyset\}$. Since $\emptyset\neq\{\emptyset\}$ (see above), $\emptyset\notin\{\{\emptyset\}\}$. Thus, $\{\emptyset\}$ contains an object that $\{\{\emptyset\}\}$ does not, implying by Definition \ref{dfn:setEquality} that $\{\emptyset\}\neq\{\{\emptyset\}\}$.\par
        % Lastly, we show that $\{\{\emptyset\}\}\neq\{\emptyset,\{\emptyset\}\}$. We proceed in a similar manner to the previous two. By Axiom \ref{axm:singletonPair}, $\emptyset\in\{\emptyset,\{\emptyset\}\}$ and $y\in\{\{\emptyset\}\}$ iff $y=\{\emptyset\}$. Since $\emptyset\neq\{\emptyset\}$ (see above), $\emptyset\notin\{\{\emptyset\}\}$. Thus, $\{\emptyset,\{\emptyset\}\}$ contains an object that $\{\{\emptyset\}\}$ does not, implying by Definition \ref{dfn:setEquality} that $\{\{\emptyset\}\}\neq\{\emptyset,\{\emptyset\}\}$.
        %  and $\{\emptyset\}\neq\{\emptyset,\{\emptyset\}\}$.
    \end{proof}
    \item \label{exr:3.1.3}Prove the following lemmas.
    \begin{lem}\label{lem:commutativityUnion}
        If $a$ and $b$ are objects, then $\{a,b\}=\{a\}\cup\{b\}$.
        \begin{proof}
            % $\{a\}$ and $\{b\}$ are sets. Thus, by Axiom \ref{axm:pairwiseUnion}, there exists a set $\{a\}\cup\{b\}$ whose elements consist of all the elements which belong to $\{a\}$ or $\{b\}$ or both. By Axiom \ref{axm:singletonPair}, $x\in\{a\}$ iff $x=a$ (implying that $a$ is the only element of $\{a\}$) and $y\in\{b\}$ iff $y=b$ (implying that $b$ is the only element of $\{b\}$). Thus, $\{a\}\cup\{b\}$ must contain exclusively $a,b$. By Axiom \ref{axm:singletonPair}, the set that contains just $a,b$ is $\{a,b\}$. Therefore, $\{a,b\}=\{a\}\cup\{b\}$.
            By Definition \ref{dfn:setEquality}, it will suffice to show that every element $x$ of $\{a,b\}$ is an element of $\{a\}\cup\{b\}$ and vice versa. By Axiom \ref{axm:singletonPair}, if $x\in\{a,b\}$, then $x=a$ or $x=b$. By Axiom \ref{axm:pairwiseUnion}, if $x\in\{a\}\cup\{b\}$, then $x\in\{a\}$ or $x\in\{b\}$, implying by Axiom \ref{axm:singletonPair} that $x=a$ or $x=b$. Thus, the elements of $\{a,b\}$ and of $\{a\}\cup\{b\}$ are both $a,b$, so by Definition \ref{dfn:setEquality}, the sets are equal.
        \end{proof}
    \end{lem}
    \begin{lem}
        If $A,B,C$ are sets, then the union operation is commutative, i.e., $A\cup B=B\cup A$.
        \begin{proof}
            By Definition \ref{dfn:setEquality}, it will suffice to show that every element $x$ of $A\cup B$ is an element of $B\cup A$ and vice versa. Let $x\in A\cup B$. By Axiom \ref{axm:pairwiseUnion}, $x\in A$ or $x\in B$. If, on the one hand, $x\in A$, then $x\in B\cup A$ (by Axiom \ref{axm:pairwiseUnion}). If, on the other hand, $x\in B$, then $x\in B\cup A$ (by Axiom \ref{axm:pairwiseUnion}). A similar argument holds if we choose an element $y\in B\cup A$ first.
        \end{proof}
    \end{lem}
    \begin{lem}
        If $A$ is a set, then $A=A\cup\emptyset=\emptyset\cup A=A\cup A$.
        \begin{proof}
            First, we show that $A\cup\emptyset=\emptyset\cup A$. This is a direct consequence of Lemma \ref{lem:commutativityUnion}.\par
            Next, we show that $A=A\cup\emptyset$. By Definition \ref{dfn:setEquality}, it will suffice to show that every element $x$ of $A$ is an element of $A\cup\emptyset$ and vice versa. By Axiom \ref{axm:pairwiseUnion}, every element $x$ of $A$ is an element of $A\cup\emptyset$. Now let $x\in A\cup\emptyset$. Then by Axiom \ref{axm:pairwiseUnion}, $x\in A$ or $x\in\emptyset$. By Axiom \ref{axm:emptyset}, $x\notin\emptyset$, so $x\in A$. Thus, every element of $A\cup\emptyset$ is an element of $A$. We now have by the transitive property that $A=A\cup\emptyset=\emptyset\cup A$.\par
            Lastly, we show that $A=A\cup A$. By Definition \ref{dfn:setEquality}, it will suffice to show that every element $x$ of $A$ is an element of $A\cup A$ and vice versa. By Axiom \ref{axm:pairwiseUnion}, every element $x$ of $A$ is an element of $A\cup A$. Now let $x\in A\cup A$. Then by Axiom \ref{axm:pairwiseUnion}, $x\in A$ or $x\in A$, implying $x\in A$. We have, at last, by the transitive property that $A=A\cup\emptyset=\emptyset\cup A=A\cup A$.
        \end{proof}
    \end{lem}
    \item \label{exr:3.1.4}Prove the following propositions.
    \begin{prp}[Sets are partially ordered by set inclusion 2]
        Let $A,B$ be sets. If $A\subseteq B$ and $B\subseteq A$, then $A=B$.
        \begin{proof}
            Suppose that $A\subseteq B$ and $B\subseteq A$. By Definition \ref{dfn:subsets}, $A\subseteq B$ implies that every element of $A$ is also an element of $B$ and $B\subseteq A$ implies that every element of $B$ is also an element of $A$. Thus, by Definition \ref{dfn:setEquality}, $A=B$.
        \end{proof}
    \end{prp}
    \begin{prp}[Sets are partially ordered by set inclusion 3]
        Let $A,B,C$ be sets. If $A\subsetneq B$ and $B\subsetneq C$, then $A\subsetneq C$.
        \begin{proof}
            Suppose that $A\subsetneq B$ and $B\subsetneq C$. Then, by Definition \ref{dfn:subsets}, $A\subseteq B$ and $B\subseteq C$, implying $A\subseteq C$ by the first claim proved. Since $A\subsetneq B$, $A\neq B$ (implying, by Definition \ref{dfn:setEquality}, that every element of $A$ is not an element of $B$ or every element of $B$ is not an element of $A$) and every element of $A$ is an element of $B$; hence, every element of $B$ is not an element of $A$. Therefore, $B$ \emph{must} contain some element that $A$ does not. Similarly, $B\subsetneq C$ implies that $C$ must contain some element that $B$ does not. Hence, $C$ contains at least two elements that $A$ does not, proving that $A\neq C$, too.
        \end{proof}
    \end{prp}
    \item \label{exr:3.1.5}Let $A,B$ be sets. Show that the three statements $A\subseteq B$, $A\cup B=B$, and $A\cap B=A$ are logically equivalent (any one of them implies the other two).
    \begin{proof}
        First, we show that $A\subseteq B \Longrightarrow (A\cup B=B\text{ and }A\cap B=A)$. Next, we show that $A\cup B=B \Longrightarrow A\subseteq B$ (which, in turn, implies $A\cap B=A$). Lastly, we show that $A\cap B=A \Longrightarrow A\subseteq B$ (which, in turn, implies $A\cup B=B$). Let's begin.\par
        Suppose that $A\subseteq B$. To prove $A\cup B=B$, Definition \ref{dfn:setEquality} tells us that it will suffice to show that every element $x$ of $A\cup B$ is an element of $B$ and vice versa. By Axiom \ref{axm:pairwiseUnion}, $x\in B \Longrightarrow x\in A\cup B$. By Axiom \ref{axm:pairwiseUnion}, $x\in A\cup B \Longrightarrow (x\in A\text{ or }x\in B)$. By Definition \ref{dfn:subsets}, $A\subseteq B$ means that $x\in A \Longrightarrow x\in B$. Thus, $x\in A\cup B \Longrightarrow (x\in A\text{ or }x\in B) \Longrightarrow x\in B$. Therefore, if $A\subseteq B$, then $A\cup B=B$. To prove that $A\cap B=A$, Definition \ref{dfn:setEquality} tells us that it will suffice to show that every element $x$ of $A\cap B$ is an element of $A$ and vice versa. By Definition \ref{dfn:intersection}, $x\in A\cap B \Longrightarrow x\in A$ (and $x\in B$). Since $x\in A \Longrightarrow x\in B$ (see above), $x\in A \Longrightarrow (x\in A\text{ and }x\in B) \Longrightarrow x\in A\cap B$ (Definition \ref{dfn:intersection}). Therefore, if $A\subseteq B$, then $A\cap B=A$.\par
        Suppose that $A\cup B=B$. To prove $A\subseteq B$, Definition \ref{dfn:subsets} tells us that it will suffice to show that every element of $A$ is also an element of $B$. By Axiom \ref{axm:pairwiseUnion}, $x\in A \Longrightarrow x\in A\cup B$. By Definition \ref{dfn:setEquality}, $y\in A\cup B \Longrightarrow y\in B$. Thus, $x\in A \Longrightarrow x\in A\cup B \Longrightarrow x\in B$. Therefore, if $A\cup B=B$, $A\subseteq B$ (and $A\cap B=A$).\par
        Suppose that $A\cap B=A$. To prove that $A\subseteq B$, Definition \ref{dfn:subsets} tells us that it will suffice to show that every element of $A$ is also an element of $B$. By Definition \ref{dfn:setEquality}, $x\in A \Longrightarrow x\in A\cap B$. By Definition \ref{dfn:intersection}, $y\in A\cap B \Longrightarrow y\in B$. Thus, $x\in A \Longrightarrow x\in A\cap B \Longrightarrow x\in B$. Therefore, if $A\cap B=A$, $A\subseteq B$ (and $A\cup B=B$).
    \end{proof}
    \item \label{exr:3.1.6}\marginnote{7/15:}Prove the following proposition. (Hint: one can use some of these claims to prove others. Some of the claims have also appeared previously in Lemma \ref{lem:associativitySetUnion} and Exercise \ref{exr:3.1.3}.)
    \begin{prp}[Sets form a boolean algebra]\label{prp:booleanAlgebra}
        Let $A,B,C$ be sets, and let $X$ be a set containing $A,B,C$ as subsets.
        \begin{enumerate}[label={\textup{(}\alph*\textup{)}},ref={\theenumi\alph*}]
            \item \label{exr:3.1.6a}(Minimal element) We have $A\cup\emptyset=A$ and $A\cap\emptyset=\emptyset$.
            \begin{proof}
                See Exercise \ref{exr:3.1.3} for the first claim.\par
                To prove $A\cap\emptyset=\emptyset$, Definition \ref{dfn:setEquality} tells us that it will suffice to show that every element $x$ of $A\cap\emptyset$ is an element of $\emptyset$ and vice versa.
                % By Axiom \ref{axm:emptyset}, $x\notin\emptyset$ for all objects $x$. By Definition \ref{dfn:intersection}, $A\cap\emptyset=\{x\in A:x\in\emptyset\}$. But $x\in\emptyset$ is necessarily false for all $x$ (see above). Thus, $x\notin A\cap\emptyset$ for all objects $x$ (specifically, all $x\in A$). Therefore, the statement "every element $x$ of $A\cap\emptyset$ is an element of $\emptyset$ and vice versa" is vacuously true.
                First off, "every element $x\in\emptyset$ is an element of $A\cap\emptyset$" is vacuously true (by Axiom \ref{axm:emptyset}, there exists no $x\in\emptyset$). In the other direction, suppose for the sake of contradiction that $x\in A\cap\emptyset$ for some object $x$. By Definition \ref{dfn:intersection}, $x\in A$ and $x\in \emptyset$. But $x\in\emptyset$ contradicts Axiom \ref{axm:emptyset}. Therefore, $x\notin A\cap\emptyset$ for all objects $x$. Thus, the statement "every element $x\in A\cap\emptyset$ is an element of $\emptyset$" is vacuously true (there exists no $x\in A\cap\emptyset$).
            \end{proof}
            \item \label{exr:3.1.6b}(Maximal element) We have $A\cup X=X$ and $A\cap X=A$.
            \begin{proof}
                See Exercise \ref{exr:3.1.5}.
            \end{proof}
            \item \label{exr:3.1.6c}(Identity) We have $A\cap A=A$ and $A\cup A=A$.
            \begin{proof}
                To prove that $A\cap A=A$, Definition \ref{dfn:setEquality} tells us that it will suffice to show that every element $x$ of $A\cap A$ is an element of $A$ and vice versa. By Definition \ref{dfn:intersection}, $x\in A\cap A \Longrightarrow (x\in A\text{ and }x\in A) \Longrightarrow x\in A$. On the other hand, $x\in A \Longrightarrow (x\in A\text{ and }x\in A)$ (idempotent law for conjunction), $(x\in A\text{ and }x\in A) \Longrightarrow x\in\{x\in A:x\in A\}$ (Axiom \ref{axm:specification}), and $x\in\{x\in A:x\in A\} \Longrightarrow x\in A\cap A$ (Definition \ref{dfn:intersection}).\par
                See Exercise \ref{exr:3.1.3} for the second claim.
            \end{proof}
            \item \label{exr:3.1.6d}(Commutativity) We have $A\cup B=B\cup A$ and $A\cap B=B\cap A$.
            \begin{proof}
                See Exercise \ref{exr:3.1.3} for the first claim.\par
                To prove $A\cap B=B\cap A$, Definition \ref{dfn:setEquality} tells us that it will suffice to show that every element $x$ of $A\cap B$ is an element of $B\cap A$ and vice versa. By two applications of Definition \ref{dfn:intersection} with the commutative law for conjunction in between, $x\in A\cap B \Longrightarrow (x\in A\text{ and }x\in B) \Longrightarrow (x\in B\text{ and }x\in A) \Longrightarrow x\in B\cap A$. A similar argument works in the opposite direction.
            \end{proof}
            \item \label{exr:3.1.6e}(Associativity) We have $(A\cup B)\cup C=A\cup(B\cup C)$ and $(A\cap B)\cap C=A\cap(B\cap C)$.
            \begin{proof}
                See Lemma \ref{lem:associativitySetUnion} for the first claim.\par
                By Definition \ref{dfn:setEquality}, showing that every element $x$ of $(A\cap B)\cap C$ is an element of $A\cap(B\cap C)$ and vice versa will suffice to prove this lemma. Suppose first that $x\in (A\cap B)\cap C$. By Definition \ref{dfn:intersection}, $x\in A\cap B$ and $x\in C$, which implies by a second application of Definition \ref{dfn:intersection} that $x\in A$ and $x\in B$ and $x\in C$. It follows by consecutive applications of Definition \ref{dfn:intersection} that $x\in A$ and $x\in B\cap C$, and that $x\in A\cap(B\cap C)$. A similar argument shows that every element of $A\cap(B\cap C)$ lies in $(A\cap B)\cap C$.
            \end{proof}
            \item \label{exr:3.1.6f}(Distributivity) We have $A\cap(B\cup C)=(A\cap B)\cup(A\cap C)$ and $A\cup(B\cap C)=(A\cup B)\cap(A\cup C)$.
            \begin{proof}
                To prove $A\cap(B\cup C)=(A\cap B)\cup(A\cap C)$, Definition \ref{dfn:setEquality} tells us that it will suffice to show that every element $x$ of $A\cap(B\cup C)$ is an element of $(A\cap B)\cup(A\cap C)$ and vice versa. By Definition \ref{dfn:intersection}, $x\in A\cap(B\cup C) \Longrightarrow (x\in A\text{ and }x\in B\cup C)$. By Axiom \ref{axm:pairwiseUnion}, $x\in B\cup C \Longrightarrow x\in B\text{ or }x\in C$. Thus, we know that $x\in A$, and we know that $x\in B$ or $x\in C$ (or both). We divide into two cases. Suppose first that $x\in B$. Since $x\in A$ as well, $x\in A\cap B$ (Definition \ref{dfn:intersection}). This implies by Axiom \ref{axm:pairwiseUnion} that $x\in(A\cap B)\cup(A\cap C)$. Now suppose that $x\in C$. Since $x\in A$ as well, $x\in A\cap C$ (Definition \ref{dfn:intersection}). This implies by Axiom \ref{axm:pairwiseUnion} that $x\in(A\cap B)\cup(A\cap C)$. A similar argument shows that every element of $(A\cap B)\cup(A\cap C)$ lies in $A\cap(B\cup C)$.\par
                The second proof is similar to the first (and similar to all the rest of the proofs written for this exercise thus far). Thus, for the sake of variety, we will do this one entirely symbolically, using the laws of propositional logic from Section \ref{sse:A.4}.
                \begin{align*}
                    x\in A\cup(B\cap C) &\Longrightarrow (x\in A) \vee (x\in B\cap C)\tag*{Axiom \ref{axm:pairwiseUnion}}\\
                    &\Longrightarrow (x\in A) \vee ((x\in B) \wedge (x\in C))\tag*{Definition \ref{dfn:intersection}}\\
                    &\Longrightarrow ((x\in A) \wedge (x\in B)) \vee ((x\in A) \wedge (x\in C))\tag*{Distributive law for disjunction}\\
                    &\Longrightarrow (x\in A\cap B) \vee (x\in A\cap C)\tag*{Definition \ref{dfn:intersection}}\\
                    &\Longrightarrow x\in (A\cap B)\cup(A\cap C)\tag*{Axiom \ref{axm:pairwiseUnion}}
                \end{align*}
                A similar argument works in reverse.
            \end{proof}
            \item \label{exr:3.1.6g}(Partition) We have $A\cup(X\setminus A)=X$ and $A\cap(X\setminus A)=\emptyset$.
            \begin{proof}
                To prove $A\cup(X\setminus A)=X$, Definition \ref{dfn:setEquality} tells us that it will suffice to show that every element $x$ of $A\cup(X\setminus A)$ is an element of $X$ and vice versa. Suppose $x\in A\cup(X\setminus A)$. Then by Axiom \ref{axm:pairwiseUnion}, $x\in A$ or $x\in X\setminus A$. We divide into two cases. Suppose first that $x\in A$. Since $A\subseteq X$, Definition \ref{dfn:subsets} asserts that $x\in X$. Now suppose that $x\in X\setminus A$. This implies by Definition \ref{dfn:differenceSets} that $x\in\{y\in X:y\notin A\}$, which means by Axiom \ref{axm:specification} that $x\in X$ (and $x\notin A$, but that's not relevant). On the other hand, suppose $x\in X$. Naturally, either $x\in A$ or $x\notin A$ ($x\in A$ is false). If $x\in A$, then $x\in A\cup(X\setminus A)$ (Axiom \ref{axm:pairwiseUnion}). If $x\notin A$, since $x\in X$ as well, Axiom \ref{axm:specification} asserts that $x\in\{x\in X:x\notin A\}$, which, by Definition \ref{dfn:differenceSets}, implies that $x\in X\setminus A$. This, in turn, implies that $x\in A\cup(X\setminus A)$ (Axiom \ref{axm:pairwiseUnion}).\par
                To prove $A\cap(X\setminus A)=\emptyset$, it suffices to prove that for every object $x$, we have $x\notin A\cap(X\setminus A)$ (because of the uniqueness of the empty set). Suppose for the sake of contradiction that $x\in A\cap(X\setminus A)$ for some object $x$. By Definition \ref{dfn:intersection}, $x\in A$ and $x\in X\setminus A$. By Definition \ref{dfn:differenceSets}, $x\in\{y\in X:y\notin A\}$. By Axiom \ref{axm:specification}, $x\in X$ and $x\notin A$. But this contradicts the previously derived fact that $x\in A$. Therefore, $x\notin A\cap(X\setminus A)$ for all objects $x$.
                % To prove $A\cap(X\setminus A)=\emptyset$, Definition \ref{dfn:setEquality} tells us that it will suffice to show that every element $x$ of $A\cap(X\setminus A)$ is an element of $\emptyset$ and vice versa. First off, "every element $x\in\emptyset$ is an element of $A\cap(X\setminus A)$" is vacuously true (by Axiom \ref{axm:emptyset}, there exists no $x\in\emptyset$). In the other direction, suppose for the sake of contradiction that $x\in A\cap(X\setminus A)$ for some object $x$. By Definition \ref{dfn:intersection}, $x\in A$ and $x\in X\setminus A$. By Definition \ref{dfn:differenceSets}, $x\in\{y\in X:y\notin A\}$. By Axiom \ref{axm:specification}, $x\in X$ and $x\notin A$. But this contradicts the previously derived fact that $x\in A$. Therefore, $x\notin A\cap(X\setminus A)$ for all objects $x$. Thus, the statement "every element $x\in A\cap(X\setminus A)$ is an element of $\emptyset$" is vacuously true (there exists no $x\in A\cap(X\setminus A)$).
            \end{proof}
            \item \label{exr:3.1.6h}(De Morgan laws) We have $X\setminus(A\cup B)=(X\setminus A)\cap(X\setminus B)$ and $X\setminus(A\cap B)=(X\setminus A)\cup(X\setminus B)$.
            \begin{proof}
                To prove $X\setminus(A\cup B)=(X\setminus A)\cap(X\setminus B)$, Definition \ref{dfn:setEquality} tells us that it will suffice to show that every element $x$ of $X\setminus(A\cup B)$ is an element of $(X\setminus A)\cap(X\setminus B)$ and vice versa. Suppose that $x\in X\setminus(A\cup B)$. Then by Definition \ref{dfn:differenceSets}, $x\in\{y\in X:y\notin A\cup B\}$. By Axiom \ref{axm:specification}, $x\in X$ and $x\notin A\cup B$. By the inverse of Axiom \ref{axm:pairwiseUnion} (which is a valid assertion since Axiom \ref{axm:pairwiseUnion} asserts the logical equivalence of "$x\in A\cup B$" and "$x\in A$ and $x\in B$"), $x\notin A\cup B \Longrightarrow (x\notin A\text{ and }x\notin B)$. By Axiom \ref{axm:specification} and Definition \ref{dfn:differenceSets}, $(x\in X\text{ and }x\notin A) \Longrightarrow x\in\{y\in X:y\notin A\} \Longrightarrow x\in X\setminus A$. Similarly, $(x\in X\text{ and }x\notin B) \Longrightarrow x\in\{y\in X:y\notin B\} \Longrightarrow x\in X\setminus B$. Thus, by Definition \ref{dfn:intersection}, $x\in(X\setminus A)\cap(X\setminus B)$. A similar, reversed argument will work in the other direction.\par
                To prove $X\setminus(A\cap B)=(X\setminus A)\cup(X\setminus B)$, Definition \ref{dfn:setEquality} tells us that it will suffice to show that every element $x$ of $X\setminus(A\cap B)$ is an element of $(X\setminus A)\cup(X\setminus B)$ and vice versa. By Definition \ref{dfn:differenceSets} and Axiom \ref{axm:specification}, $x\in X\setminus(A\cap B) \Longrightarrow x\in\{y\in X:y\notin A\cap B\} \Longrightarrow (x\in X\text{ and }x\notin A\cap B)$. By the inverse of Definition \ref{dfn:intersection} (which, again, is a valid assertion since $x\in A\cap B \leftrightarrow ((x\in A) \wedge (x\in B))$), $x\notin A\cap B \Longrightarrow (x\notin A\text{ or }x\notin B)$. We divide into two cases. Suppose $x\notin A$. Then by Axiom \ref{axm:specification} and Definition \ref{dfn:differenceSets}, $(x\in X\text{ and }x\notin A) \Longrightarrow x\in\{y\in X:y\notin A\} \Longrightarrow x\in X\setminus A$. Thus, by Axiom \ref{axm:pairwiseUnion}, $x\in(X\setminus A)\cup(X\setminus B)$. Now suppose $x\notin B$. Then by Axiom \ref{axm:specification} and Definition \ref{dfn:differenceSets}, $(x\in X\text{ and }x\notin B) \Longrightarrow x\in\{y\in X:y\notin B\} \Longrightarrow x\in X\setminus B$. Thus, by Axiom \ref{axm:pairwiseUnion}, $x\in(X\setminus A)\cup(X\setminus B)$.
            \end{proof}
        \end{enumerate}
    \end{prp}
    \item \label{exr:3.1.7}\marginnote{7/17:}Let $A,B,C$ be sets. Show that $A\cap B\subseteq A$ and $A\cap B\subseteq B$. Furthermore, show that $C\subseteq A$ and $C\subseteq B$ iff $C\subseteq A\cap B$. In a similar spirit, show that $A\subseteq A\cup B$ and $B\subseteq A\cup B$, and furthermore that $A\subseteq C$ and $B\subseteq C$ iff $A\cup B\subseteq C$.
    \begin{proof}
        To prove that $A\cap B\subseteq A$ and $A\cap B\subseteq B$, Definition \ref{dfn:subsets} tells us that it will suffice to show that every element of $A\cap B$ is an element of $A$ and $B$. By Definition \ref{dfn:intersection}, $x\in A\cap B \Longrightarrow (x\in A\text{ and }x\in B)$.\par
        \marginnote{7/19:}To prove that $C\subseteq A$ and $C\subseteq B$ iff $C\subseteq A\cap B$, it will suffice to show that $C\subseteq A$ and $C\subseteq B$ imply $C\subseteq A\cap B$ and vice versa. Suppose first that $C\subseteq A$ and $C\subseteq B$. Then by two applications of Definition \ref{dfn:subsets}, $x\in C \Longrightarrow x\in A$ and $x\in C \Longrightarrow x\in B$. By Definition \ref{dfn:intersection}, $(x\in A\text{ and }x\in B) \Longrightarrow x\in A\cap B$. Thus, every element $x$ of $C$ is an element of $A\cap B$, so by Definition \ref{dfn:subsets}, $C\subseteq A\cap B$. Now suppose that $C\subseteq A\cap B$. Then by Definition \ref{dfn:subsets}, $x\in C \Longrightarrow x\in A\cap B$. By Definition \ref{dfn:intersection}, $x\in A\cap B \Longrightarrow (x\in A\text{ and }x\in B)$. Thus, every element $x$ of $C$ is an element of $A$ and $B$, so by two applications of Definition \ref{dfn:subsets}, $C\subseteq A$ and $C\subseteq B$.\par
        To prove that $A\subseteq A\cup B$ and that $B\subseteq A\cup B$, Definition \ref{dfn:subsets} tells us that it will suffice to show that every element $x$ of $A$ is an element of $A\cup B$ and that every element $y$ of $B$ is an element of $A\cup B$, respectively. By two applications of Axiom \ref{axm:pairwiseUnion}, $x\in A \Longrightarrow x\in A\cup B$, and $y\in B \Longrightarrow y\in A\cup B$.\par
        To prove that $A\subseteq C$ and $B\subseteq C$ iff $A\cup B\subseteq C$, it will suffice to show that $A\subseteq C$ and $B\subseteq C$ imply $A\cup B\subseteq C$ and vice versa. Suppose first that $A\subseteq C$ and $B\subseteq C$. By Axiom \ref{axm:pairwiseUnion}, $x\in A\cup B \Longrightarrow (x\in A\text{ or }x\in B)$. We divide into two cases. If $x\in A$, then Definition \ref{dfn:subsets} ensures that $x\in C$. If $x\in B$, then Definition \ref{dfn:subsets} similarly ensures that $x\in C$. Thus, either way, $x\in A\cup B \Longrightarrow x\in C$, so by Definition \ref{dfn:subsets}, $A\cup B\subseteq C$. Now suppose that $A\cup B\subseteq C$. By Axiom \ref{axm:pairwiseUnion} followed by Definition \ref{dfn:subsets}, $x\in A \Longrightarrow x\in A\cup B \Longrightarrow x\in C$. A similar argument shows that $x\in B \Longrightarrow x\in C$. Thus, $A\subseteq C$ and $B\subseteq C$.
    \end{proof}
    \item \label{exr:3.1.8}Let $A,B$ be sets. Prove the \textbf{absorption laws} $A\cap(A\cup B)=A$ and $A\cup(A\cap B)=A$.
    \begin{proof}
        By Exercise \ref{exr:3.1.7}, $A\subseteq A\cup B$. By Exercise \ref{exr:3.1.6b} (with $A=A$ and $X=A\cup B$), $A\cap(A\cup B)=A$.\par
        By Exercise \ref{exr:3.1.7}, $A\cap B\subseteq A$. By Exercise \ref{exr:3.1.6b} (with $A=A\cap B$ and $X=A$), $A\cup(A\cap B)=A$.
    \end{proof}
    \item \label{exr:3.1.9}Let $A,B,X$ be sets such that $A\cup B=X$ and $A\cap B=\emptyset$. Show that $A=X\setminus B$ and $B=X\setminus A$.
    \begin{proof}
        To prove $A=X\setminus B$, Definition \ref{dfn:setEquality} tells us that it will suffice to show that every element $x$ of $A$ is an element of $X\setminus B$ and vice versa. Suppose first that $x\in A$. Then by Axiom \ref{axm:pairwiseUnion}, $x\in A\cup B$. Since $A\cup B=X$, we have by Definition \ref{dfn:setEquality} that $x\in X$. Now suppose for the sake of contradiction that $x\in B$. Since $x\in A$ as well, by Definition \ref{dfn:intersection}, $x\in A\cap B$. But $A\cap B=\emptyset$, so by Definition \ref{dfn:setEquality}, $x\in\emptyset$, which contradicts Axiom \ref{axm:emptyset}. Therefore, $x\notin B$. Since $x\in X$ and $x\notin B$, by Definition \ref{dfn:differenceSets}, $x\in X\setminus B$. Now suppose that $x\in X\setminus B$. Then by Definition \ref{dfn:differenceSets}, $x\in X$ and $x\notin B$. By Definition \ref{dfn:setEquality}, $x\in A\cup B$. Thus, by Axiom \ref{axm:pairwiseUnion}, $x\in A$ or $x\in B$. Since $x\notin B$, $x$ must be an element of $A$. A similar argument shows that $B=X\setminus A$.
    \end{proof}
    \item \label{exr:3.1.10}\marginnote{7/22:}Let $A$ and $B$ be sets. Show that the three sets $A\setminus B$, $A\cap B$, and $B\setminus A$ are disjoint, and that their union is $A\cup B$.
    \begin{proof}
        To show that $A\setminus B$, $A\cap B$, and $B\setminus A$ are disjoint, we must show that
        \begin{equation*}
            (A\setminus B)\cap(A\cap B) = (A\setminus B)\cap(B\setminus A)
            = (A\cap B)\cap(B\setminus A)
            = \emptyset
        \end{equation*}
        We may do this by showing that no object $x$ is an element of any of the left three sets above (because of the uniqueness of the empty set). We do this via three contradiction proofs, as follows.\par
        Suppose for the sake of contradiction that $x\in(A\setminus B)\cap(A\cap B)$. Then by Definition \ref{dfn:intersection}, $x\in A\setminus B$ and $x\in A\cap B$. Since $x\in A\setminus B$, Definition \ref{dfn:differenceSets} tells us that $x\notin B$ (and $x\in A$). But since $x\in A\cap B$, Definition \ref{dfn:intersection} tells us that $x\in B$ (and $x\in A$), a contradiction.\par
        A similar argument to the above can handle $(A\cap B)\cap(B\setminus A)$.\par
        Suppose for the sake of contradiction that $x\in (A\setminus B)\cap(B\setminus A)$. Then by Definition \ref{dfn:intersection}, $x\in A\setminus B$ and $x\in B\setminus A$. By the first statement, $x\in A$ and $x\notin B$, while by the second statement, $x\in B$ and $x\notin A$, two contradictions.\par
        We now turn our attention to proving the following.
        \begin{equation*}
            A\cup B = (A\setminus B)\cup(A\cap B)\cup(B\setminus A)
        \end{equation*}
        We can actually prove this solely on the basis of prior results (and one additional lemma).
        \begin{lem}\label{lem:unionMinus}
            Let $A$ and $B$ be sets. Show that $(A\cup B)\setminus A=B\setminus A$.
            \begin{proof}
                To prove $(A\cup B)\setminus A=B\setminus A$, Definition \ref{dfn:setEquality} tells us that it will suffice to show that every element $x$ of $(A\cup B)\setminus A$ is an element of $B\setminus A$ and vice versa. Suppose first that $x\in(A\cup B)\setminus A$. Then by Definition \ref{dfn:differenceSets}, $x\in A\cup B$ and $x\notin A$. Since $x\in A\cup B$, by Axiom \ref{axm:pairwiseUnion}, $x\in A$ or $x\in B$. But $x\notin A$, so $x$ must be an element of $B$. Having established that $x\in B$ and $x\notin A$, Definition \ref{dfn:differenceSets} tells us that $x\in B\setminus A$. Now suppose that $x\in B\setminus A$. Then by Definition \ref{dfn:differenceSets}, $x\in B$ and $x\notin A$. Since $x\in B$, by Axiom \ref{axm:pairwiseUnion}, $x\in A\cup B$. Consequently, by Definition \ref{dfn:differenceSets}, $x\in(A\cup B)\setminus A$.
            \end{proof}
        \end{lem}
        Now we can begin. By Exercise \ref{exr:3.1.7}, $A\cap B\subseteq A$ and $A\subseteq A\cup B$. This implies by Proposition \ref{prp:subsetTransitive} that $A\cap B\subseteq A\cup B$. Thus,
        \begin{align*}
            A\cup B &= (A\cap B)\cup((A\cup B)\setminus(A\cap B))\tag*{Exercise \ref{exr:3.1.6g}}\\
            &= (A\cap B)\cup((A\cup B)\setminus A)\cup((A\cup B)\setminus B)\tag*{Exercise \ref{exr:3.1.6h}}\\
            &= (A\cap B)\cup(B\setminus A)\cup(A\setminus B)\tag*{Lemma \ref{lem:unionMinus}}
        \end{align*}
    \end{proof}
    \item \label{exr:3.1.11}Show that the axiom of replacement implies the axiom of specification.
    \begin{proof}
        Let $P(x,y)$ be the statement "$y=x$ and $P(y)$ is true." Then by Axiom \ref{axm:replacement}, there exists a set
        \begin{align*}
            \{y:P(x,y)\text{ is true for some }x\in A\} &= \{y:y=x\text{ and }P(y)\text{ is true for some }x\in A\}\\
            &= \{y:y\in A\text{ and }P(y)\text{ is true}\}\\
            &= \{y\in A:P(y)\text{ is true}\}
        \end{align*}
        Moreover,
        \begin{align*}
            z\in \{y\in A:P(y)\text{ is true}\} &\Longrightarrow z\in\{y:P(x,y)\text{ is true for some }x\in A\}\\
            &\Longrightarrow P(x,z)\text{ is true for some }x\in A\\
            &\Longrightarrow z=x\text{ and }P(z)\text{ is true for some }x\in A\\
            &\Longrightarrow z\in A\text{ and }P(z)\text{ is true}
        \end{align*}
        The above logic also works in reverse. Thus, all the tenets of Axiom \ref{axm:specification} have been shown to follow from Axiom \ref{axm:replacement} (proof modified from \cite{bib:ReplacementToSpecification}).
    \end{proof}
\end{enumerate}



\section{Russell's Paradox}
\begin{itemize}
    \item Suppose that we could unify the multitude of axioms in Section \ref{sse:3.1} into a single axiom. The following would be a good candidate (in fact, it implies the majority of the Section \ref{sse:3.1} axioms --- see Exercise \ref{exr:3.2.1}).
    \begin{axm}[Universal specification]\label{axm:universalSpecification}
        (Dangerous!) Suppose for every object $x$ we have a property $P(x)$ pertaining to $x$ (so that for every $x$, $P(x)$ is either a true statement or a false statement). Then there exists a set $\{x:P(x)\text{ is true}\}$ such that for every object $y$,
        \begin{equation*}
            y\in\{x:P(x)\text{ is true}\} \Longleftrightarrow P(y)\text{ is true}
        \end{equation*}
    \end{axm}
    \item \emph{Also known as} \textbf{axiom of comprehension}.
    \item Basically, Axiom \ref{axm:universalSpecification} asserts that \dq{every property corresponds to a set}{46}
    \item Unfortunately, Axiom \ref{axm:universalSpecification} cannot be introduced into set theory because it creates a logical contradiction known as \textbf{Russell's paradox}.
    \begin{itemize}
        \item Discovered by philosopher and logician Bertrand Russell (1872-1970) in 1901.
    \end{itemize}
    \item \textbf{Russell's paradox}: \dq{Let $P(x)$ be the statement\dots "$x$ is a set, and $x\notin x$"; i.e., $P(x)$ is true only when $x$ is a set which does not contain itself. For instance, $P(\{2,3,4\})$ is true, since the set $\{2,3,4\}$ is not one of the three elements 2, 3, 4 of $\{2,3,4\}$. On the other hand, if we let $S$ be the set of all sets (which we would know to exist from the axiom of universal specification), then since $S$ is itself a set, it is an element of $S$, and so $P(S)$ is false. Now use the axiom of universal specification to create the set
    \begin{equation*}
        \Omega := \{x:P(x)\text{ is true}\} = \{x:x\text{ is a set and }x\notin x\}
    \end{equation*}
    i.e., the set of all sets which do not contain themselves. Now ask the question: does $\Omega$ contain itself, i.e. is $\Omega\in\Omega$? If $\Omega$ did contain itself, then by definition this means that $P(\Omega)$ is true, i.e., $\Omega$ is a set and $\Omega\notin\Omega$. On the other hand, if $\Omega$ did not contain itself, then $P(\Omega)$ would be true, and hence $\Omega\in\Omega$. Thus in either case we have both $\Omega\in\Omega$ and $\Omega\notin\Omega$, which is absurd}{46-47}
    \begin{itemize}
        \item To clarify the last point: Is $\Omega\in\Omega$? Suppose $\Omega\in\Omega$. Then since $\Omega$ contains only sets for which $P(x)$ is true, $P(\Omega)$ must be true. But this implies, by the definition of $P(x)$, that $\Omega\notin\Omega$. Similarly, suppose $\Omega\notin\Omega$. Then since "$\Omega$ is a set and $\Omega\notin\Omega$" is a true statement, $P(\Omega)$ must be true. But this implies, since $\Omega$ contains all sets for which $P(x)$ is true, that $\Omega\in\Omega$. In either case, we have both $\Omega\in\Omega$ and $\Omega\notin\Omega$ (contradictions).
    \end{itemize}
    \item \marginnote{7/23:}The main problem highlighted by Russell's paradox is that Axiom \ref{axm:universalSpecification} allows for the creation of sets that are too "large," i.e., sets that contain themselves, which is somewhat silly.
    \begin{itemize}
        \item This problem can be informally resolved by creating a hierarchy: primitive objects are below primitive sets (which only contain primitive objects), are below second-level sets (which only contain primitive objects and primitive sets), and so on and so forth. Formalizing this notion is complicated and will not be explored further here.
        \item Note that in pure set theory, there are no primitive objects --- only one primitive set (the empty set).
    \end{itemize}
    \item To avoid the complications of Russell's paradox, we create a new axiom.
    \begin{axm}[Regularity]\label{axm:regularity}
        If $A$ is a non-empty set, then there is at least one element $x$ of $A$ which is either not a set or is disjoint from $A$.
    \end{axm}
    \item \emph{Also known as} \textbf{axiom of foundation}.
    \item Axiom \ref{axm:regularity} asserts that \dq{at least one of the elements of $A$ is so low on the hierarchy of objects that it does not contain any of the other elements of $A$}{48} It also asserts that sets may not contain themselves (see Exercise \ref{exr:3.2.2}).
    \item As a less intuitive axiom, one might question whether or not Axiom \ref{axm:regularity} is needed. In fact, it is not necessary for the purposes of doing analysis, as all sets considered in analysis are very low on the hierarchy. However, it is necessary to perform more advanced set theory, so \cite{bib:AnalysisI} included it for the sake of completeness.
\end{itemize}


\subsection*{Exercises}
\begin{enumerate}[ref={\thesection.\arabic*}]
    \item \label{exr:3.2.1}\marginnote{7/22:}Show that the universal specification axiom, Axiom \ref{axm:universalSpecification}, if assumed to be true, would imply Axioms \ref{axm:emptyset}, \ref{axm:singletonPair}, \ref{axm:pairwiseUnion}, \ref{axm:specification}, and \ref{axm:replacement}. (If we assume that all natural numbers are objects, we also obtain Axiom \ref{axm:infinity}.) Thus, this axiom, if permitted, would simplify the foundations of set theory tremendously (and can be viewed as one basis for an intuitive model of set theory known as "naive set theory"). Unfortunately, as we have seen, Axiom \ref{axm:universalSpecification} is "too good to be true!"
    \begin{proof}
        Axiom \ref{axm:emptyset}: Let $P(x)$ be a false statement for all objects $x$. By Axiom \ref{axm:universalSpecification}, there exists a set $\{x:P(x)\text{ is true}\}$, which contains no elements. (Suppose for the sake of contradiction that $y\in\{x:P(x)\text{ is true}\}$ for some object $y$. Then $P(y)$ is true. But $P(y)$ is false by definition, a contradiction. Therefore, $y\notin\{x:P(x)\text{ is true}\}$ for all objects $y$.) Incidentally, that contradiction proof solidifies the symbolic statement of Axiom \ref{axm:emptyset}. Lastly, this set may be denoted $\emptyset$ or $\{\}$.\par
        Axiom \ref{axm:singletonPair}: Let $P(x)$ be the statement $x=a$. By Axiom \ref{axm:universalSpecification}, there exists a set $\{x:P(x)\text{ is true}\}=\{x:x=a\}=\{a\}$ whose element is $a$. For every object $y$, we have $y\in\{a\}$ iff $P(y)$ is true, i.e., iff $y=a$. This set may be called the \textbf{singleton set} whose element is $a$. Now let $P(x)$ be the statement "$x=a$ or $x=b$." By Axiom \ref{axm:universalSpecification}, there exists a set $\{x:P(x)\text{ is true}\}=\{x:x=a\text{ or }x=b\}=\{a,b\}$ whose elements are $a$ and $b$. For every object $y$, we have $y\in\{a,b\}$ iff $P(y)$ is true, i.e., iff $y=a$ or $y=b$. This set may be called the \textbf{pair set} formed by $a$ and $b$.\par
        Axiom \ref{axm:pairwiseUnion}: Let $A,B$ be sets. Let $P(x)$ be the statement "$x\in A$ or $x\in B$." By Axiom \ref{axm:universalSpecification}, there exists a set $\{x:P(x)\text{ is true}\}=\{x:x\in A\text{ or }x\in B\}$. This set may be called the \textbf{union} $A\cup B$ of $A$ and $B$. By Axiom \ref{axm:universalSpecification}, $y\in A\cup B$ iff $P(y)$ is true, i.e., iff $y\in A$ or $y\in B$. Thus, $A\cup B$ is clearly a set whose elements consist of all the elements which belong to $A$ or $B$ or both.\par
        Axiom \ref{axm:replacement}: Let $A$ be a set. Let $P(y)$ be the statement "$P(x,y)$ is true for some $x\in A$," where $P(x,y)$ is a statement pertaining to $x$ and $y$ such that for each $x\in A$, there is at most one $y$ for which $P(x,y)$ is true. By Axiom \ref{axm:universalSpecification}, there exists a set $\{y:P(y)\text{ is true}\}=\{y:P(x,y)\text{ is true for some }x\in A\}$. By Axiom \ref{axm:universalSpecification}, $z\in\{y:P(x,y)\text{ is true for some }x\in A\}$ iff $P(z)$ is true, i.e., iff $P(x,z)$ is true for some $x\in A$.\par
        Axiom \ref{axm:specification}: Implied by the axiom of replacement (see Exercise \ref{exr:3.1.11}).
    \end{proof}
    \item \label{exr:3.2.2}\marginnote{7/23:}Use the axiom of regularity (and the singleton set axiom) to show that if $A$ is a set, then $A\notin A$. Furthermore, show that if $A$ and $B$ are two sets, then either $A\notin B$ or $B\notin A$ (or both).
    \begin{proof}
        % Let $A$ be a set. Then by Axiom \ref{axm:setsAreObjects}, $A$ is an object. Thus, by Axiom \ref{axm:singletonPair}, there exists a set $\{A\}$ whose only element is $A$. Now by Axiom \ref{axm:regularity}, there exists an element of $\{A\}$ (which must be $A$, as referenced above) that is either not a set or is disjoint from $\{A\}$. Since $A$ is a set, we have that $A$ is disjoint from $\{A\}$, i.e., $A\cap\{A\}=\emptyset$. By Axiom \ref{axm:emptyset}, $x\notin A\cap\{A\}$ for all objects $x$. This implies by Definition \ref{dfn:intersection} that $x\notin A$ or $x\notin\{A\}$. But $x\notin\{A\}$ implies that $x\neq A$, so we have either $x\neq A$ or $x\notin A$. Therefore, if $x=A$, then $x\notin A$, implying that $A\notin A$.\par
        Suppose for the sake of contradiction that $A$ is a set and $A\in A$. By Axioms \ref{axm:setsAreObjects} and \ref{axm:singletonPair}, $A\in\{A\}$. Since $A\in A$ and $A\in\{A\}$, Definition \ref{dfn:intersection} tells us that $A\in A\cap\{A\}$. Since there exists an object $x$ (namely $A$) such that $x\in A\cap\{A\}$, by Axiom \ref{axm:emptyset} and Definition \ref{dfn:setEquality}, $A\cap\{A\}\neq\emptyset$. Thus, we have $A$ is a set and $A\cap\{A\}\neq\emptyset$. But by Axiom \ref{axm:regularity}, as the only element of $\{A\}$, $A$ must either not be a set or satisfy $A\cap\{A\}=\emptyset$, a contradiction. Therefore, $A$ is not a set or $A\notin A$. Thus, if $A$ is a set, then $A\notin A$.\par
        Suppose for the sake of contradiction that for two sets $A,B$, $A\in B$ and $B\in A$. By Axioms \ref{axm:setsAreObjects} and \ref{axm:singletonPair}, there exists a set $\{A,B\}$ whose only elements are $A$ and $B$. Since $A\in B$ and $A\in\{A,B\}$, Definition \ref{dfn:intersection}, tells us that $A\in B\cap\{A,B\}$. Since there exists an object $x$ (namely $A$) such that $x\in B\cap\{A,B\}$, by Axiom \ref{axm:emptyset} and Definition \ref{dfn:setEquality}, $B\cap\{A,B\}\neq\emptyset$. By a similar argument, $A\cap\{A,B\}\neq\emptyset$. Thus, we have $A,B$ are sets, $B\cap\{A,B\}\neq\emptyset$, and $A\cap\{A,B\}\neq\emptyset$. But by Axiom \ref{axm:regularity}, an element $x$ of $\{A,B\}$ (namely $A$ or $B$) must either not be a set or satisfy $x\cap\{A,B\}=\emptyset$, a contradiction. Therefore, $A\notin B$ or $B\notin A$.
    \end{proof}
    \item \label{exr:3.2.3}Show (assuming the other axioms of set theory) that the universal specification axiom, Axiom \ref{axm:universalSpecification}, is equivalent to an axiom postulating the existence of a "universal set" $\Omega$ consisting of all objects (i.e., for all objects $x$, we have $x\in\Omega$). In other words, if Axiom \ref{axm:universalSpecification} is true, then a universal set exists, and conversely, if a universal set exists, then Axiom \ref{axm:universalSpecification} is true. (This may explain why Axiom \ref{axm:universalSpecification} is called the axiom of \emph{universal} specification). Note that if a universal set $\Omega$ existed, then we would have $\Omega\in\Omega$ by Axiom \ref{axm:setsAreObjects}, contradicting Exercise \ref{exr:3.2.2}. Thus, the axiom of foundation specifically rules out the axiom of universal specification.
    \begin{proof}
        Suppose Axiom \ref{axm:universalSpecification} is true. Let $P(x)$ be a true statement for all objects $x$. Then there exists a set $\Omega:=\{x:P(x)\text{ is true}\}$, and we have $y\in\Omega$ iff $P(y)$ is true, i.e., iff $y$ is an object, i.e., for all objects $y$. Therefore, a universal set exists. Now suppose that a universal set $\Omega$ exists. By Axiom \ref{axm:specification}, there exists a set $\{x\in\Omega:P(x)\text{ is true}\}$ for some property $P(x)$ pertaining to $x$ (note that this implies that $P(x)$ pertains to all $x$) and $y\in\{x\in\Omega:P(x)\text{ is true}\}$ iff $y\in\Omega$ and $P(y)$ is true. Since $x\in\Omega$ and $y\in\Omega$ are, by the definition of $\Omega$, always true, we have $y\in\{x:P(x)\text{ is true}\}$ iff $P(y)$ is true. Therefore, Axiom \ref{axm:universalSpecification} is true.
    \end{proof}
\end{enumerate}



\section{Functions}
\begin{itemize}
    \item For analysis, we need not just the notion of a set but the notion of a function from one set to another.
    \begin{dfn}[Functions]\label{dfn:functions}
        Let $X,Y$ be sets, and let $P(x,y)$ be a property pertaining to an object $x\in X$ and an object $y\in Y$, such that for every $x\in X$, there is exactly one $y\in Y$ for which $P(x,y)$ is true (this is sometimes known as the \textbf{vertical line test}). Then we define the \textbf{function} $f:X\to Y$ defined by $P$ on the \textbf{domain} $X$ and \textbf{range} $Y$ to be the object which, given any \textbf{input} $x\in X$, assigns an output $f(x)\in Y$, defined to be the unique object $f(x)$ for which $P(x,f(x))$ is true. Thus, for any $x\in X$ and $y\in Y$,
        \begin{equation*}
            y = f(x) \Longleftrightarrow P(x,y)\text{ is true}
        \end{equation*}
    \end{dfn}
    \item \emph{Also known as} \textbf{maps}, \textbf{transformations}, and \textbf{morphisms}.
    \begin{itemize}
        \item Note, however, that a morphism \dq{refers to a more general class of object, which may or may not correspond to actual functions, depending on the context}{49}
    \end{itemize}
    \item Functions obey the axiom of substitution.
    \begin{itemize}
        \item Note that equal inputs imply equal outputs, but unequal inputs do not necessary ensure unequal outputs.
    \end{itemize}
    \item It can be proven that this notion of equality is reflexive, symmetric, and transitive (see Exercise \ref{exr:3.3.1}).
    \item We can now formally define the increment function: Let $X=\N$, $Y=\N$, and $P(x,y)$ be the property that $y=x\pplus$. By Axiom \ref{axm:successorDistinctness}, for each $x\in\N$, there is exactly one $y$ for which $P(x,y)$ is true. Thus, we can define the increment function $f:\N\to\N$ so that $f(x)=x\pplus$ for all $x$. While we cannot define a \emph{decrement} function $g:\N\to\N$ ($0\neq n\pplus$ for any $n\in\N$ by Axiom \ref{axm:0NotSuccessor}), we can define a decrement function $g:\N\setminus\{0\}\to\N$ (by Lemma \ref{lem:backwardsIncrement}).
    \item Informally: Note that while we cannot define a square root function $\sqrt{}:\R\to\R$, we can define the square root function $\sqrt{}:[0,+\infty)\to[0,+\infty)$.
    \item Functions can be defined \textbf{explicitly} or \textbf{implicitly}.
    \begin{itemize}
        \item Explicit definitions: Specify the domain, range, and how one generates the output $f(x)$ from each input.
        \begin{itemize}
            \item For example, the increment function $f$ could be defined explicitly by saying that the domain and range of $f$ are equal to $\N$, and $f(x):=x\pplus$ for all $x\in\N$.
        \end{itemize}
        \item Implicit definitions: Specify what property $P(x,y)$ links the input $x$ with the output $f(x)$.
        \begin{itemize}
            \item For example, the square root function $\sqrt{}$ was defined implicitly by the relation $(\sqrt{x})^2=x$.
            \item Note that implicit definitions are only valid if we know that for every input, there is only one output that obeys the implicit relation.
        \end{itemize}
    \end{itemize}
    \item Often the domain and range are not specified for the sake of brevity.
    \begin{itemize}
        \item For example, we could refer to the increment function $f$ as "the function $f(x):=x\pplus$," "the function $x\mapsto x\pplus$," "the function $x\pplus$," or the extremely abbreviated "$\pplus$."
        \item Note, however, that too much abbreviation can be dangerous, omitting valuable or even necessary information.
    \end{itemize}
    \item Note that while we now use parentheses to clarify the order of operations and enclose the arguments of functions and properties, the usages should be unambiguous from context.
    \begin{itemize}
        \item For example, if $a$ is a number, then $a(b+c)$ denotes $a\times(b+c)$, but if $a$ is a function, then $a(b+c)$ denotes the output of $a$ when the input is $b+c$.
        \item Note that argument are sometimes subscripted --- \dq{a sequence of natural numbers $a_0,a_1,a_2,a_3,\dots$ is, strictly speaking, a function from $\N$ to $\N$, but is denoted by $n\mapsto a_n$ rather than $n\mapsto a(n)$}{51}
    \end{itemize}
    \item Note that functions are not sets and sets are not functions (no $x\in f$ and $A:X\nrightarrow Y$), but we can start with a function $f:X\to Y$ and construct its \textbf{graph} $\{(x,f(x)):x\in X\}$, which describes the function completely (see Section \ref{sse:3.5} for more).
    \item We now define equality for functions.
    \begin{dfn}[Equality of functions]\label{dfn:functionEquality}
        Two functions $f:X\to Y$, $g:X\to Y$ with the same domain and range are said to be equal, $f=g$, if and only if $f(x)=g(x)$ for all $x\in X$. (If $f(x)$ and $g(x)$ agree for some values of $x$, but not others, then we do not consider $f$ and $g$ to be equal.)
    \end{dfn}
    \item Note that functions can be equal over only a certain domain.
    \begin{itemize}
        \item For example, $x\mapsto x$ and $x\mapsto|x|$ are equal if defined only on the positive real axis, and are not equal if defined on $\R$.
    \end{itemize}
    \item \textbf{Empty function}: The function $f:\emptyset\to X$.
    \begin{itemize}
        \item We need not specify what $f$ does to any input (since there are none), and Definition \ref{dfn:functionEquality} asserts that for each set $X$, there is only one function from $\emptyset$ to $X$.
    \end{itemize}
    \item A fundamental operation of functions is composition.
    \begin{dfn}[Composition]\label{dfn:composition}
        Let $f:X\to Y$ and $g:Y\to Z$ be two functions, such that the range of $f$ is the same set as the domain of $g$. We then define the \textbf{composition} $g\circ f:X\to Z$ of the two functions $g$ and $f$ to be the function defined explicitly by the formula
        \begin{equation*}
            (g\circ f)(x) := g(f(x))
        \end{equation*}
    \end{dfn}
    \item It can be proven that composition obeys the axiom of substitution (see Exercise \ref{exr:3.3.1}).
    \item Composition is not commutative, but it is associative.
    \begin{lem}[Composition is associative]
        Let $f:Z\to W$, $g:Y\to Z$, and $h:X\to Y$ be functions. Then $f\circ(g\circ h)=(f\circ g)\circ h$.
        \begin{proof}
            Since $g\circ h$ is a function from $X$ to $Z$, $f\circ(g\circ h)$ is a function from $X$ to $W$. Similarly, $f\circ g$ is a function from $Y\to W$, and hence $(f\circ g)\circ h$ is a function from $X\to W$. Thus, $f\circ(g\circ h)$ and $(f\circ g)\circ h$ have the same domain and range. In order to check that they are equal, we see from Definition \ref{dfn:functionEquality} that we have to verify that $(f\circ(g\circ h))(x)=((f\circ g)\circ h)(x)$ for all $x\in X$. But by Definition \ref{dfn:composition}, we have
            \begin{align*}
                (f\circ(g\circ h))(x) &= f((g\circ h)(x))\\
                &= f(g(h(x)))\\
                &= (f\circ g)(h(x))\\
                &= ((f\circ g)\circ h)(x)
            \end{align*}
            as desired.
        \end{proof}
    \end{lem}
    \item \marginnote{7/24:}We now define several special types of functions, beginning with the following.
    \begin{dfn}[One-to-one functions]\label{dfn:injective}
        A function $f$ is \textbf{one-to-one} if different elements map to different elements:
        \begin{equation*}
            x \neq x' \Longrightarrow f(x) \neq f(x')
        \end{equation*}
        Equivalently, a function is one-to-one if
        \begin{equation*}
            f(x) = f(x') \Longrightarrow x = x'
        \end{equation*}
    \end{dfn}
    \item \emph{Also known as} \textbf{injective} (function).
    \item Informally: Note that while the function $f:\Z\to\Z$ defined by $f(n):=n^2$ is not one-to-one, the function $g:\N\to\Z$ defined by $g(n):=n^2$ is one-to-one. Thus, being one-to-one can depend not just on the relation, but on the domain.
    \item \textbf{Two-to-one} (function): A function $f:X\to Y$ such that one can find distinct $x$ and $x'$ in the domain 
    $X$ such that $f(x)=f(x')$.
    \item Another special type is given by the following.
    \begin{dfn}[Onto functions]\label{dfn:surjective}
        A function $f$ is \textbf{onto} if $f(X)=Y$, i.e., every element in $Y$ comes from applying $f$ to some element in $X$:
        \begin{center}
            For every $y\in Y$, there exists $x\in X$ such that $f(x)=y$
        \end{center}
    \end{dfn}
    \item \emph{Also known as} \textbf{surjective} (function).
    \item Informally: Note that while the function $f:\Z\to\Z$ defined by $f(n):=n^2$ is not onto, if we define the set $A:=\{n^2:n\in\Z\}$, then the function $g:\Z\to A$ defined by $f(n)=n^2$ is onto. Thus, being onto can depend not just on the relation, but on the range.
    \item Injectivity and surjectivity are rather dual to each other (see Exercises \ref{exr:3.3.2}, \ref{exr:3.3.4}, and \ref{exr:3.3.5}).
    \item \marginnote{7/28:}A third special type is given by the following.
    \begin{dfn}[Bijective functions]\label{dfn:bijective}
        A function $f:X\to Y$ is \textbf{bijective} if it is both one-to-one and onto:
        \begin{center}
            For every $y\in Y$, there is exactly one $x$ such that $f(x)=y$.
        \end{center}
    \end{dfn}
    \pagebreak
    \item \emph{Also known as} \textbf{invertible} (function), \textbf{perfect matching}, \textbf{one-to-one correspondence}\footnote{Not to be confused with the notion of a one-to-one function.}.
    \begin{itemize}
        \item Instead of being denoted $x\mapsto f(x)$, we sometimes use $x\leftrightarrow f(x)$.
    \end{itemize}
    \item Bijectivity of functions:
    \begin{itemize}
        \item $f:\{0,1,2\}\to\{3,4\}$ defined by $f(0):=3$, $f(1):=3$, and $f(2):=4$ is not bijective (fails injectivity).
        \item $f:\{0,1\}\to\{2,3,4\}$ defined by $f(0):=2$ and $f(1):=3$ is not bijective (fails surjectivity).
        \item $f:\{0,1,2\}\to\{3,4,5\}$ defined by $f(0):=3$, $f(1):=4$, and $f(2):=5$ is bijective.
        \item $f:\N\to\N$ defined by $f(n):=n\pplus$ is not bijective (fails surjectivity).
        \item $f:\N\to\N\setminus\{0\}$ defined by $f(n):=n\pplus$ is bijective.
        \item Bijectivity depends on the domain and range, not just the relation.
    \end{itemize}
    \item \textbf{Inverse} (of a function $f:X\to Y$): The function $f^{-1}:Y\to X$ associated with the property $P(y,x)$ defined by $f(x)=y$, i.e., we let $f^{-1}(y)$ be the entity $x$ satisfying $f(x)=y$.
\end{itemize}


\subsection*{Exercises}
\begin{enumerate}[ref={\thesection.\arabic*}]
    \item \label{exr:3.3.1}\marginnote{7/23:}Show that the definition of equality in Definition \ref{dfn:functionEquality} is reflexive, symmetric, and transitive\footnote{Note that since Definition \ref{dfn:functionEquality} should be an axiom (should axiomatize equality and all that that entails for functions), this part of the exercise is silly (see \cite{bib:TaoErrata}).}. Also verify the substitution property: if $f,\tilde{f}:X\to Y$ and $g,\tilde{g}:Y\to Z$ are functions such that $f=\tilde{f}$ and $g=\tilde{g}$, then $g\circ f=\tilde{g}\circ\tilde{f}$.
    \begin{proof}
        Let $f:X\to Y$ be a function. For any object $f(x)\in Y$, we have $f(x)=f(x)$ by the reflexive axiom of equality (see Section \ref{sse:A.7}). Thus, we have $f(x)=f(x)$ for all $f(x)\in Y$, i.e., for all $x\in X$. Therefore, by Definition \ref{dfn:functionEquality}, we have $f=f$.\par
        Let $f:X\to Y$, $g:X\to Y$ be functions, and let $f=g$. By Definition \ref{dfn:functionEquality}, $f(x)=g(x)$ for all $x\in X$. Since equal objects are of the same type (i.e., follow the reflexive axiom of equality), we have $g(x)=f(x)$ for all $x\in X$. Therefore, by Definition \ref{dfn:functionEquality}, we have $g=f$.\par
        Let $f:X\to Y$, $g:X\to Y$, $h:X\to Y$ be functions, $f=g$, and $g=h$. By Definition \ref{dfn:functionEquality}, $f(x)=g(x)$ for all $x\in X$ and $g(x)=h(x)$ for all $x\in X$. Since equal objects are of the same type (i.e., follow the transitive axiom of equality), we have $f(x)=h(x)$ for all $x\in X$. Therefore, by Definition \ref{dfn:functionEquality}, we have $f=h$.\par
        First, we see that both $g\circ f$ and $\tilde{g}\circ\tilde{f}$ are functions from $X$ to $Z$, i.e., have the same domain and range. To show that $g\circ f=\tilde{g}\circ\tilde{f}$, Definition \ref{dfn:functionEquality} tells us that it will suffice to verify that $(g\circ f)(x)=(\tilde{g}\circ\tilde{f})(x)$ for all $x\in X$. To begin, we see from Definition \ref{dfn:functionEquality} that $f(x)=\tilde{f}(x)$ for all $x\in X$, and that $g(y)=\tilde{g}(y)$ for all $y\in Y$. Since $f(x)\in Y$ for all $f(x)$, we have by Definition \ref{dfn:composition} that
        \begin{align*}
            (g\circ f)(x) &= g(f(x))\\
            &= \tilde{g}(f(x))\\
            &= \tilde{g}(\tilde{f}(x))\\
            &= (\tilde{g}\circ\tilde{f})(x)
        \end{align*}
        as desired.
    \end{proof}
    \item \label{exr:3.3.2}\marginnote{7/24:}Let $f:X\to Y$ and $g:Y\to Z$ be functions. Show that if $f$ and $g$ are both injective, then so is $g\circ f$; similarly, show that if $f$ and $g$ are both surjective, then so is $g\circ f$.
    \begin{proof}
        To prove that $g\circ f$ is injective given the injectivity of $f,g$, Definition \ref{dfn:injective} tells us that we have to verify that $x\neq x' \Longrightarrow (g\circ f)(x)\neq(g\circ f)(x')$ for any distinct elements $x,x'$ of $X$. By Definition \ref{dfn:injective}, we have that if $x$ and $x'$ are two distinct elements of $X$ (such that $x\neq x'$), then $f(x)\neq f(x')$. By Definition \ref{dfn:functions}, we know that $f(x)$ and $f(x')$ are both elements of $Y$. Thus, we have by Definition \ref{dfn:injective} that $g(f(x))\neq g(f(x'))$. Therefore, we have by Definition \ref{dfn:composition} that
        \begin{equation*}
            x\neq x' \Longrightarrow f(x)\neq f(x')
            \Longrightarrow g(f(x))\neq g(f(x'))
            \Longrightarrow (g\circ f)(x)\neq(g\circ f)(x')
        \end{equation*}
        as desired.\par
        By Definition \ref{dfn:composition}, we have $g\circ f:X\to Z$. Thus, to prove that $g\circ f$ is surjective given the surjectivity of $f,g$, Definition \ref{dfn:surjective} tells us that we have to verify that for every $z\in Z$, there exists $x\in X$ such that $(g\circ f)(x)=z$. Let $z$ be any element of $Z$. Then by Definition \ref{dfn:surjective} and the surjectivity of $g$, we have $g(y)=z$ for some $y\in Y$. Similarly, we have $f(x)=y$ for that $y$ and for some $x\in X$. Substituting, we have $g(f(x))=z$ for some $x\in X$. Since $g(f(x))=(g\circ f)(x)$ by Definition \ref{dfn:composition}, we have that for any $z\in Z$, there exists $x\in X$ such that $(g\circ f)(x)=z$.
    \end{proof}
    \item \label{exr:3.3.3}\marginnote{7/28:}When is the empty function injective? Surjective? Bijective?
    \begin{proof}
        The empty function is always injective: The statement "all distinct inputs map to distinct outputs" is vacuously true, since there exist no distinct inputs, or any inputs to speak of. The empty function is surjective iff its range is the empty set: In this case, the statement "for every $y\in\emptyset$, there exists $x\in\emptyset$ such that $f(x)=y$" is similarly vacuously true. Since a function is bijective iff it is both injective and surjective, we must take both take a sort of union of the two constraints above: We can conclude that the empty function is bijective iff "always and its range is the empty set" is true, i.e., iff its range is the empty set.
    \end{proof}
    \item \label{exr:3.3.4}\marginnote{7/24:}In this section, we give some cancellation laws for composition. Let $f:X\to Y$, $\tilde{f}:X\to Y$, $g:Y\to Z$, and $\tilde{g}:Y\to Z$ be functions. Show that if $g\circ f=g\circ \tilde{f}$ and $g$ is injective, then $f=\tilde{f}$. Is the same statement true if $g$ is not injective? Show that if $g\circ f=\tilde{g}\circ f$ and $f$ is surjective, then $g=\tilde{g}$. Is the same statement true if $f$ is not surjective?
    \begin{proof}
        To prove that $f=\tilde{f}$, Definition \ref{dfn:functionEquality} tells us that it will suffice to show that $f(x)=\tilde{f}(x)$ for all $x\in X$. By Definition \ref{dfn:functionEquality}, $g\circ f=g\circ\tilde{f}$ implies $(g\circ f)(x)=(g\circ\tilde{f})(x)$ for all $x\in X$. Then by Definition \ref{dfn:composition}, $g(f(x))=g(\tilde{f}(x))$ for all $x\in X$. Now by Definition \ref{dfn:injective} and the fact that $g$ is injective, we know that $g(y)=g(y') \Longrightarrow y=y'$. Therefore, since $g(f(x))=g(\tilde{f}(x))$ for all $x\in X$, we have $f(x)=\tilde{f}(x)$ for all $x\in X$. Note that $f$ is not necessarily equal to $\tilde{f}$ if we drop the condition that $g$ is injective (if $g$ is not one-to-one, then we have $g(y)=g(y')$ for some elements $y$ and $y'$ of $Y$ such that $y\neq y'$. Thus, if $f(x)\neq\tilde{f}(x)$, we may still have $g(f(x))=g(\tilde{f}(x))$).\par
        % To prove that $g=\tilde{g}$, we must verify (by Definition \ref{dfn:functionEquality}) that $g(y)=\tilde{g}(y)$ for all $y\in Y$. By Definition \ref{dfn:functionEquality}, $g\circ f=\tilde{g}\circ f$ implies $(g\circ f)(x)=(\tilde{g}\circ f)(x)$ for all $x\in X$. Then by Definition \ref{dfn:composition}, $g(f(x))=\tilde{g}(f(x))$ for all $x\in X$. Now by Definition \ref{dfn:surjective} and the fact that $f$ is surjective, we know that for every $y\in Y$, there exists some $x\in X$ such that $f(x)=y$. Thus, no element $y\in Y$ is not assigned to an $f(x)$. Therefore, we have $g(f(x))=\tilde{g}(f(x))$ for all $f(x)\in Y$, i.e., for all $y\in Y$.
        Suppose for the sake of contradiction that $g\neq\tilde{g}$. Then by Definition \ref{dfn:functionEquality}, either $g$ and $\tilde{g}$ have a different domain or range, or $g(y)\neq\tilde{g}(y)$ for some $y\in Y$. Since $g$ and $\tilde{g}$ have the same domain and range, we must have $g(y)\neq\tilde{g}(y)$ for some $y\in Y$. Now since $f$ is surjective, by Definition \ref{dfn:surjective}, for every $y\in Y$, there is some $x\in X$ such that $y=f(x)$. Thus, $g(f(x))\neq\tilde{g}(f(x))$ for some $x\in X$. Consequently, by Definition \ref{dfn:composition}, $(g\circ f)(x)\neq(\tilde{g}\circ f)(x)$ for some $x\in X$. But this contradicts Definition \ref{dfn:functionEquality}, which, since $g\circ f=\tilde{g}\circ f$, implies that $(g\circ f)(x)=(\tilde{g}\circ f)(x)$ for all $x\in X$. Note that $g$ is not necessarily equal to $\tilde{g}$ if we drop the condition that $f$ is surjective (if $f$ is not surjective, then, as we've only proven the assertion for all $y=f(x)$, we could have some $y\in Y$ not associated with an $f(x)$ for which $g(y)\neq\tilde{g}(y)$).
    \end{proof}
    \item \label{exr:3.3.5}\marginnote{7/28:}Let $f:X\to Y$ and $g:Y\to Z$ be functions. Show that if $g\circ f$ is injective, then $f$ must be injective. Is it true that $g$ must also be injective? Show that if $g\circ f$ is surjective, then $g$ must be surjective. Is it true that $f$ must also be surjective?
    \begin{proof}
        By Definition \ref{dfn:injective}, to prove that $f$ is injective, it will suffice to show that for all $x\neq x'$, $f(x)\neq f(x')$. Suppose for the sake of contradiction that for some $x\neq x'$, $f(x)=f(x')$. Then by Definition \ref{dfn:functions}, which guarantees that identical inputs map to a single output, we have $g(f(x))=g(f(x'))$. Thus, by Definition \ref{dfn:composition}, we have $(g\circ f)(x)=(g\circ f)(x')$. But by Definition \ref{dfn:injective}, this means that $g\circ f$ is not injective, a contradiction. Therefore, for all $x\neq x'$, $f(x)\neq f(x')$. Note that $g$ need not be injective to prove the above, as this condition was not needed for the above proof to be valid.\par
        By Definition \ref{dfn:surjective}, to prove that $g$ is surjective, we must verify that for every $z\in Z$, there exists $y\in Y$ such that $g(y)=z$. Let $z$ be any element of $Z$. Since we know that $g\circ f$ is surjective, Definition \ref{dfn:surjective} tells us that there exists an element $x$ of $X$ such that $(g\circ f)(x)=z$. By Definition \ref{dfn:composition}, we have $g(f(x))=z$. Now by Definition \ref{dfn:functions}, we know that $f(x)=y$ for some element $y\in Y$. Thus, $g(y)=z$ for some $y\in Y$. Therefore, we have proven that for any (i.e., for every) element $z\in Z$, there exists $y\in Y$ such that $g(y)=z$. Note that $f$ need not be surjective to prove the above, as this condition was not needed for the above proof to be valid.
    \end{proof}
    \item \label{exr:3.3.6}Let $f:X\to Y$ be a bijective function, and let $f^{-1}:Y\to X$ be its inverse. Verify the cancellation laws $f^{-1}(f(x))=x$ for all $x\in X$ and $f(f^{-1}(y))=y$ for all $y\in Y$. Conclude that $f^{-1}$ is also invertible, and has $f$ as its inverse (thus $(f^{-1})^{-1}=f$).
    \begin{proof}
        % Let $x$ be any element of $X$. Then by Definition \ref{dfn:functions}, there exists exactly one $y$ such that $f(x)=y$. Considering this $y$, by Definition \ref{dfn:bijective}, there exists exactly one $x'$ such that $f(x')=y$. But since both $f(x)$ and $f(x')$ equal $y$, $f(x)=f(x')$. Thus, since $f^{-1}$ will map $y$ to the element $x'$ such that $f(x')=y$, we know that $f^{-1}$ maps $y$ to $x$, i.e., $f^{-1}(y)=f^{-1}(f(x))=x$.\par
        Let $y$ be any element of $Y$. We know that $f^{-1}$ maps $y$ to the element $x\in X$ satisfying $f(x)=y$ (the uniqueness of $x$ being guaranteed by Definition \ref{dfn:bijective}). Thus, since $y=f(x)$, $f^{-1}$ maps $f(x)$ to $x$, i.e., $f^{-1}(f(x))=x$. A similar argument can treat the other cancellation law.\par
        To prove that $f^{-1}$ is invertible, we must verify that for every $x\in X$, there is exactly one $y$ such that $f^{-1}(y)=x$. Since we know that for every $x\in X$, $f^{-1}(f(x))=x$, and $f(x)=y$ for some $y\in Y$ (Definition \ref{dfn:functions}), we know that for every $x\in X$, $f^{-1}(y)=x$ for some $y\in Y$. Now suppose for the sake of contradiction that there exist some $y\neq y'$ such that $f^{-1}(y)=x=f^{-1}(y')$. Then $f(x)=y$ and $f(x)=y'$, which contradicts Definition \ref{dfn:functions}. Therefore, for every $x\in X$, $f^{-1}(y)=x$ for some and for not more than one $y\in Y$, i.e., for exactly one $y\in Y$.\par
        The inverse of $f^{-1}$ is, by definition, the function $(f^{-1})^{-1}:X\to Y$ associated with the property $P(x,y)$ defined by $f^{-1}(y)=x$. Clearly $(f^{-1})^{-1}$ has the same domain and range as $f$. All that's left is to show that $f(x)=(f^{-1})^{-1}(x)$. $f(x)$ is the element $y\in Y$ such that $f(x)=y$, and $(f^{-1})^{-1}(x)$ is the element $y\in Y$ such that $f^{-1}(y)=x$, i.e., $f(x)=y$. Since the functions are associated with the same property, they must be equal.
    \end{proof}
    \item \label{exr:3.3.7}Let $f:X\to Y$ and $g:Y\to Z$ be functions. Show that if $f$ and $g$ are bijective, then so is $g\circ f$, and we have $(g\circ f)^{-1}=f^{-1}\circ g^{-1}$.
    \begin{proof}
        By Definition \ref{dfn:bijective}, $f$ and $g$ are both both injective and surjective. Thus, by consecutive applications of Exercise \ref{exr:3.3.2}, $g\circ f$ is both injective and surjective. Thus, by Definition \ref{dfn:bijective} again, $g\circ f$ is bijective.\par
        To prove that $(g\circ f)^{-1}=f^{-1}\circ g^{-1}$, Definition \ref{dfn:functionEquality} tells us that it will suffice to show that $(g\circ f)^{-1}$ and $f^{-1}\circ g^{-1}$ have the same domain and range, and $(g\circ f)^{-1}(z)=(f^{-1}\circ g^{-1})(z)$ for all $z\in Z$. Since $f:X\to Y$ and $g:Y\to Z$, by Definition \ref{dfn:composition}, we have $g\circ f:X\to Z$. Thus, the inverse $(g\circ f)^{-1}:Z\to X$. Similarly, since $g^{-1}:Z\to Y$ and $f^{-1}:Y\to X$, by Definition \ref{dfn:composition}, $f^{-1}\circ g^{-1}:Z\to X$. Thus, $(g\circ f)^{-1}$ and $f^{-1}\circ g^{-1}$ have the same domain and range. As to the other part of the question, let $(f^{-1}\circ g^{-1})(z)=x$. Then
        \begin{align*}
            (f^{-1}\circ g^{-1})(z)=x &\Longrightarrow f^{-1}(g^{-1}(z))=x \tag*{Definition \ref{dfn:composition}}\\
            &\Longrightarrow g^{-1}(z)=f(x) \tag*{Definition of inverse}\\
            &\Longrightarrow z=g(f(x)) \tag*{Definition of inverse}\\
            &\Longrightarrow z=(g\circ f)(x) \tag*{Definition \ref{dfn:composition}}\\
            &\Longrightarrow (g\circ f)^{-1}(z)=x \tag*{Definition of inverse}
        \end{align*}
        so by transitivity we have $(g\circ f)^{-1}(z)=(f^{-1}\circ g^{-1})(z)$, as desired.
    \end{proof}
    \item \label{exr:3.3.8}If $X$ is a subset of $Y$, let $\iota_{X\to Y}:X\to Y$ be the \textbf{inclusion map} (from $X$ to $Y$), defined by mapping $x\mapsto x$ for all $x\in X$, i.e., $\iota_{X\to Y}(x):=x$ for all $x\in X$. The map $\iota_{X\to X}$ is in particular called the \textbf{identity map} (on $X$).
    \begin{enumerate}
        \item Show that if $X\subseteq Y\subseteq Z$, then $\iota_{Y\to Z}\circ\iota_{X\to Y}=\iota_{X\to Z}$.
        \begin{proof}
            By Definition \ref{dfn:composition}, $\iota_{Y\to Z}\circ\iota_{X\to Y}:X\to Z$, so $\iota_{Y\to Z}\circ\iota_{X\to Y}$ and $\iota_{X\to Z}$ have the same domain and range. Now we must check that $\iota_{Y\to Z}\circ\iota_{X\to Y}$ and $\iota_{X\to Z}$ match the same inputs to the same outputs. Since $X\subseteq Y\subseteq Z$, $X\subseteq Z$ (Proposition \ref{prp:subsetTransitive}), so $\iota_{X\to Z}(x)=x$ for all $x\in X$. On the other hand, $X\subseteq Y$ guarantees that $\iota_{X\to Y}(x)=x$ for all $x\in X$ and $Y\subseteq Z$ guarantees that $\iota_{Y\to Z}(y)=y$ for all $y\in Y$. Since $x\in Y$ for all $x\in X$ (Definition \ref{dfn:subsets}),
            \begin{align*}
                \iota_{Y\to Z}\circ\iota_{X\to Y}(x) &= \iota_{Y\to Z}(\iota_{X\to Y}(x)) \tag*{Definition \ref{dfn:composition}}\\
                &= \iota_{Y\to Z}(x)\\
                &= x
            \end{align*}
            for all $x\in X$. Therefore, $\iota_{Y\to Z}\circ\iota_{X\to Y}=\iota_{X\to Z}$.
        \end{proof}
        \item Show that if $f:A\to B$ is any function, then $f=f\circ\iota_{A\to A}=\iota_{B\to B}\circ f$.
        \begin{proof}
            We first prove that $f=f\circ\iota_{A\to A}$, and next prove that $f=\iota_{B\to B}\circ f$. Transitivity will guarantee that $f=f\circ\iota_{A\to A}=\iota_{B\to B}\circ f$.\par
            To prove that $f=f\circ\iota_{A\to A}$, Definition \ref{dfn:functionEquality} tells us that it will suffice to show that $f$ and $f\circ\iota_{A\to A}$ have the same domain and range, and that $f(a)=(f\circ\iota_{A\to A})(a)$ for all $a\in A$. By Definition \ref{dfn:composition}, $f\circ\iota_{A\to A}:A\to B$, so $f$ and $f\circ\iota_{A\to A}$ have the same domain and range. Now since $A\subseteq A$, $\iota_{A\to A}(a)=a$ for all $a\in A$. Thus, by Definition \ref{dfn:composition}, $(f\circ\iota_{A\to A})(a)=f(\iota_{A\to A}(a))=f(a)$.\par
            To prove that $f=\iota_{B\to B}\circ f$, Definition \ref{dfn:functionEquality} tells us that it will suffice to show that $f$ and $\iota_{B\to B}\circ f$ have the same domain and range, and that $f(a)=(\iota_{B\to B}\circ f)(a)$ for all $a\in A$. By Definition \ref{dfn:composition}, $\iota_{B\to B}\circ f:A\to B$, so $f$ and $\iota_{B\to B}\circ f$ have the same domain and range. Now since $B\subseteq B$, $\iota_{B\to B}(b)=b$ for all $b\in B$. Since every $f(a)\in B$, by Definition \ref{dfn:composition}, $(\iota_{B\to B}\circ f)(a)=\iota_{B\to B}(f(a))=f(a)$.
        \end{proof}
        \item Show that if $f:A\to B$ is a bijective function, then $f\circ f^{-1}=\iota_{B\to B}$, and $f^{-1}\circ f=\iota_{A\to A}$.
        \begin{proof}
            By Exercise \ref{exr:3.3.6} (and Definition \ref{dfn:composition}), $(f\circ f^{-1})(b)=b$ for all $b\in B$. By definition, $\iota_{B\to B}(b)=b$ for all $b\in B$. Therefore, $f\circ f^{-1}=\iota_{B\to B}$. A similar argument holds in the other case.
        \end{proof}
        \item Show that if $X$ and $Y$ are disjoint sets, and $f:X\to Z$ and $g:Y\to Z$ are functions, then there is a unique function $h:X\cup Y\to Z$ such that $h\circ\iota_{X\to X\cup Y}=f$ and $h\circ\iota_{Y\to X\cup Y}=g$.
        \begin{proof}
            Let $h:X\cup Y\to Z$ be defined by the following (note that it is the fact that $X$ and $Y$ are disjoint that allows the following definition to make sense, i.e., not be contradictory for some inputs).
            \begin{equation*}
                h(a) :=
                \begin{cases}
                    f(a) & a\in X\\
                    g(a) & a\in Y
                \end{cases}
            \end{equation*}
            To prove that $h\circ\iota_{X\to X\cup Y}=f$, Definition \ref{dfn:functionEquality} tells us that it will suffice to show that $h\circ\iota_{X\to X\cup Y}$ and $f$ have the same domain and range, and that $(h\circ\iota_{X\to X\cup Y})(a)=f(a)$ for all $a\in X$. By Definition \ref{dfn:composition}, $h\circ\iota_{X\to X\cup Y}:X\to Z$, so $h\circ\iota_{X\to X\cup Y}$ and $f$ have the same domain and range. Now since $X\subseteq X\cup Y$ (Exercise \ref{exr:3.1.7}), $\iota_{X\to X\cup Y}(a)=a$ for all $a\in X$. Thus, by Definition \ref{dfn:composition}, for any $a\in X$, $(h\circ\iota_{X\to X\cup Y})(a)=h(\iota_{X\to X\cup Y}(a))=h(a)=f(a)$. A similar argument holds for the other case\footnote{How do I verify uniqueness, or do I not need to?}.
        \end{proof}
    \end{enumerate}
\end{enumerate}



\section{Images and Inverse Images}
\begin{itemize}
    \item \marginnote{7/29:}We now discuss what happens when we apply a function to a set, as opposed to individual, non-set objects.
    \begin{dfn}[Images of sets]\label{dfn:images}
        If $f:X\to Y$ is a function from $X$ to $Y$, and $S$ is a set in $X$ (or a subset of $X$), we define $f(S)$ to be the set
        \begin{equation*}
            f(S) := \{f(x):x\in S\}
        \end{equation*}
        This set is a subset of $Y$, and is sometimes called the \textbf{image} (of $S$ under the map $f$). We sometimes call $f(S)$ the \textbf{forward image} (of $S$) to distinguish it from the concept of the \textbf{inverse image} $f^{-1}(S)$ (of $S$), which is defined below.\par
        Note that, symbolically,
        \begin{equation*}
            x\in S \Longrightarrow f(x)\in f(S)
        \end{equation*}
        and
        \begin{equation*}
            y\in f(S) \Longleftrightarrow y=f(x)\text{ for some }x\in S
        \end{equation*}
    \end{dfn}
    \item Note that $f(S)$ is well-defined, as Definition \ref{dfn:images} is based in Axiom \ref{axm:replacement}. Also note that Definition \ref{dfn:images} could be defined in terms of Axiom \ref{axm:specification}.
    \item As alluded to above, we now define inverse images.
    \begin{dfn}[Inverse images]\label{dfn:inverseImages}
        If $U$ is a subset of $Y$, we define the set $f^{-1}(U)$ to be the set
        \begin{equation*}
            f^{-1}(U) := \{x\in X:f(x)\in U\}
        \end{equation*}
        In other words, $f^{-1}(U)$ consists of all the elements of $X$ which map into $U$:
        \begin{equation*}
            f(x)\in U \Longleftrightarrow x\in f^{-1}(U)
        \end{equation*}
        We call $f^{-1}(U)$ the \textbf{inverse image} (of $U$).
    \end{dfn}
    \item Note that if $f$ is bijective, then we have defined $f^{-1}$ in two different ways. However, the definitions are equivalent (see Exercise \ref{exr:3.4.1}).
    \item Since we wish to treat functions as objects so that we can create sets of functions, we axiomatize this notion.
    \begin{axm}[Power set axiom]\label{axm:powerSets}
        Let $X$ and $Y$ be sets. Then there exists a set, denoted $Y^X$, which consists of all the functions from $X$ to $Y$, thus
        \begin{equation*}
            f\in Y^X \Longleftrightarrow f\text{ is a function with domain }X\text{ and range }Y
        \end{equation*}
    \end{axm}
    \item \dq{Let $X=\{4,7\}$ and $Y=\{0,1\}$. Then the set $Y^X$ consists of four functions: the function that maps $4\mapsto 0$ and $7\mapsto 0$; the function that maps $4\mapsto 0$ and $7\mapsto 1$; the function that maps $4\mapsto 1$ and $7\mapsto 0$; and the function that maps $4\mapsto 1$ and $7\mapsto 1$}{58}
    \item Note that we use the notation $Y^X$ because if $Y$ has $n$ elements and $X$ has $m$ elements, then $Y^X$ has $n^m$ elements (see Proposition \ref{prp:cardinalArithmetic}).
    \item See Exercise \ref{exr:3.4.6} for a consequence of Axiom \ref{axm:powerSets}.
    \item \textbf{Power set} (of $X$): The set $2^X=\{Y:Y\text{ is a subset of }X\}$.
    \item Note that we use the notation $2^X$ because if $X$ has $m$ elements, then $2^X$ has $2^m$ elements (see Chapter \ref{chr:8}).
    \item For the sake of completeness, we enhance Axiom \ref{axm:pairwiseUnion} with the following.
    \begin{axm}[Union]\label{axm:union}
        Let $A$ be a set, all of whose elements are themselves sets. Then there exists a set $\bigcup A$ whose elements are precisely those objects which are elements of the elements of $A$. Thus, for all objects $x$,
        \begin{equation*}
            x\in\bigcup A \Longleftrightarrow x\in S\text{ for some }S\in A
        \end{equation*}
    \end{axm}
    \item Note that Axioms \ref{axm:union} and \ref{axm:singletonPair} imply Axiom \ref{axm:pairwiseUnion} (see Exercise \ref{exr:3.4.8}).
    \item An important consequence of Axiom \ref{axm:union} is that \dq{if one has some set $I$, and for every element $\alpha\in I$ we have some set $A_\alpha$, then we can form the union set $\bigcup_{\alpha\in I}A_\alpha$\footnote{"the union of the family of sets ae sub alpha indexed by the labels alpha in the set ie"} by defining
    \begin{equation}\label{eqn:union}
        \bigcup_{\alpha\in I}A_\alpha := \bigcup\{A_\alpha:\alpha\in I\}
    \end{equation}
    which is a set thanks to the axiom of replacement and the axiom of union}{59}
    \item Note that for any object $y$,
    \begin{equation}\label{eqn:unionElements}
        y\in\bigcup_{\alpha\in I}A_\alpha \Longleftrightarrow y\in A_\alpha\text{ for some }\alpha\in I
    \end{equation}
    \item To explain the above, consider the following example.
    \begin{itemize}
        \item If $I=\{1,2,3\}$, $A_1:=\{2,3\}$, $A_2:=\{3,4\}$, and $A_3:=\{4,5\}$, then $\bigcup_{\alpha\in\{1,2,3\}}A_\alpha=\{2,3,4,5\}$.
    \end{itemize}
    \item There is some notation associated with this paradigm, defined as follows.
    \item \textbf{Index set}: The set $I$.
    \item \textbf{Labels}: The elements $\alpha$ of the index set $I$.
    \item \textbf{Family of sets}: The group of sets $A_\alpha$ for all $\alpha\in I$.
    \begin{itemize}
        \item Note that the family of sets is \textbf{indexed} by the labels $\alpha\in I$.
    \end{itemize}
    \item Note that if $I$ were empty, then $\bigcup_{\alpha\in I}A_\alpha$ would also be empty.
    \item \dq{Given any non-empty set $I$, and given an assignment of a set $A_\alpha$ to each $\alpha\in I$, we can define the intersection $\bigcap_{\alpha\in I}A_\alpha$ by first choosing some element $\beta$ of $I$ (which we can do since $I$ is nonempty), and setting
    \begin{equation}\label{eqn:intersection}
        \bigcap_{\alpha\in I}A_\alpha := \{x\in A_\beta:x\in A_\alpha\text{ for all }\alpha\in I\}
    \end{equation}
    which is a set by the axiom of specification}{60}
    \item Note that for any object $y$,
    \begin{equation}\label{eqn:intersectionElements}
        y\in\bigcap_{\alpha\in I}A_\alpha \Longleftrightarrow y\in A_\alpha\text{ for all }\alpha\in I
    \end{equation}
    \item Note that this definition does not depend on the choice of $\beta$ (see Exercise \ref{exr:3.4.9}).
    \item Note that Axioms 3.1-3.11 (excluding Axiom 3.8) are known as the \textbf{Zermelo-Fraenkel axioms of set theory}, after Ernest Zermello (1871-1953) and Abraham Fraenkel (1891-1965).
    \begin{itemize}
        \item Note that \dq{these axioms are formulated slightly differently in other texts, but all the formulations can be shown to be equivalent to each other}{60}
        \item Also note that there is one additional axiom (the \textbf{axiom of choice} --- see Section \ref{sse:8.4}), giving rise to the \textbf{Zermelo-Fraenkel-Choice (ZFC) axioms of set theory}, but we do not need this axiom at this moment.
    \end{itemize}
\end{itemize}


\subsection*{Exercises}
\begin{enumerate}[ref={\thesection.\arabic*}]
    \item \label{exr:3.4.1}Let $f:X\to Y$ be a bijective function, and let $f^{-1}:Y\to X$ be its inverse. Let $V$ be any subset of $Y$. Prove that the forward image of $V$ under $f^{-1}$ is the same set as the inverse image of $V$ under $f$; thus, the fact that both sets are denoted by $f^{-1}(V)$ will not lead to any inconsistency.
    \begin{proof}
        By Definition \ref{dfn:images}, the forward image of $V$ under $f^{-1}$ is the set $\{f^{-1}(x):x\in V\}$. By Definition \ref{dfn:inverseImages}, the inverse image of $V$ under $f$ is the set $\{x\in X:f(x)\in V\}$. Now to prove that these sets are equal, by Definition \ref{dfn:setEquality}, we must verify that every element $y$ of $\{f^{-1}(x):x\in V\}$ is an element of $\{x\in X:f(x)\in V\}$ and vice versa. Suppose first that $y$ is an arbitrary element of $\{f^{-1}(x):x\in V\}$. By Definition \ref{dfn:images}, this implies that $y=f^{-1}(x)$ for some $x\in V$. Thus, by the definition of the inverse (and the fact that $f$ is bijective), $f(y)=x$ for some $x\in V$, or, more simply, $f(y)\in V$. But by Definition \ref{dfn:inverseImages}, $f(y)\in V \Longrightarrow y\in\{x\in X:f(x)\in V\}$. Using the above implications in reverse will suffice to prove that $y\in\{x\in X:f(x)\in V\} \Longrightarrow y\in\{f^{-1}(x):x\in V\}$.
    \end{proof}
    \item \label{exr:3.4.2}\marginnote{7/30:}Let $f:X\to Y$ be a function from one set $X$ to another set $Y$, let $S$ be a subset of $X$, and let $U$ be a subset of $Y$. What, in general, can one say about $f^{-1}(f(S))$ and $S$? What about $f(f^{-1}(U))$ and $U$?
    \begin{proof}
        It is hard to generalize much from the rigid definitions, but we will give those and then prove one property. So first off,
        \begin{align*}
            f^{-1}(f(S)) &= \{x\in X:f(x)\in\{f(x'):x'\in S\}\}\\
            f(f^{-1}(U)) &= \{f(x):x\in\{x'\in X:f(x')\in U\}\}
        \end{align*}
        Now, we can actually prove that $S\subseteq f^{-1}(f(S))$ and $f(f^{-1}(U))\subseteq U$, which we will do as follows.\par
        By Definition \ref{dfn:subsets}, to prove that $S\subseteq f^{-1}(f(S))$, we must verify that every element $x\in S$ is an element of $f^{-1}(f(S))$. Suppose $x$ is any element of $S$. To prove that $x\in\{x'\in X:f(x')\in\{f(x''):x''\in S\}\}$, Axiom \ref{axm:specification} tells us that it will suffice to show that $x\in X$ and "$f(x)\in\{f(x'):x'\in S\}$" is a true statement. Since $S\subseteq X$ and $x\in S$, we have by Definition \ref{dfn:subsets} that $x\in X$. On the other hand, let $y=f(x)$. Then since $x\in S$, by Axiom \ref{axm:replacement}, $y\in\{y':y'=f(x')\text{ for some }x'\in S\} \Longrightarrow f(x)\in\{f(x'):x'\in S\}$.\par
        By Definition \ref{dfn:subsets}, to prove that $f(f^{-1}(U))\subseteq U$, we must verify that every element $y\in f(f^{-1}(U))$ is an element of $U$. Suppose $y$ is any element of $f(f^{-1}(U))$. Then by the rigid definition of $f(f^{-1}(U))$, above, and Definition \ref{dfn:setEquality}, we have $y\in\{f(x):x\in\{x'\in X:f(x')\in U\}\}$. By Axiom \ref{axm:replacement}, this means that $y=f(x)$ for some $x\in\{x'\in X:f(x')\in U\}$. But by Axiom \ref{axm:specification}, $x\in\{x'\in X:f(x')\in U\} \Longrightarrow (x\in X\text{ and }f(x)\in U)$. Thus, we know that $y=f(x)$ for some $f(x)\in U$, so we have $y\in U$.
    \end{proof}
    \item \label{exr:3.4.3}Let $A,B$ be two subsets of a set $X$, and let $f:X\to Y$ be a function. Show that $f(A\cap B)\subseteq f(A)\cap f(B)$, that $f(A)\setminus f(B)\subseteq f(A\setminus B)$, and that $f(A\cup B)=f(A)\cup f(B)$. For the first two statements, is it true that the $\subseteq$ relation can be improved to $=$?
    \begin{proof}
        Because of the number of statements to prove and the logical simplicity of the implications involved, we will shorthand these proofs.\par
        We begin by proving that $f(A\cap B)\subseteq f(A)\cap f(B)$, for which it will suffice (Definition \ref{dfn:subsets}) to show that every element $y$ of $f(A\cap B)$ is an element of $f(A)\cap f(B)$. To this end, let $y$ be any element of $f(A\cap B)$. Since $A\cap B\subseteq A$ (Exercise \ref{exr:3.1.7}) and $A\subseteq X$, $A\cap B\subseteq X$ (Proposition \ref{prp:subsetTransitive}). Thus,
        \begin{align*}
            y\in f(A\cap B) &\Longrightarrow y=f(x)\text{ for some }x\in A\cap B \tag*{Definition \ref{dfn:images}}\\
            &\Longrightarrow (y=f(x)\text{ for some }x\in A\text{ and }y=f(x)\text{ for some }x\in B) \tag*{Definition \ref{dfn:intersection}}\\
            &\Longrightarrow (y\in f(A)\text{ and }y\in f(B)) \tag*{Definition \ref{dfn:images}}\\
            &\Longrightarrow y\in f(A)\cap f(B) \tag*{Definition \ref{dfn:intersection}}
        \end{align*}
        To prove that $f(A)\setminus f(B)\subseteq f(A\setminus B)$, we must verify (Definition \ref{dfn:subsets}) that every element $y$ of $f(A)\setminus f(B)$ is an element of $f(A\setminus B)$, which can be done as follows. Note that $A\setminus B\subseteq X$ since $A\setminus B\subseteq A$ (Definition \ref{dfn:differenceSets} and Axiom \ref{axm:specification}) and $A\subseteq X$ (Proposition \ref{prp:subsetTransitive}).
        \begin{align*}
            y\in f(A)\setminus f(B) &\Longrightarrow y\in\{z\in f(A):z\notin f(B)\} \tag*{Definition \ref{dfn:differenceSets}}\\
            &\Longrightarrow (y\in f(A)\text{ and }y\notin f(B)) \tag*{Axiom \ref{axm:specification}}\\
            &\Longrightarrow (y=f(x)\text{ for some }x\in A\text{ and }y=f(x)\text{ for no }x\in B) \tag*{Definition \ref{dfn:images}}\\
            &\Longrightarrow y=f(x)\text{ for some }x\in A\setminus B \tag*{Definition \ref{dfn:differenceSets}}\\
            &\Longrightarrow y\in f(A\setminus B) \tag*{Definition \ref{dfn:images}}
        \end{align*}
        To prove that $f(A\cup B)=f(A)\cup f(B)$, Definition \ref{dfn:setEquality} tells us that it will suffice to show that $y\in f(A\cup B) \Longleftrightarrow y\in f(A)\cup f(B)$, which can be done as follows.
        \begin{align*}
            y\in f(A\cup B) &\Longleftrightarrow y=f(x)\text{ for some }x\in A\cup B \tag*{Definition \ref{dfn:images}}\\
            &\Longleftrightarrow (y=f(x)\text{ for some }x\in A\text{ or }y=f(x)\text{ for some }x\in B) \tag*{Axiom \ref{axm:pairwiseUnion}}\\
            &\Longleftrightarrow (y\in f(A)\text{ or }y\in f(B)) \tag*{Definition \ref{dfn:images}}\\
            &\Longleftrightarrow y\in f(A)\cup f(B) \tag*{Axiom \ref{axm:pairwiseUnion}}
        \end{align*}
        Because of the transitivity of logically equivalent statements (we may induct the result of Exercise \ref{exr:A.1.5}), we have shown what was desired.\par
        As to the other part of the question, it is \emph{not} true that the $\subseteq$ relation can be improved to $=$ for the first or the second statement, which we may verify thorugh two counterexamples, as follows. For the first statement, let $X=\{1,2\}$, $Y=\{3,4\}$, $A=\{1\}$, $B=\{2\}$, and $f:\{1,2\}\to\{3,4\}$ be defined by $f(x):=3$. Then we have $f(A)\cap f(B)=\{3\}$, but $f(A\cap B)=\{\}$. Thus, $3\in f(A)\cap f(B)$, but $3\notin f(A\cap B)$. Note that the break down in the would-be reverse proof of $f(A\cap B)\subseteq f(A)\cap f(B)$ comes from the fact that although $y\in f(A)\cap f(B) \Longrightarrow y=f(x)$ for some $x\in A$ and $y=f(x')$ for some $x'\in B$, we cannot guarantee that $x=x'$, i.e., that $x\in A\cap B$. As to the second statement, we consider the same function, except that we let $A=\{1,2\}$. Thus, we have $f(A\setminus B)=\{3\}$, but $f(A)\setminus f(B)=\{\}$, implying that $3\in f(A\setminus B)$ but $3\notin f(A)\setminus f(B)$. Note that the breakdown in the would-be reverse proof of $f(A)\setminus f(B)\subseteq f(A\setminus B)$ comes from the fact that although $y\in f(A\setminus B) \Longrightarrow y=f(x)$ for some $x\in A$, we cannot guarantee that there exists no $x'\in B$ such that $y=f(x')$.
    \end{proof}
    \item \label{exr:3.4.4}Let $f:X\to Y$ be a function from one set $X$ to another set $Y$, and let $U,V$ be subsets of $Y$. Show that $f^{-1}(U\cup V)=f^{-1}(U)\cup f^{-1}(V)$, that $f^{-1}(U\cap V)=f^{-1}(U)\cap f^{-1}(V)$, and that $f^{-1}(U\setminus V)=f^{-1}(U)\setminus f^{-1}(V)$.
    \begin{proof}
        For the same reasons as in Exercise \ref{exr:3.4.3}, we shorthand these proofs.\par
        To prove that $f^{-1}(U\cup V)=f^{-1}(U)\cup f^{-1}(V)$, Definition \ref{dfn:setEquality} tells us that it will suffice to show that $x\in f^{-1}(U\cup V) \Longleftrightarrow x\in f^{-1}(U)\cup f^{-1}(V)$, which can be done as follows.
        \begin{align*}
            x\in f^{-1}(U\cup V) &\Longleftrightarrow f(x)\in U\cup V \tag*{Definition \ref{dfn:inverseImages}}\\
            &\Longleftrightarrow (f(x)\in U\text{ or }f(x)\in V) \tag*{Axiom \ref{axm:pairwiseUnion}}\\
            &\Longleftrightarrow (x\in f^{-1}(U)\text{ or }x\in f^{-1}(V)) \tag*{Definition \ref{dfn:inverseImages}}\\
            &\Longleftrightarrow x\in f^{-1}(U)\cup f^{-1}(V) \tag*{Axiom \ref{axm:pairwiseUnion}}
        \end{align*}
        To prove that $f^{-1}(U\cap V)=f^{-1}(U)\cap f^{-1}(V)$, Definition \ref{dfn:setEquality} tells us that it will suffice to show that $x\in f^{-1}(U\cap V) \Longleftrightarrow x\in f^{-1}(U)\cap f^{-1}(V)$, which can be done as follows.
        \begin{align*}
            x\in f^{-1}(U\cap V) &\Longleftrightarrow f(x)\in U\cap V \tag*{Definition \ref{dfn:inverseImages}}\\
            &\Longleftrightarrow (f(x)\in U\text{ and }f(x)\in V) \tag*{Definition \ref{dfn:intersection}}\\
            &\Longleftrightarrow (x\in f^{-1}(U)\text{ and }x\in f^{-1}(V)) \tag*{Definition \ref{dfn:inverseImages}}\\
            &\Longleftrightarrow x\in f^{-1}(U)\cap f^{-1}(V) \tag*{Definition \ref{dfn:intersection}}
        \end{align*}
        To prove that $f^{-1}(U\setminus V)=f^{-1}(U)\setminus f^{-1}(V)$, Definition \ref{dfn:setEquality} tells us that it will suffice to show that $x\in f^{-1}(U\setminus V) \Longleftrightarrow x\in f^{-1}(U)\setminus f^{-1}(V)$, which can be done as follows.
        \begin{align*}
            x\in f^{-1}(U\setminus V) &\Longleftrightarrow f(x)\in U\setminus V \tag*{Definition \ref{dfn:inverseImages}}\\
            &\Longleftrightarrow f(x)\in\{y\in U:y\notin V\} \tag*{Definition \ref{dfn:differenceSets}}\\
            &\Longleftrightarrow (f(x)\in U\text{ and }f(x)\notin V) \tag*{Axiom \ref{axm:specification}}\\
            &\Longleftrightarrow (x\in f^{-1}(U)\text{ and }x\notin f^{-1}(V)) \tag*{Definition \ref{dfn:inverseImages}}\\
            &\Longleftrightarrow x\in\{f^{-1}(U):x\notin f^{-1}(V)\} \tag*{Axiom \ref{axm:specification}}\\
            &\Longleftrightarrow x\in f^{-1}(U)\setminus f^{-1}(V) \tag*{Definition \ref{dfn:inverseImages}}
        \end{align*}
    \end{proof}
    \item \label{exr:3.4.5}Let $f:X\to Y$ be a function from one set $X$ to another set $Y$. Show that $f(f^{-1}(S))=S$ for every $S\subseteq Y$ if and only if $f$ is surjective. Show that $f^{-1}(f(S))=S$ for every $S\subseteq X$ if and only if $f$ is injective.
    \begin{proof}
        To show that $f(f^{-1}(S))=S$ for every $S\subseteq Y$ if and only if $f$ is surjective, we must verify that the implication "if $f(f^{-1}(S))=S$ for every $S\subseteq Y$, then $f$ is surjective" and its converse are both true.\par
        Suppose first that $f(f^{-1}(S))=S$ for every $S\subseteq Y$. To show that $f$ is surjective, Definition \ref{dfn:surjective} tells us that it will suffice to show that for every $y\in Y$, there exists $x\in X$ such that $f(x)=y$. Let $y$ be any element of $Y$. Then by Axiom \ref{axm:singletonPair}, there exists a set $\{y\}$. Since $y\in Y$ and $y$ is the only element of $\{y\}$, by Definition \ref{dfn:subsets}, $\{y\}\subseteq Y$. Thus, $f(f^{-1}(\{y\}))=\{y\}$. Since $\{y\}$ is a nonempty set, we know that $f^{-1}(\{y\})$ is a nonempty set (if $f^{-1}(\{y\})$ were empty, then $f(f^{-1}(\{y\}))=\{y\}$ would be empty, a contradiction). Thus, there exists some $x\in f^{-1}(\{y\})$. Since Definition \ref{dfn:inverseImages} implies that $x\in\{x'\in X:f(x')\in\{y\}\}$, meaning by Axiom \ref{axm:specification} that $x\in X$ and $f(x)\in\{y\}$, and, thus, by Axiom \ref{axm:singletonPair} that $f(x)=y$, we have that there exists $x\in X$ such that $f(x)=y$, as desired.\par
        % Let $S$ be an arbitrary subset of $Y$, and suppose that $y$ is an arbitrary element of $f(f^{-1}(S))$. Then by Definition \ref{dfn:images}, $y=f(x)$ for some $x\in f^{-1}(S)$. Now consider this $x$: by Definition \ref{dfn:inverseImages}, $x\in f^{-1}(S) \Longrightarrow f(x)\in S$. But $y=f(x)$, so $y$ must be an element of $S$.
        Now suppose that $f$ is surjective. To show that $f(f^{-1}(S))=S$ for every $S\subseteq Y$, we must verify that for any subset $S$ of $Y$, every element $y$ of $f(f^{-1}(S))$ is an element of $S$ and vice versa (Definition \ref{dfn:setEquality}). Let $S$ be an arbitrary subset of $Y$. By Exercise \ref{exr:3.4.2}, $f(f^{-1}(S))\subseteq S$. On the other hand, suppose that $y$ is an arbitrary element of $S$. Since $f$ is surjective, we have by Definition \ref{dfn:surjective} that $f(x)=y$ for some $x\in X$. Now consider this $x$: since $y\in S$, we know that $f(x)\in S$. Thus, by Definition \ref{dfn:inverseImages}, $x\in f^{-1}(S)$. Therefore, we have $y=f(x)$ for some $x\in f^{-1}(S)$, which by Definition \ref{dfn:images} and the fact that $f^{-1}(S)\subseteq X$ (as implied by Definition \ref{dfn:inverseImages} and Axiom \ref{axm:specification}) implies that $y\in f(f^{-1}(S))$.\par\medskip
        To show that $f^{-1}(f(S))=S$ for every $S\subseteq X$ if and only if $f$ is injective, we must verify that the implication "if $f^{-1}(f(S))=S$ for every $S\subseteq X$, then $f$ is injective" and its converse are both true.\par
        Suppose first that $f^{-1}(f(S))=S$ for every $S\subseteq X$. To show that $f$ is injective, Definition \ref{dfn:injective} tells us that it will suffice to show that if $x\neq x'$ for two elements $x,x'$ of $X$, then $f(x)\neq f(x')$. Let $x,x'$ be any two elements of $X$ such that $x\neq x'$. By Axiom \ref{axm:singletonPair}, the sets $\{x\}$ and $\{x'\}$ exist. Note that since $x\in X$ and $x$ is the only element of $\{x\}$, by Definition \ref{dfn:subsets}, $\{x\}\subseteq X$, and, similarly, $\{x'\}\subseteq X$. Since there exists an element of $\{x\}$ (namely $x$) that is not an element of $\{x'\}$, by Definition \ref{dfn:setEquality}, $\{x\}\neq\{x'\}$. Thus, since $f^{-1}(f(\{x\}))=\{x\}$ and $f^{-1}(f(\{x'\}))=\{x'\}$, we have $f^{-1}(f(\{x\}))\neq f^{-1}(f(\{x'\}))$. This implies that $f(\{x\})\neq f(\{x'\})$ (if $f(\{x\})$ were equal to $f(\{x'\})$, then $f^{-1}(f(\{x\}))$ would be equal to $f^{-1}(f(\{x'\}))$, a contradiction). Since $f(\{x\})=\{f(x''):x''\in\{x\}\}=\{f(x)\}$, and, similarly, $f(\{x'\})=\{f(x')\}$, we have $\{f(x)\}\neq\{f(x')\}$. Therefore, $f(x)\neq f(x')$ (if $f(x)=f(x')$, then, as singleton sets (Axiom \ref{axm:singletonPair}), every element of $\{f(x)\}$ would be an element of $\{f(x')\}$, and vice versa, proving their equality by Definition \ref{dfn:setEquality}, a contradiction).\par
        Now suppose that $f$ is injective. To show that $f^{-1}(f(S))=S$ for every $S\subseteq X$, we must verify that for any subset $S$ of $X$, every element $x$ of $f^{-1}(f(S))$ is an element of $S$ and vice versa (Definition \ref{dfn:setEquality}). Let $S$ be an arbitrary subset of $X$, and suppose that $x$ is an arbitrary element of $f^{-1}(f(S))$. First off, $f(S)\subseteq Y$. (We must show that every $y\in f(S)$ is an element of $Y$ (Definition \ref{dfn:subsets}). If $y\in f(S)$, then we know that $y=f(x)$ for some $x\in S$ by Definition \ref{dfn:images}. Since $S\subseteq X$, we have that $y=f(x)$ for some $x\in X$, which, since $f(x)\in Y$, implies that $y\in Y$). Thus, by Definition \ref{dfn:inverseImages}, $f(x)\in f(S)$. By Definition \ref{dfn:images} and the fact that $f$ is injective, this implies that $x\in S$ (if $f$ were not injective, then it would be possible that $x\notin S$ but $f(x)\in f(S)$, since there could exist some $x'\in S$ such that $f(x')=f(x)$). On the other hand, by Exercise \ref{exr:3.4.2}, $S\subseteq f^{-1}(f(S))$.
        % Now suppose that $x$ is an arbitrary element of $S$. Since $S\subseteq X$, $f(x)$ exists and is an element of $f(S)$. Thus, since $f(S)\subseteq Y$ (as outlined above), by Definition \ref{dfn:inverseImages}, $x\in f^{-1}(f(S))$.
    \end{proof}
    \item \label{exr:3.4.6}\marginnote{7/29:}Prove the following lemma. (Hint: start with the set $\{0,1\}^X$ and apply the replacement axiom, replacing each function $f$ with the object $f^{-1}(\{1\})$.) See also Exercise \ref{exr:3.5.11}.
    \begin{lem}
        Let $X$ be a set. Then the set
        \begin{equation*}
            \{Y:Y\text{ is a subset of }X\}
        \end{equation*}
        is a set.
        \begin{proof}
            We first define a set composed entirely of subsets of $X$. We then guarantee that this set includes all subsets of $X$.\par
            Consider the set $\{0,1\}^X$. For any $f\in\{0,1\}^X$, since $\{1\}\subseteq\{0,1\}$ (i.e., is a subset of the range of $f$), Definition \ref{dfn:inverseImages} allows us to define the (unique) set $f^{-1}(\{1\})$. Now let $P(f,A)$ be the property $A=f^{-1}(\{1\})$. Then for every element $f\in\{0,1\}^X$, there is exactly, i.e., at most one $A$ for which $P(f,A)$ is true. Consequently, by Axiom \ref{axm:replacement}, there exists a set $\{A:P(f,A)\text{ is true for some }f\in\{0,1\}^X\}$, or, more briefly, $\{f^{-1}(\{1\}):f\in\{0,1\}^X\}$. Now by Definition \ref{dfn:inverseImages}, each $f^{-1}(\{1\})$ represents the corresponding set $\{x\in X:f(x)\in\{1\}\}$. This implies by Axiom \ref{axm:specification} that every $f^{-1}(\{1\})$ is a subset of $X$. Thus, $\{f^{-1}(\{1\}):f\in\{0,1\}^X\}$ is a set composed entirely of subsets of $X$.\par
            At this point, all that's left is to guarantee that $\{f^{-1}(\{1\}):f\in\{0,1\}^X\}$ includes \emph{all} subsets of $X$, which may be done as follows. Let $Z$ be any subset of $X$. Then we may define the function $f:X\to\{0,1\}$ by the following.
            \begin{equation*}
                f(x) :=
                \begin{cases}
                    1 & x\in Z\\
                    0 & x\notin Z
                \end{cases}
            \end{equation*}
            By Axiom \ref{axm:powerSets}, $f\in\{0,1\}^X$. Thus, by the first part of this proof, we know that $f^{-1}(\{1\})$ exists and is defined to be equal to $\{x\in X:f(x)\in\{1\}\}$. But Axiom \ref{axm:singletonPair}, $f(x)\in\{1\}$ iff $f(x)=1$, and by the definition of $f$, $f(x)=1$ iff $x\in Z$. Thus, $f^{-1}(\{1\})=\{x\in X:x\in Z\}=Z$. Since $f^{-1}(\{1\})\in\{f^{-1}(\{1\}):f\in\{0,1\}^X\}$, we have $Z\in\{f^{-1}(\{1\}):f\in\{0,1\}^X\}$. Therefore, there exists a set that contains exactly the subsets of $X$, which we may denote by $\{Y:Y\text{ is a subset of }X\}$.
            % Suppose for the sake of contradiction that $Z\subseteq X$ and $Z\notin\{f^{-1}(\{1\}):f\in\{0,1\}^X\}$.
        \end{proof}
    \end{lem}
    \item \label{exr:3.4.7}\marginnote{7/31:}Let $X,Y$ be sets. Define a \textbf{partial function} from $X$ to $Y$ to be any function $f:X'\to Y'$ whose domain $X'$ is a subset of $X$, and whose range $Y'$ is a subset of $Y$. Show that the collection of all partial functions from $X$ to $Y$ is itself a set. (Hint: use Exercise \ref{exr:3.4.6}, the power set axiom, the replacement axiom, and the union axiom.)
    % \begin{enumerate}
    %     \item Use Exercise \ref{exr:3.4.6} to define the sets $2^X$ and $2^Y$.
    %     \item Define the set $A_{X'}$ corresponding to the element $X'\in 2^X$ as follows. Consider the set $2^Y$. Let $P(Y',Z)$ be the statement $Z=Y'^{X'}$. By Axiom \ref{axm:powerSets}, for every $Y'\in 2^Y$, there is exactly one $Z$ for which $P(Y',Z)$ is true. Thus, by Axiom \ref{axm:replacement}, the set $\{Z:Z=Y'^{X'}\text{ for some }Y'\in 2^Y\}$, which we denote by $A_{X'}$, exists.
    %     \begin{enumerate}
    %         \item $A_{X'}$ is the set of all sets of partial functions from $X$ to $Y$ with domain $X'$ and a certain range.
    %     \end{enumerate}
    %     \item Letting $2^X$ be an index set, use Axiom \ref{axm:union} to create the set $\bigcup_{\alpha\in 2^X}A_\alpha$.
    %     \begin{enumerate}
    %         \item $\bigcup_{\alpha\in 2^X}A_\alpha$ is the set of all sets of partial functions from $X$ to $Y$ with a certain domain and range.
    %     \end{enumerate}
    %     \item Use Axiom \ref{axm:union} to create the set $\bigcup \left( \bigcup_{\alpha\in 2^X}A_\alpha \right)$.
    %     \begin{enumerate}
    %         \item This will be the set of all partial functions from $X$ to $Y$, which is what's desired.
    %     \end{enumerate}
    % \end{enumerate}
    \begin{proof}
        To create the set of all partial functions from $X$ to $Y$, we follow a four step plan. First, we define the set $2^X$ of all subsets of $X$, and the set $2^Y$ of all subsets of $Y$. Second, we define the set $A_{X'}$ of all sets of all partial functions from $X$ to $Y$ with domain $X'$ and a certain range (namely, a range that is any particular element of $2^Y$). Third, we create the set $\bigcup_{X'\in 2^X}A_{X'}$ of all sets of all partial functions from $X$ to $Y$ with a certain domain (namely, any particular element of $2^X$) and a certain range (namely, any particular element of $2^Y$). Fourth, we create the set $\bigcup \left( \bigcup_{X'\in 2^X}A_{X'} \right)$, which is the set of all partial functions from $X$ to $Y$, as desired\footnote{Do I need to verify that every element of this set is a partial function from $X$ to $Y$ and that every partial function from $X$ to $Y$ is an element of this set?}. Let's begin.
        \begin{enumerate}[label={(\arabic*)}]
            \item By Exercise \ref{exr:3.4.6}, there exists a set $\{X':X'\text{ is a subset of }X\}$ (which we may denote by $2^X$) and a set $\{Y':Y'\text{ is a subset of }Y\}$ (which we may denote by $2^Y$). In sum, we have defined the sets
            \begin{align*}
                2^X &:= \{X':X'\text{ is a subset of }X\}&
                2^Y &:= \{Y':Y'\text{ is a subset of }Y\}
            \end{align*}
            \item Let $X'$ be an element of $2^X$, let $Y'$ be any element of $2^Y$, and let $P(Y',Z)$ be the statement $Z=Y'^{X'}$. (Note that $Y'^{X'}$ is the set of all functions with domain $X'$ and range $Y'$, i.e., the set of all partial functions from $X$ to $Y$ with domain $X'$ and range $Y'$, as defined in Axiom \ref{axm:powerSets}.) Since Axiom \ref{axm:powerSets} guarantees the uniqueness of $B^A$ for any sets $A,B$, for every $Y'\in 2^Y$, there is exactly one $Z$ for which $P(Y',Z)$ is true. Thus, by Axiom \ref{axm:replacement}, there exists a set $\{Z:P(Y',Z)\text{ is true for some }Y'\in 2^Y\}$ (which we may denote by $A_{X'}$). In sum, we have defined for every $X'\in 2^X$ the set
            \begin{equation*}
                A_{X'} := \{Y'^{X'}:Y'\in 2^Y\}
            \end{equation*}
            \item Let $2^X$ be an index set. Then by (2), for every $X'\in 2^X$, we have some set $A_{X'}$. Thus, by Equation \ref{eqn:union}, we can form the union set $\bigcup_{X'\in 2^X}A_{X'}$ by defining $\bigcup_{X'\in 2^X}A_{X'}:=\bigcup\{A_{X'}:X'\in 2^X\}$. If we inspect this definition, we see that this set is the set of all elements in every possible $A_{X'}$ (i.e., every $A_{X'}$ such that $X'\in 2^X$). Since every element of any $A_{X'}$ is a set of the type $Y'^{X'}$, the above union is really just the set of all $Y'^{X'}$ for any $X'\in 2^X$ and any $Y'\in 2^Y$. In sum, we have defined the set
            \begin{equation*}
                \bigcup_{X'\in 2^X}A_{X'} := \{Y'^{X'}:X'\in 2^X;Y'\in 2^Y\}
            \end{equation*}
            \item By Axiom \ref{axm:powerSets}, every $Y'^{X'}$ is a set. Thus, the set defined in (3) is a set whose elements are all themselves sets. Therefore, Axiom \ref{axm:union} applies, and we can create the set $\bigcup \left( \bigcup_{X'\in 2^X}A_{X'} \right)$, whose elements are precisely those objects which are elements of the elements of $\bigcup_{X'\in 2^X}A_{X'}$; in other words, whose elements are every partial function in every $Y'^{X'}$, i.e., all partial functions from $X$ to $Y$.
        \end{enumerate}
    \end{proof}
    \item \label{exr:3.4.8}\marginnote{7/29:}Show that Axiom \ref{axm:pairwiseUnion} can be deduced from Axiom \ref{axm:setsAreObjects}, Axiom \ref{axm:singletonPair}, and Axiom \ref{axm:union}.
    \begin{proof}
        To demonstrate Axiom \ref{axm:pairwiseUnion}, we must show that for any two sets $A,B$, there exists a set $A\cup B$ satisfying $x\in A\cup B$ iff $x\in A$ or $x\in B$.\par
        Let $A,B$ be sets. Then by Axiom \ref{axm:setsAreObjects}, $A$ and $B$ are objects. Thus, by Axiom \ref{axm:singletonPair}, we can create the pair set $\{A,B\}$. Since $\{A,B\}$ has only sets for elements, Axiom \ref{axm:union} applies, and implies the existence of $\bigcup\{A,B\}$, which we may denote by $A\cup B$. Axiom \ref{axm:union} also asserts that
        \begin{equation*}
            x\in A\cup B \Longleftrightarrow x\in S\text{ for some }S\in\{A,B\}
        \end{equation*}
        But by Axiom \ref{axm:singletonPair}, $S\in A\cup B$ iff $S=A$ or $S=B$. Therefore, we have
        \begin{equation*}
            x\in A\cup B \Longleftrightarrow x\in S\text{, }S\text{ being a set that satisfies }S=A\text{ or }S=B
        \end{equation*}
        or
        \begin{equation*}
            x\in A\cup B \Longleftrightarrow x\in A\text{ or }x\in B
        \end{equation*}
    \end{proof}
    \item \label{exr:3.4.9}Show that if $\beta$ and $\beta'$ are two elements of a set $I$, and to each $\alpha\in I$ we assign a set $A_\alpha$, then
    \begin{equation*}
        \{x\in A_\beta:x\in A_\alpha\text{ for all }\alpha\in I\} = \{x\in A_{\beta'}:x\in A_\alpha\text{ for all }\alpha\in I\}
    \end{equation*}
    and so the definition of $\bigcap_{\alpha\in I}A_\alpha$ defined in Equation \ref{eqn:intersection} does not depend on $\beta$. Also explain why Equation \ref{eqn:intersectionElements} is true.
    \begin{proof}
        To prove $\{x\in A_\beta:x\in A_\alpha\text{ for all }\alpha\in I\}=\{x\in A_{\beta'}:x\in A_\alpha\text{ for all }\alpha\in I\}$, Definition \ref{dfn:setEquality} tells us that it will suffice to show that
        \begin{equation*}
            y\in\{x\in A_\beta:x\in A_\alpha\text{ for all }\alpha\in I\} \Longleftrightarrow y\in\{x\in A_{\beta'}:x\in A_\alpha\text{ for all }\alpha\in I\}
        \end{equation*}
        By Equations \ref{eqn:intersection} and \ref{eqn:intersectionElements}, we have
        \begin{equation*}
            y\in\{x\in A_\beta:x\in A_\alpha\text{ for all }\alpha\in I\} \Longleftrightarrow y\in A_\alpha\text{ for all }\alpha\in I
        \end{equation*}
        Similarly, we have
        \begin{equation*}
            y\in\{x\in A_{\beta'}:x\in A_\alpha\text{ for all }\alpha\in I\} \Longleftrightarrow y\in A_\alpha\text{ for all }\alpha\in I
        \end{equation*}
        Therefore, by Exercise \ref{exr:A.1.5}, we have 
        \begin{equation*}
            y\in\{x\in A_\beta:x\in A_\alpha\text{ for all }\alpha\in I\} \Longleftrightarrow y\in\{x\in A_{\beta'}:x\in A_\alpha\text{ for all }\alpha\in I\}
        \end{equation*}
        as desired.\par
        As to the other question, by Definition \ref{dfn:setEquality}, $y\in\bigcap_{\alpha\in I}A_\alpha \Longrightarrow y\in\{x\in A_\beta:x\in A_\alpha\text{ for all }\alpha\in I\}$. Thus, by Axiom \ref{axm:specification}, $y\in A_\beta$ and $y\in A_\alpha$ for all $\alpha\in I$. But $A_\beta=A_\alpha$ for some $\alpha\in I$, so the condition that $y\in A_\beta$ is a tautology. Thus, $y\in A_\alpha$ for all $\alpha\in I$. A similar argument works in the other direction.
    \end{proof}
    \item \label{exr:3.4.10}\marginnote{7/31:}Suppose that $I$ and $J$ are two sets, and for all $\alpha\in I\cup J$ let $A_\alpha$ be a set. Show that $(\bigcup_{\alpha\in I}A_\alpha)\cup(\bigcup_{\alpha\in J}A_\alpha)=\bigcup_{\alpha\in I\cup J}A_\alpha$. If $I$ and $J$ are non-empty, show that $(\bigcap_{\alpha\in I}A_\alpha)\cap(\bigcap_{\alpha\in J}A_\alpha)=\bigcap_{\alpha\in I\cup J}A_\alpha$.
    \begin{proof}
        To prove that $(\bigcup_{\alpha\in I}A_\alpha)\cup(\bigcup_{\alpha\in J}A_\alpha)=\bigcup_{\alpha\in I\cup J}A_\alpha$, Definition \ref{dfn:setEquality} tells us that it will suffice to show that every element $x$ of $(\bigcup_{\alpha\in I}A_\alpha)\cup(\bigcup_{\alpha\in J}A_\alpha)$ is an element of $\bigcup_{\alpha\in I\cup J}A_\alpha$ and vice versa. Suppose first that $x\in(\bigcup_{\alpha\in I}A_\alpha)\cup(\bigcup_{\alpha\in J}A_\alpha)$. Then by Axiom \ref{axm:pairwiseUnion}, $x\in\bigcup_{\alpha\in I}A_\alpha$ or $x\in\bigcup_{\alpha\in J}A_\alpha$. We now divide into two cases. If $x\in\bigcup_{\alpha\in I}A_\alpha$, then by Equation \ref{eqn:unionElements}, $x\in A_\alpha$ for some $\alpha\in I$. By Axiom \ref{axm:pairwiseUnion}, this implies that $x\in A_\alpha$ for some $\alpha\in I\cup J$. Therefore, by Equation \ref{eqn:unionElements}, $x\in\bigcup_{\alpha\in I\cup J}A_\alpha$. On the other hand, if $x\in\bigcup_{\alpha\in J}A_\alpha$, then by an analogous argument to the other case, we have $x\in\bigcup_{\alpha\in I\cup J}A_\alpha$. Now suppose that $x\in\bigcup_{\alpha\in I\cup J}A_\alpha$. Then by Equation \ref{eqn:unionElements}, $x\in A_\alpha$ for some $\alpha\in I\cup J$. Thus, by Axiom \ref{axm:pairwiseUnion}, $x\in A_\alpha$ for some $\alpha\in I$, or $x\in A_\alpha$ for some $\alpha\in J$. We now divide into two cases. If $x\in A_\alpha$ for some $\alpha\in I$, then by Equation \ref{eqn:unionElements}, we have $x\in\bigcup_{\alpha\in I}A_\alpha$. Therefore, by Axiom \ref{axm:pairwiseUnion}, we have $x\in(\bigcup_{\alpha\in I}A_\alpha)\cup(\bigcup_{\alpha\in J}A_\alpha)$. On the other hand, if $x\in A_\alpha$ for some $\alpha\in J$, then by an analogous argument to the other case, we have $x\in(\bigcup_{\alpha\in I}A_\alpha)\cup(\bigcup_{\alpha\in J}A_\alpha)$.\par
        To prove that $(\bigcap_{\alpha\in I}A_\alpha)\cap(\bigcap_{\alpha\in J}A_\alpha)=\bigcap_{\alpha\in I\cup J}A_\alpha$, Definition \ref{dfn:setEquality} tells us that it will suffice to show that every element $x$ of $(\bigcap_{\alpha\in I}A_\alpha)\cap(\bigcap_{\alpha\in J}A_\alpha)$ is an element of $\bigcap_{\alpha\in I\cup J}A_\alpha$ and vice versa. Suppose first that $x\in(\bigcap_{\alpha\in I}A_\alpha)\cap(\bigcap_{\alpha\in J}A_\alpha)$. Then by Definition \ref{dfn:intersection}, $x\in\bigcap_{\alpha\in I}A_\alpha$ and $x\in\bigcap_{\alpha\in J}A_\alpha$. By Equation \ref{eqn:intersectionElements}, this implies that $x\in A_\alpha$ for all $\alpha\in I$ and $x\in A_\alpha$ for all $\alpha\in J$. Thus, by Axiom \ref{axm:pairwiseUnion}, $x\in A_\alpha$ for all $\alpha\in I\cup J$. Therefore, by Equation \ref{eqn:intersectionElements}, $x\in\bigcap_{\alpha\in I\cup J}A_\alpha$. Now suppose that $x\in\bigcap_{\alpha\in I\cup J}A_\alpha$. Then by Equation \ref{eqn:intersectionElements}, $x\in A_\alpha$ for all $\alpha\in I\cup J$. By Axiom \ref{axm:pairwiseUnion}, this implies that $x\in A_\alpha$ for all $\alpha\in I$ and $x\in A_\alpha$ for all $\alpha\in J$. Thus, by Equation \ref{eqn:intersectionElements}, we have $x\in\bigcap_{\alpha\in I}A_\alpha$ and $x\in\bigcap_{\alpha\in J}A_\alpha$. Therefore, by Definition \ref{dfn:intersection}, $x\in(\bigcap_{\alpha\in I}A_\alpha)\cap(\bigcap_{\alpha\in J}A_\alpha)$.
    \end{proof}
    \item \label{exr:3.4.11}Let $X$ be a set, let $I$ be a non-empty set, and for all $\alpha\in I$ let $A_\alpha$ be a subset of $X$. Show that $X\setminus\bigcup_{\alpha\in I}A_\alpha=\bigcap_{\alpha\in I}(X\setminus A_\alpha)$ and $X\setminus\bigcap_{\alpha\in I}A_\alpha=\bigcup_{\alpha\in I}(X\setminus A_\alpha)$. This should be compared with de Morgan's laws in Proposition \ref{prp:booleanAlgebra} (although one cannot derive the above identities directly from de Morgan's laws, as $I$ could be infinite).
    \begin{proof}
        To prove that $X\setminus\bigcup_{\alpha\in I}A_\alpha=\bigcap_{\alpha\in I}(X\setminus A_\alpha)$, Definition \ref{dfn:setEquality} tells us that it will suffice to show that every element $x$ of $X\setminus\bigcup_{\alpha\in I}A_\alpha$ is an element of $\bigcap_{\alpha\in I}(X\setminus A_\alpha)$ and vice versa. Suppose first that $x\in X\setminus\bigcup_{\alpha\in I}A_\alpha$. Then by Definition \ref{dfn:differenceSets} and Axiom \ref{axm:specification}, $x\in X$ and $x\notin\bigcup_{\alpha\in I}A_\alpha$. By Equation \ref{eqn:unionElements}, this implies that $x\notin A_\alpha$ for any $\alpha\in I$. Thus, by Axiom \ref{axm:specification} and Definition \ref{dfn:differenceSets}, $x\in X\setminus A_\alpha$ for all $\alpha\in I$. Therefore, by Equation \ref{eqn:intersectionElements}, $x\in\bigcap_{\alpha\in I}(X\setminus A_\alpha)$. Now suppose that $x\in\bigcap_{\alpha\in I}(X\setminus A_\alpha)$. Then by Equation \ref{eqn:intersectionElements}, $x\in X\setminus A_\alpha$ for all $\alpha\in I$. By Definition \ref{dfn:differenceSets} and Axiom \ref{axm:specification}, this implies that $x\in X$ and $x\notin A_\alpha$ for any $\alpha\in I$. Thus, by Equation \ref{eqn:unionElements}, $x\notin\bigcup_{\alpha\in I}A_\alpha$. Therefore, by Axiom \ref{axm:specification} and Definition \ref{dfn:differenceSets}, $x\in X\setminus\bigcup_{\alpha\in I}A_\alpha$.\par
        A similar argument can treat the other case.
    \end{proof}
\end{enumerate}



\section{Cartesian Products}\label{sse:3.5}
\begin{itemize}
    \item We begin by defining the ordered pair.
    \begin{dfn}[Ordered pair]\label{dfn:orderedPair}
        If $x$ and $y$ are any objects (possibly equal), we define the \textbf{ordered pair} $(x,y)$ to be a new object, consisting of $x$ as its first component and $y$ as its second component. Two ordered pairs $(x,y)$ and $(x',y')$ are considered equal if and only if both their components match, i.e.,
        \begin{equation*}
            (x,y)=(x',y') \Longleftrightarrow (x=x'\text{ and }y=y')
        \end{equation*}
    \end{dfn}
    \item It can be proven that this notion of equality obeys the usual axioms of equality (see Exercise \ref{exr:3.5.3}).
    \item \marginnote{8/1:}Note that while order doesn't matter in sets, it does matter in ordered pairs.
    \begin{itemize}
        \item For example, $\{3,5\}=\{5,3\}$, but $(3,5)\neq(5,3)$.
    \end{itemize}
    \item Note that Definition \ref{dfn:orderedPair} is stated as an axiom (since we have simply postulated that for any two objects $x,y$, an object of the form $(x,y)$ exists), but it \emph{is} possible to define ordered pairs solely in terms of the axioms of set theory (see Exercise \ref{exr:3.5.1}).
    \item \marginnote{8/4:}Note that we now use parentheses to enclose ordered pairs, in addition to clarifying the order of operations and enclosing the arguments of functions and properties. However, the usages should still be unambiguous from context.
    \item We now define the Cartesian product.
    \begin{dfn}[Cartesian product]\label{dfn:cartesianProduct}
        If $X$ and $Y$ are sets, then we define the \textbf{Cartesian product} $X\times Y$ to be the collection of ordered pairs whose first component lies in $X$ and second component lies in $Y$; thus,
        \begin{equation*}
            X\times Y = \{(x,y):x\in X,y\in Y\}
        \end{equation*}
        or equivalently
        \begin{equation*}
            a\in(X\times Y) \Longleftrightarrow a=(x,y)\text{ for some }x\in X\text{ and }y\in Y
        \end{equation*}
    \end{dfn}
    \item Although we simply state it here, it can be proven that the Cartesian product is a set (see Exercise \ref{exr:3.5.1}).
    \item Although $X\times Y$ and $Y\times X$ are often different sets, they are very similar.
    \begin{itemize}
        \item For example, they have the same number of elements (see Exercise \ref{exr:3.6.5}).
    \end{itemize}
    \item Cartesian products allow us to consider functions of two variables.
    \begin{itemize}
        \item \dq{Let $f:X\times Y\to Z$ be a function whose domain $X\times Y$ is a Cartesian product of two other sets $X$ and $Y$. Then $f$ can either be thought of as a function of one variable, mapping the single input of an ordered pair $(x,y)$ in $X\times Y$ to an output $f(x,y)$ in $Z$, or as a function of two variables, mapping an input $x\in X$ and another input $y\in Y$ to a single output $f(x,y)$ in $Z$}{63}
        \item While the notions are technically different, \cite{bib:AnalysisI} will not distinguish between the two (will hold them both equally true).
        \item Note that this notion allows us to reinterpret previously defined operations. For example, addition can be defined as the function $+:\N\times\N\to\N$, defined by $(x,y)\mapsto x+y$.
    \end{itemize}
    \item We now generalize the ordered pair and Cartesian product.
    \begin{dfn}[Ordered $n$-tuple and $n$-fold Cartesian product]\label{dfn:nTuple}
        Let $n$ be a natural number. An \textbf{ordered $\bm{n}$-tuple} $(x_i)_{1\leq i\leq n}$ (also denoted $(x_1,\dots,x_n)$) is a collection of objects $x_i$, one for every natural number $i$ between 1 and $n$; we refer to $x_i$ as the \textbf{$\bm{i^\text{th}}$ component} of the ordered $n$-tuple. Two ordered $n$-tuples $(x_i)_{1\leq i\leq n}$ and $(y_i)_{1\leq i\leq n}$ are said to be equal iff $x_i=y_i$ for all $1\leq i\leq n$. If $(X_i)_{1\leq i\leq n}$ is an ordered $n$-tuple of sets, we define their \textbf{Cartesian product} $\prod_{1\leq i\leq n}X_i$ (also denoted $\prod_{i=1}^nX_i$ or $X_1\times\dots\times X_n$) by
        \begin{equation*}
            \prod_{1\leq i\leq n}X_i := \{(x_i)_{1\leq i\leq n}:x_i\in X_i\text{ for all }1\leq i\leq n\}
        \end{equation*}
    \end{dfn}
    \item \emph{Also known as} [$n$-tuple] \textbf{ordered sequence} (of $n$ elements), \textbf{finite sequence}.
    \begin{itemize}
        \item There also exist \textbf{infinite sequences} (see Chapter \ref{chr:5}).
    \end{itemize}
    \item Note, again, that while Definition \ref{dfn:nTuple} simply postulates that an ordered $n$-tuple and a Cartesian product always exist when needed, it is possible to construct these objects solely in terms of the axioms of set theory (see Exercise \ref{exr:3.5.2}).
    \item The construction of $n$-fold Cartesian products in Definition \ref{dfn:nTuple} can be generalized to infinite Cartesian products (see Definition \ref{dfn:infiniteCartesianProducts}).
    \item Although $X_1\times X_2\times X_3$ and $(X_1\times X_2)\times X_3$ (or $X_1\times(X_2\times X_3)$) are often different sets, there exist obvious \textbf{bijections} (bijective functions) between them, and it is common to treat them as equal.
    \item $n$-fold Cartesian products allow us to consider functions of three or more variables.
    \begin{itemize}
        \item Again, we can make trouble by using parentheses as above, and by considering different notions of what qualifies as an input, but \cite{bib:AnalysisI} will not distinguish.
    \end{itemize}
    \item \dq{If $x$ is an object, then $(x)$ is a 1-tuple, which we shall identify with $x$ itself (even though the two are, strictly speaking, not the same object)}{64}
    \begin{itemize}
        \item It follows that if $X_1$ is any set, then the Cartesian product $\prod_{1\leq i\leq 1}X_i$ is just $X_1$ (since the set of all 1-tuples of the elements of $X_1$ is equal to the set of all elements of $X_1$, i.e., $X_1$).
    \end{itemize}
    \item \textbf{0-tuple}: The ordered $n$-tuple $()$ containing no objects. \emph{Also known as} \textbf{empty tuple}.
    \item \textbf{Empty Cartesian product}: The Cartesian product $\prod_{1\leq i\leq 0}X_i$, which is equal to the singleton set $\{()\}$ whose only element is the 0-tuple.
    \item \dq{If $n$ is a natural number, we often write $X^n$ as shorthand for the $n$-fold Cartesian product $X^n:=\prod_{1\leq i\leq n}X$}{65}
    \begin{itemize}
        \item For example, $X^2=X\times X$, $X^1=X$ (assuming $(x)=x$), and $X^0=\{()\}$.
    \end{itemize}
    \item We now generalize Lemma \ref{lem:singleChoice}.
    \begin{lem}[Finite choice]\label{lem:finiteChoice}
        Let $n\geq 1$ be a natural number, and for each natural number $1\leq i\leq n$, let $X_i$ be a nonempty set. Then there exists an $n$-tuple $(x_i)_{1\leq i\leq n}$ such that $x_i\in X_i$ for all $1\leq i\leq n$. In other words, if each $X_i$ is nonempty, then the set $\prod_{1\leq i\leq n}X_i$ is also non-empty.
        \begin{proof}
            We induct on $n$ (starting with the base case $n=1$; the claim is also vacuously true with $n=0$ but is not particularly interesting in that case). When $n=1$, the claim follows from Lemma \ref{lem:singleChoice}. Now suppose inductively that the claim has already been proven for some $n$; we will now prove it for $n\pplus$. Let $X_1,\dots,X_{n\pplus}$ be a collection of non-empty sets. By induction hypothesis, we can find an $n$-tuple $(x_i)_{1\leq i\leq n}$ such that $x_i\in X_i$ for all $1\leq i\leq n$. Also, since $X_{n\pplus}$ is non-empty, by Lemma \ref{lem:singleChoice} we may find an object $a$ such that $a\in X_{n\pplus}$. If we thus define the $n\pplus$-tuple $(y_i)_{1\leq i\leq n\pplus}$ by setting $y_i:=x_i$ when $1\leq i\leq n$ and $y_i:=a$ when $i=n\pplus$ it is clear that $y_i\in X_i$ for all $1\leq i\leq n\pplus$, thus closing the induction.
        \end{proof}
    \end{lem}
    \item Note that Lemma \ref{lem:finiteChoice} can be extended to allow for an infinite number of choices, but doing so requires the \textbf{axiom of choice} (see Section \ref{sse:8.4}).
\end{itemize}


\subsection*{Exercises}
\begin{enumerate}[ref={\thesection.\arabic*}]
    \item \label{exr:3.5.1}\marginnote{8/1:}Suppose we define the ordered pair $(x,y)$ for any objects $x$ and $y$ by the formula $(x,y):=\{\{x\},\{x,y\}\}$ (thus using several applications of Axiom \ref{axm:singletonPair}). Thus for instance $(1,2)$ is the set $\{\{1\},\{1,2\}\}$, $(2,1)$ is the set $\{\{2\},\{2,1\}\}$, and $(1,1)$ is the set $\{\{1\}\}$. Show that such a definition indeed obeys the definition of equality in Definition \ref{dfn:orderedPair}, and also whenever $X$ and $Y$ are sets, the Cartesian product $X\times Y$ is also a set. Thus this definition can be validly used as a definition of an ordered pair. For an additional challenge, show that the alternate definition $(x,y):=\{x,\{x,y\}\}$ also verifies the definition of equality in Definition \ref{dfn:orderedPair} and is thus also an acceptable definition of an ordered pair. (For this latter task, one needs the axiom of regularity, and in particular Exercise \ref{exr:3.2.2}.)
    \begin{proof}
        To answer the first part of this question (to prove that the given definition of ordered pairs in terms of set theory obeys the appropriate definition of equality), a lemma will be helpful.
        % \begin{lem}\label{lem:singletonPairSetEquality}
        %     Let $a,b,c$ be objects, and let $a\neq b$. Then the singleton sets $\{a\}$ and $\{b\}$ are unequal. Similarly, the pair sets $\{a,c\}$ and $\{b,c\}$ are unequal.
        %     \begin{proof}
        %         Suppose that $a\neq b$. To prove that $\{a\}\neq\{b\}$, Definition \ref{dfn:setEquality} tells us that it will suffice to find a single element $x$ of $\{a\}$ that is not an element of $\{b\}$. Let $x\in\{a\}$. Then by Axiom \ref{axm:singletonPair}, $x=a$. Since $a\neq b$, this implies that $x\neq b$. But by Axiom \ref{axm:singletonPair}, this implies that $x\notin\{b\}$, as desired.\par
        %         % which is a valid implication since Axiom \ref{axm:singletonPair} asserts the logical equivalence of the statements $y\in\{b\}$ and $y=b$
        %         Suppose again that $a\neq b$. To prove that $\{a,c\}\neq\{b,c\}$, Definition \ref{dfn:setEquality} tells us that it will suffice to find a single element $x$ of $\{a,c\}$ that is not an element of $\{b,c\}$, or a single element $y$ of $\{b,c\}$ that is not an element of $\{a,c\}$. Now for this part of the problem, we must consider three cases of possible relations between the three objects $a,b,c$: $a=c$ (and, hence, $b\neq c$), $b=c$ (and, hence, $a\neq c$), and $a\neq c$ and $b\neq c$. Suppose first that $a=c$. By Axiom \ref{axm:singletonPair}, $b\in\{b,c\}$. Since $b\neq a$ and $b\neq c$, Axiom \ref{axm:singletonPair} implies that $b\notin\{a,c\}$. Suppose second that $b=c$. By Axiom \ref{axm:singletonPair}, $a\in\{a,c\}$. Since $a\neq b$ and $a\neq c$, Axiom \ref{axm:singletonPair} implies that $a\notin\{b,c\}$. Suppose third that $a\neq c$ and $b\neq c$. By Axiom \ref{axm:singletonPair}, $a\in\{a,c\}$. Since $a\neq b$ and $a\neq c$, Axiom \ref{axm:singletonPair} implies that $a\notin\{b,c\}$.
        %     \end{proof}
        % \end{lem}
        \begin{lem}\label{lem:singletonPairSetEquality}
            Let $a,b,c$ be objects. Then the singleton sets $\{a\}$ and $\{b\}$ are equal if and only if $a=b$. Additionally, if $a=b$, then the pair sets $\{a,c\}$ and $\{b,c\}$ are equal (the inverse of this implication is also true). To summarize, we have the following three implications:
            \begin{align*}
                a=b &\Longleftrightarrow \{a\}=\{b\}\\
                a=b &\Longrightarrow \{a,c\}=\{b,c\}\\
                a\neq b &\Longrightarrow \{a,c\}\neq\{b,c\}
            \end{align*}
            \begin{proof}
                To verify that the singleton sets $\{a\}$ and $\{b\}$ are equal if and only if $a=b$, we must prove that the implication "if $\{a\}=\{b\}$, then $a=b$" and its converse are both true. Suppose first that $\{a\}=\{b\}$. By Axiom \ref{axm:singletonPair}, $a\in\{a\}$. By Definition \ref{dfn:setEquality} and the fact that $\{a\}=\{b\}$, this implies that $a\in\{b\}$. But by Axiom \ref{axm:singletonPair}, $a\in\{b\} \Longrightarrow a=b$. Now suppose that $a=b$. To prove that $\{a\}=\{b\}$, Definition \ref{dfn:setEquality} tells us that it will suffice to show that every element of $\{a\}$ is an element of $\{b\}$ and vice versa. By Axiom \ref{axm:singletonPair}, $a$ is the only element of $\{a\}$, meaning that to show that every element of $\{a\}$ is an element of $\{b\}$, we need only verify that $a\in\{b\}$. But since $a=b$, Axiom \ref{axm:singletonPair} asserts that $a\in\{b\}$. A similar argument works in the other direction.\par
                Suppose that $a=b$. To prove that $\{a,c\}=\{b,c\}$, Definition \ref{dfn:setEquality} tells us that it will suffice to show that every element of $\{a,c\}$ is an element of $\{b,c\}$ and vice versa. By Axiom \ref{axm:singletonPair}, the only elements of $\{a,c\}$ are $a$ and $c$, so to show that every element of $\{a,c\}$ is an element of $\{b,c\}$, we need only verify that both $a$ and $c$ are elements of $\{b,c\}$. But since $a=b$, Axiom \ref{axm:singletonPair} implies that $a\in\{b,c\}$, and since $c=c$, Axiom \ref{axm:singletonPair} similarly implies that $c\in\{b,c\}$. A similar argument works in the other direction.\par
                Suppose that $a\neq b$. To prove that $\{a,c\}\neq\{b,c\}$, Definition \ref{dfn:setEquality} tells us that it will suffice to find a single element $x$ of $\{a,c\}$ that is not an element of $\{b,c\}$, or a single element $y$ of $\{b,c\}$ that is not an element of $\{a,c\}$. Now for this part of the problem, we must consider separately three cases of possible relations between the three objects $a,b,c$: $a=c$ (and, hence, $b\neq c$), $b=c$ (and, hence, $a\neq c$), and $a\neq c$ and $b\neq c$. Suppose first that $a=c$. By Axiom \ref{axm:singletonPair}, $b\in\{b,c\}$. Since $b\neq a$ and $b\neq c$, Axiom \ref{axm:singletonPair} implies that $b\notin\{a,c\}$. Therefore, there exists an element $y$ of $\{b,c\}$ (namely $b$) such that $y\in\{b,c\}$ and $y\notin\{a,c\}$. Suppose second that $b=c$. By Axiom \ref{axm:singletonPair}, $a\in\{a,c\}$. Since $a\neq b$ and $a\neq c$, Axiom \ref{axm:singletonPair} implies that $a\notin\{b,c\}$. Therefore, there exists an element $x$ of $\{a,c\}$ (namely $a$) such that $x\in\{a,c\}$ and $x\notin\{b,c\}$. Suppose third that $a\neq c$ and $b\neq c$. By Axiom \ref{axm:singletonPair}, $a\in\{a,c\}$. Since $a\neq b$ and $a\neq c$, Axiom \ref{axm:singletonPair} implies that $a\notin\{b,c\}$. Therefore, there exists an element $x$ of $\{a,c\}$ (namely $a$) such that $x\in\{a,c\}$ and $x\notin\{b,c\}$.
            \end{proof}
        \end{lem}
        Now for the (first part of the) primary proof: To verify that the definition $(x,y):=\{\{x\},\{x,y\}\}$ of ordered pairs obeys the definition of equality in Definition \ref{dfn:orderedPair}, we must prove that the sets $\{\{x\},\{x,y\}\}$ and $\{\{x'\},\{x',y'\}\}$ are equal if and only if $x=x'$ and $y=y'$. To prove this, we must show that the implication "if $\{\{x\},\{x,y\}\}=\{\{x'\},\{x',y'\}\}$, then $x=x'$ and $y=y'$" and its converse are both true. Suppose first that $\{\{x\},\{x,y\}\}=\{\{x'\},\{x',y'\}\}$, and suppose for the sake of contradiction that $x\neq x'$ or $y\neq y'$. We divide into two cases.
        \begin{enumerate}[label={(\arabic*)}]
            \item ($x\neq x'$): By Axiom \ref{axm:singletonPair}, $\{x\}\in\{\{x\},\{x,y\}\}$. Since $\{\{x\},\{x,y\}\}=\{\{x'\},\{x',y'\}\}$, this implies by Definition \ref{dfn:setEquality} that $\{x\}\in\{\{x'\},\{x',y'\}\}$. Thus, by Axiom \ref{axm:singletonPair}, $\{x\}=\{x'\}$ or $\{x\}=\{x',y'\}$. Since $x\neq x'$ by hypothesis, Lemma \ref{lem:singletonPairSetEquality} implies that $\{x\}\neq\{x'\}$. Thus, we have that $\{x\}=\{x',y'\}$. Since $x'\in\{x',y'\}$ (Axiom \ref{axm:singletonPair}), Definition \ref{dfn:setEquality} implies that $x'\in\{x\}$. But by Axiom \ref{axm:singletonPair}, this implies that $x'=x$, a contradiction.
            \item ($y\neq y'$): By (1), $x=x'$, implying by Lemma \ref{lem:singletonPairSetEquality} that $\{x\}=\{x'\}$. Additionally, by consecutive applications\footnote{By Lemma \ref{lem:singletonPairSetEquality}, $x=x' \Longrightarrow \{x,y\}=\{x',y\}$, and by Lemma \ref{lem:singletonPairSetEquality} again, $y\neq y' \Longrightarrow \{x',y\}\neq\{x',y'\}$. Thus, transitivity implies that $\{x,y\}\neq\{x',y'\}$.} of Lemma \ref{lem:singletonPairSetEquality}, the facts that $x=x'$ and $y\neq y'$ imply that $\{x,y\}\neq\{x',y'\}$. Since $\{x\}=\{x'\}$ and $\{x,y\}\neq\{x',y'\}$, consecutive applications of Lemma \ref{lem:singletonPairSetEquality} again imply that $\{\{x\},\{x,y\}\}\neq\{\{x'\},\{x',y'\}\}$, a contradiction.
            % \item ($y\neq y'$): By Axiom \ref{axm:singletonPair}, $\{x,y\}\in\{\{x\},\{x,y\}\}$. Thus, by Definition \ref{dfn:setEquality}, $\{x,y\}\in\{\{x'\},\{x',y'\}\}$. By Axiom \ref{axm:setsAreObjects}, this implies that $\{x,y\}=\{x'\}$ or $\{x,y\}=\{x',y'\}$. We now divide into two cases again.
            % \begin{enumerate}[label={(\alph*)}]
            %     \item ($\{x,y\}=\{x'\}$): Since $x$ and $y$ are elements of $\{x,y\}$, Definition \ref{dfn:setEquality} implies that $x$ and $y$ are elements of $\{x'\}$. By Axiom \ref{axm:singletonPair}, this implies that $x=y=x'$. Since $x=x'$ and $y\neq y'$, by Lemma \ref{lem:singletonPairSetEquality}, we have that $\{x,y\}\neq\{x',y'\}$. But by Axiom \ref{axm:union}, $\{\{x\},\{x,y\}\}=\{\{x'\},\{x',y'\}\} \Longrightarrow \{x,y\}=\{x',y'\}$, a contradiction.
            %     \item ($\{x,y\}=\{x',y'\}$): By Axiom \ref{axm:singletonPair}, $y\in\{x,y\}$. By Definition \ref{dfn:setEquality}, this implies that $y\in\{x',y'\}$. Thus, by Axiom \ref{axm:singletonPair}, $y=x'$ or $y=y'$. Since $y\neq y'$ by hypothesis, $y=x'$. For similar reasons, we can show that $x=y'$. Since $\{\{x\},\{x,y\}\}=\{\{x'\},\{x',y'\}\}$, this implies that $\{\{y'\},\{y',y\}\}=\{\{y\},\{y,y'\}\}$. As above, we can show that $\{y'\}=\{y\}$ or $\{y'\}=\{y,y'\}$. Since $y\in\{y,y'\}$ but $y\neq y'$ so $y\notin\{y'\}$, i.e., $\{y'\}\neq\{y,y'\}$, we have $\{y'\}=\{y\}$. But by Lemma \ref{lem:singletonPairSetEquality}, this implies that $y'=y$, a contradiction.
            % \end{enumerate}
        \end{enumerate}
        Therefore, $x=x'$ and $y=y'$. Now suppose that $x=x'$ and $y=y'$. To prove that $\{\{x\},\{x,y\}\}=\{\{x'\},\{x',y'\}\}$, Definition \ref{dfn:setEquality} tells us that it will suffice to show that every element of $\{\{x\},\{x,y\}\}$ is an element of $\{\{x'\},\{x',y'\}\}$ and vice versa. By Axiom \ref{axm:singletonPair}, the only elements of $\{\{x\},\{x,y\}\}$ are $\{x\}$ and $\{x,y\}$, so to show that every element of $\{\{x\},\{x,y\}\}$ is an element of $\{\{x'\},\{x',y'\}\}$, we need only verify that both $\{x\}$ and $\{x,y\}$ are elements of $\{\{x'\},\{x',y'\}\}$. Since $x=x'$, Lemma \ref{lem:singletonPairSetEquality} guarantees that $\{x\}=\{x'\}$. But by Axiom \ref{axm:singletonPair}, this implies that $\{x\}\in\{\{x'\},\{x',y'\}\}$. On the other hand, since $x=x'$ and $y=y'$, we have by consecutive applications of Lemma \ref{lem:singletonPairSetEquality} that $\{x,y\}=\{x',y'\}$. But by Axiom \ref{axm:singletonPair}, this implies that $\{x,y\}\in\{\{x'\},\{x',y'\}\}$. A similar argument can show that every element of $\{\{x'\},\{x',y'\}\}$ is an element of $\{\{x\},\{x,y\}\}$. Therefore, $\{\{x\},\{x,y\}\}=\{\{x'\},\{x',y'\}\}$.\par
        \medskip
        \marginnote{8/4:}Having shown that ordered pairs are indeed well-defined, we now turn our attention to proving that if $X$ and $Y$ are sets, then the Cartesian product $X\times Y$ is also a set. For this purpose, we adapt the procedure from Exercise \ref{exr:3.4.7}. First, for each $x\in X$, we define the set $A_x$ of all ordered pairs whose first component is $x$ and whose second component is some element of $Y$. Second, we create the set $\bigcup_{x\in X}A_x$, which is the set of all ordered pairs whose first component is some element of $X$ and whose second component is some element of $Y$, as desired. Let's begin.
        \begin{enumerate}[label={(\arabic*)}]
            \item Let $x$ be an element of $X$, let $y$ be any element of $Y$, and let $P(y,z)$ be the statement $z=(x,y)$. (Note that $(x,y)$ is the ordered pair with first component $x$ and second component $y$, as defined in Definition \ref{dfn:orderedPair}). Since Definition \ref{dfn:orderedPair} guarantees the uniqueness of the ordered pair $(a,b)$ for any objects $a,b$, for every $y\in Y$, there is exactly one $z$ for which $P(y,z)$ is true. Thus, by Axiom \ref{axm:replacement}, there exists a set $\{z:P(y,z)\text{ is true for some }y\in Y\}$ (which we may denote by $A_x$). In sum, we have defined for every $x\in X$ the set
            \begin{equation*}
                A_x := \{(x,y):y\in Y\}
            \end{equation*}
            \item Let $X$ be an index set. Then by (1), for every $x\in X$, we have some set $A_x$. Thus, by Equation \ref{eqn:union}, we can form the union set $\bigcup_{x\in X}A_x$ by defining $\bigcup_{x\in X}A_x:=\bigcup\{A_x:x\in X\}$. If we inspect this definition, we can see that this set is the set of all elements in every possible $A_x$ (i.e., every $A_x$ such that $x\in X$). Since every element of any $A_x$ is an ordered pair with first component $x$ and second component in $Y$, the set $\bigcup_{x\in X}A_x$ is the set whose elements are every ordered pair whose first component is some element of $X$ and whose second component is some element of $Y$. In sum, we have by Definition \ref{dfn:cartesianProduct}
            \begin{equation*}
                \bigcup_{x\in X}A_x = \{(x,y):x\in X,y\in Y\} = X\times Y
            \end{equation*}
        \end{enumerate}
        \medskip
        Lastly, we treat the additional challenge. To verify that the definition $(x,y):=\{x,\{x,y\}\}$ of ordered pairs obeys the definition of equality in Definition \ref{dfn:orderedPair}, we must prove that the sets $\{x,\{x,y\}\}$ and $\{x',\{x',y'\}\}$ are equal if and only if $x=x'$ and $y=y'$. To prove this, we must show that the implication "if $\{x,\{x,y\}\}=\{x',\{x',y'\}\}$, then $x=x'$ and $y=y'$" and its converse are both true. Suppose first that $\{x,\{x,y\}\}=\{x',\{x',y'\}\}$, and suppose for the sake of contradiction that $x\neq x'$ or $y\neq y'$. We divide into two cases.
        \begin{enumerate}[label={(\arabic*)}]
            \item ($x\neq x'$): By Axiom \ref{axm:singletonPair}, $x\in\{x,\{x,y\}\}$. Since $\{x,\{x,y\}\}=\{x',\{x',y'\}\}$, this implies by Definition \ref{dfn:setEquality} that $x\in\{x',\{x',y'\}\}$. Thus, by Axiom \ref{axm:singletonPair}, $x=x'$ or $x=\{x',y'\}$. Since $x\neq x'$ by hypothesis, we have $x=\{x',y'\}$. A similar argument can show that $x'=\{x,y\}$. Substituting, we have that $x=\{\{x,y\},y'\}$. Since $\{x,y\}\in\{\{x,y\},y'\}$ (Axiom \ref{axm:singletonPair}), Definition \ref{dfn:setEquality} implies that $\{x,y\}\in x$. But by Axiom \ref{axm:singletonPair}, we also have $x\in\{x,y\}$, which contradicts Exercise \ref{exr:3.2.2} (which states that we cannot have $A\in B$ and $B\in A$ for two sets $A,B$).
            \item ($y\neq y'$): By (1), $x=x'$. Additionally, by consecutive applications of Lemma \ref{lem:singletonPairSetEquality}, the facts that $x=x'$ and $y\neq y'$ imply that $\{x,y\}\neq\{x',y'\}$. Since $x=x'$ and $\{x,y\}\neq\{x',y'\}$, consecutive applications of Lemma \ref{lem:singletonPairSetEquality} again imply that $\{x,\{x,y\}\}\neq\{x',\{x',y'\}\}$, a contradiction.
        \end{enumerate}
        Therefore, $x=x'$ and $y=y'$. Now suppose that $x=x'$ and $y=y'$. To prove that $\{x,\{x,y\}\}=\{x',\{x',y'\}\}$, Definition \ref{dfn:setEquality} tells us that it will suffice to show that every element of $\{x,\{x,y\}\}$ is an element of $\{x',\{x',y'\}\}$ and vice versa. By Axiom \ref{axm:singletonPair}, the only elements of $\{x,\{x,y\}\}$ are $x$ and $\{x,y\}$, so to show that every element of $\{x,\{x,y\}\}$ is an element of $\{x',\{x',y'\}\}$, we need only verify that both $x$ and $\{x,y\}$ are elements of $\{x',\{x',y'\}\}$. Since $x=x'$, Axiom \ref{axm:singletonPair} implies that $x\in\{x',\{x',y'\}\}$. On the other hand, since $x=x'$ and $y=y'$, we have by consecutive applications of Lemma \ref{lem:singletonPairSetEquality} that $\{x,y\}=\{x',y'\}$. But by Axiom \ref{axm:singletonPair}, this implies that $\{x,y\}\in\{x',\{x',y'\}\}$. A similar argument can show that every element of $\{x',\{x',y'\}\}$ is an element of $\{x,\{x,y\}\}$. Therefore, $\{x,\{x,y\}\}=\{x',\{x',y'\}\}$.
    \end{proof}
    \item \label{exr:3.5.2}Suppose we define an ordered $n$-tuple to be a surjective function $x:\{i\in\N:1\leq i\leq n\}\to X$ whose range is some arbitrary set $X$ (so different ordered $n$-tuples are allowed to have different ranges); we then write $x_i$ for $x(i)$, and also write $x$ as $(x_i)_{1\leq i\leq n}$. Using this definition, verify that we have $(x_i)_{1\leq i\leq n}=(y_i)_{1\leq i\leq n}$ if and only if $x_i=y_i$ for all $1\leq i\leq n$. Also, show that if $(X_i)_{1\leq i\leq n}$ are an ordered $n$-tuple of sets, then the Cartesian product, as defined in Definition \ref{dfn:nTuple}, is indeed a set. (Hint: use Exercise \ref{exr:3.4.7} and the axiom of specification.)
    \begin{proof}
        To verify that the definition $x:\{i\in\N:1\leq i\leq n\}\to X$ of ordered $n$-tuples makes the statement "$(x_i)_{1\leq i\leq n}=(y_i)_{1\leq i\leq n}$ if and only if $x_i=y_i$ for all $1\leq i\leq n$" true, we must show that the implication "if $(x_i)_{1\leq i\leq n}=(y_i)_{1\leq i\leq n}$, then $x_i=y_i$ for all $1\leq i\leq n$" and its converse are both true under this definition. Suppose first that $(x_i)_{1\leq i\leq n}=(y_i)_{1\leq i\leq n}$. Since $(x_i)_{1\leq i\leq n}$ and $(y_i)_{1\leq i\leq n}$ are both functions, Definition \ref{dfn:functionEquality} implies that $((x_i)_{1\leq i\leq n})(i)=((y_i)_{1\leq i\leq n})(i)$ for all $i\in\{i\in\N:1\leq i\leq n\}$. But notationally, this is the same as $x_i=y_i$ for all $1\leq i\leq n$. Now suppose that $x_i=y_i$ for all $1\leq i\leq n$. Let $X:=\{x_i:1\leq i\leq n\}$ and $Y:=\{y_i:1\leq i\leq n\}$. It can be shown that $X$ and $Y$ are equal by Definition \ref{dfn:setEquality}. Similarly, Exercise \ref{exr:3.1.1} implies that $\{i\in\N:1\leq i\leq n\}=\{i\in\N:1\leq i\leq n\}$ (note that we are allowed to create this set because Axiom \ref{axm:infinity} implies that the set $\N$ exists, and Axiom \ref{axm:specification} implies that we are allowed to cherry-pick elements from it). Thus, since the domain, range, and mappings are equal, we have by Definition \ref{dfn:functionEquality} that $(x_i)_{1\leq i\leq n}=(y_i)_{1\leq i\leq n}$.\par
        As to the second part of the question, begin by using Equation \ref{eqn:union} to define the set $\bigcup_{i\in\{i\in\N:1\leq i\leq n\}}X_i$ containing every element in every set in the ordered $n$-tuple $(X_i)_{1\leq i\leq n}$. Now we define the function $(x_i)_{1\leq i\leq n}:\{i\in\N:1\leq i\leq n\}\to\bigcup_{i\in\{i\in\N:1\leq i\leq n\}}X_i$ (the mapping of this specific function will be irrelevant to our discussion, so we forego defining one). By Exercise \ref{exr:3.4.7}, there exists a set $\{(x_i)_{1\leq i\leq n}:I\to X\Big|I\subseteq\{i\in\N:1\leq i\leq n\},X\subseteq\bigcup_{i\in\{i\in\N:1\leq i\leq n\}}X_i\}$ of all partial functions $(x_i)_{1\leq i\leq n}$ from $\{i\in\N:1\leq i\leq n\}$ to $\bigcup_{i\in\{i\in\N:1\leq i\leq n\}}X_i$. Let $P((x_i)_{1\leq i\leq n}:I\to X)$ be the statement "($I=\{i\in\N:1\leq i\leq n\}$) and (for any $x,x'$ in $X$, $x\in X_i$ and $x'\in X_i$ implies that $x=x'$) and (for all $x\in X$, $((x_i)_{1\leq i\leq n})(i)=x$ if and only if $x\in X_i$)." As defined, $P((x_i)_{1\leq i\leq n}:I\to X)$ is a property pertaining to each $(x_i)_{1\leq i\leq n}:I\to X$ in $\{(x_i)_{1\leq i\leq n}:I\to X\Big|I\subseteq\{i\in\N:1\leq i\leq n\},X\subseteq\bigcup_{i\in\{i\in\N:1\leq i\leq n\}}X_i\}$ that is either true or false. Thus, we can use Axiom \ref{axm:specification} to create the set
        \begin{equation*}
            \textstyle\{(x_i)_{1\leq i\leq n}:I\to X\Big|I\subseteq\{i\in\N:1\leq i\leq n\},X\subseteq\bigcup_{i\in\{i\in\N:1\leq i\leq n\}}X_i\Bigg|P((x_i)_{1\leq i\leq n}:I\to X)\text{ is true}\}
        \end{equation*}
        which, in layman's terms, is the set of all partial functions that map every number $i$ in $\{i\in\N:1\leq i\leq n\}$ to an element $x_i$ of $X_i$, and that are surjective (we know [Definition \ref{dfn:surjective}] that every $x\in X$ maps to an $i\in I$ since every $x$ is an element of $X_i$ for some $i$ [Equation \ref{eqn:unionElements}] and there exists only one element from each $X_i$ in $X$ [if there were two, $P((x_i)_{1\leq i\leq n}:I\to X)$ guarantees that they are equal]). Thus, by the function definition of ordered $n$-tuples, the above set could be expressed as follows
        \begin{equation*}
            \{(x_i)_{1\leq i\leq n}:x_i\in X_i\text{ for all }1\leq i\leq n\}
        \end{equation*}
        making it, by Definition \ref{dfn:nTuple}, the set of all ordered $n$-tuples that should be in the $n$-fold Cartesian product of $(X_i)_{1\leq i\leq n}$.
        % $\chi$.We can create the set of all partial functions from the full domain to the full range, i.e., the set of all possible mappings of the numbers 1-$n$ to elements of $\{X_1,X_2,\dots,X_n\}$. That would be the set of all ordered pairs $(X_1,X_2,\dots,X_n)$ generated by putting the components in any order. Hold the elements of $X_1,X_2,\dots,X_{n-1}$ constant and iterate through the elements of $X_n$ to create a set. This would be the set of all ordered $n$-tuples with a certain element of $X_1,X_2,\dots,X_{n-1}$ and any element of $X_n$, i.e., the set of all functions that map the numbers 1-$(n-1)$ to a specific element of $X_i:1\leq i\leq n-1$ and the number $n$ to any element of $X_n$. Then move onto the next element of $X_{n-1}$, and so forth. Lastly, unionize.
    \end{proof}
    \item \label{exr:3.5.3}\marginnote{7/31:}Show that the definitions of equality for ordered pair and ordered $n$-tuple obey the reflexivity, symmetry, and transitivity axioms.
    \begin{proof}
        We first treat ordered pairs. To prove that ordered pairs obey the reflexivity axiom, we must show that for any ordered pair $(x,y)$, we have $(x,y)=(x,y)$. By Definition \ref{dfn:orderedPair}, to prove that $(x,y)=(x,y)$, we must verify that $x=x$ and $y=y$. But since $x$ and $y$ are clearly objects of the same type(s) as themselves, respectively, the reflexive axiom of equality (see Section \ref{sse:A.7}) implies that $x=x$ and $y=y$. To prove that ordered pairs obey the symmetry axiom, we must show that for any two ordered pairs $(x,y),(x',y')$, if $(x,y)=(x',y')$, then $(x',y')=(x,y)$. Since $(x,y)$ and $(x',y')$ are both objects of the same type (i.e., both ordered pairs), we know by the symmetry axiom of equality that if $(x,y)=(x',y')$, then $(x',y')=(x,y)$. To prove that ordered pairs obey the transitivity axiom, we must show that for any three ordered pairs $(x,y),(x',y'),(x'',y'')$, if $(x,y)=(x',y')$ and $(x',y')=(x'',y'')$, then $(x,y)=(x'',y'')$. By Definition \ref{dfn:orderedPair}, $(x,y)=(x',y') \Longrightarrow (x=x'\text{ and }y=y')$ and $(x',y')=(x'',y'') \Longrightarrow (x'=x''\text{ and }y'=y'')$. Since $x=x'$ and $x'=x''$, we have by the transitive axiom of equality that $x=x''$. Similarly, we have $y=y''$. Thus, by Definition \ref{dfn:orderedPair}, we have $(x,y)=(x'',y'')$.\par
        \marginnote{8/4:}We now turn our attention to ordered $n$-tuples. To prove that ordered $n$-tuples obey the reflexivity axiom, we must show that for any ordered $n$-tuple $(x_i)_{1\leq i\leq n}$, we have $(x_i)_{1\leq i\leq n}=(x_i)_{1\leq i\leq n}$. By Definition \ref{dfn:nTuple}, to prove that $(x_i)_{1\leq i\leq n}=(x_i)_{1\leq i\leq n}$, we must verify that $x_i=x_i$ for all $1\leq i\leq n$. But since each $x_i$ is clearly an object of the same type as itself, the reflexive axiom of equality implies that $x_i=x_i$ for all $1\leq i\leq n$. To prove that ordered $n$-tuples obey the symmetry axiom, we must show that for any two ordered $n$-tuples $(x_i)_{1\leq i\leq n},(y_i)_{1\leq i\leq n}$, if $(x_i)_{1\leq i\leq n}=(y_i)_{1\leq i\leq n}$, then $(y_i)_{1\leq i\leq n}=(x_i)_{1\leq i\leq n}$. Since $(x_i)_{1\leq i\leq n}$ and $(y_i)_{1\leq i\leq n}$ are both objects of the same type (i.e., both ordered $n$-tuples), we know by the symmetry axiom of equality that if $(x_i)_{1\leq i\leq n}=(y_i)_{1\leq i\leq n}$, then $(y_i)_{1\leq i\leq n}=(x_i)_{1\leq i\leq n}$. To prove that ordered pairs obey the transitivity axiom, we must show that for any three ordered pairs $(x_i)_{1\leq i\leq n},(y_i)_{1\leq i\leq n},(z_i)_{1\leq i\leq n}$, if $(x_i)_{1\leq i\leq n}=(y_i)_{1\leq i\leq n}$ and $(y_i)_{1\leq i\leq n}=(z_i)_{1\leq i\leq n}$, then $(x_i)_{1\leq i\leq n}=(z_i)_{1\leq i\leq n}$. By Definition \ref{dfn:nTuple}, $(x_i)_{1\leq i\leq n}=(y_i)_{1\leq i\leq n} \Longrightarrow x_i=y_i\text{ for all }1\leq i\leq n$ and $(y_i)_{1\leq i\leq n}=(z_i)_{1\leq i\leq n} \Longrightarrow y_i=z_i\text{ for all }1\leq i\leq n$. Since $x_i=y_i$ and $y_i=z_i$ for all $1\leq i\leq n$, we have by the transitive axiom of equality that $x_i=z_i$ for all $1\leq i\leq n$. Thus, by Definition \ref{dfn:nTuple}, we have $(x_i)_{1\leq i\leq n}=(z_i)_{1\leq i\leq n}$.
    \end{proof}
    \item \label{exr:3.5.4}Let $A,B,C$ be sets. Show that $A\times(B\cup C)=(A\times B)\cup(A\times C)$, that $A\times(B\cap C)=(A\times B)\cap(A\times C)$, and that $A\times(B\setminus C)=(A\times B)\setminus(A\times C)$. (One can of course prove similar identities in which the roles of the left and right factors of the Cartesian product are reversed.)
    \begin{proof}
        Because of the number of statements to prove and the logical simplicity of the implications involved, we will shorthand these proofs.\par
        We begin by proving that $A\times(B\cup C)=(A\times B)\cup(A\times C)$, for which it will suffice (Definition \ref{dfn:setEquality}) to show that $x\in A\times(B\cup C) \Longleftrightarrow x\in(A\times B)\cup(A\times C)$, which can be done as follows.
        \begin{align*}
            x\in(A\times(B\cup C)) &\Longleftrightarrow x=(a,b)\text{ for some }a\in A\text{ and }b\in B\cup C \tag*{Definition \ref{dfn:cartesianProduct}}\\
            &\Longleftrightarrow (x=(a,b)\text{ for some }a\in A\text{ and }b\in B\text{ or }x=(a,b)\text{ for some }a\in A\text{ and }b\in C) \tag*{Axiom \ref{axm:pairwiseUnion}}\\
            &\Longleftrightarrow (x\in(A\times B)\text{ or }x\in(A\times C)) \tag*{Definition \ref{dfn:cartesianProduct}}\\
            &\Longleftrightarrow x\in(A\times B)\cup(A\times C) \tag*{Axiom \ref{axm:pairwiseUnion}}
        \end{align*}
        To prove that $A\times(B\cap C)=(A\times B)\cap(A\times C)$, we must verify (Definition \ref{dfn:setEquality}) that $x\in A\times(B\cap C) \Longleftrightarrow x\in(A\times B)\cap(A\times C)$, which can be done as follows.
        \begin{align*}
            x\in A\times(B\cap C) &\Longleftrightarrow x=(a,b)\text{ for some }a\in A\text{ and }b\in B\cap C \tag*{Definition \ref{dfn:cartesianProduct}}\\
            &\Longleftrightarrow (x=(a,b)\text{ for some }a\in A\text{ and }b\in B\text{ and }x=(a,b)\text{ for some }a\in A\text{ and }b\in C) \tag*{Definition \ref{dfn:intersection}}\\
            &\Longleftrightarrow (x\in A\times B\text{ and }x\in A\times C) \tag*{Definition \ref{dfn:cartesianProduct}}\\
            &\Longleftrightarrow x\in(A\times B)\cap(A\times C) \tag*{Definition \ref{dfn:intersection}}
        \end{align*}
        To prove that $A\times(B\setminus C)=(A\times B)\setminus(A\times C)$, Definition \ref{dfn:setEquality} tells us that it will suffice to show that $x\in A\times(B\setminus C) \Longleftrightarrow x\in(A\times B)\setminus(A\times C)$, which can be done as follows.
        \begin{align*}
            x\in A\times(B\setminus C) &\Longleftrightarrow x=(a,b)\text{ for some }a\in A\text{ and }b\in B\setminus C \tag*{Definition \ref{dfn:cartesianProduct}}\\
            &\Longleftrightarrow x=(a,b)\text{ for some }a\in A\text{ and }b\in\{y\in B:y\notin C\} \tag*{Definition \ref{dfn:differenceSets}}\\
            &\Longleftrightarrow (x=(a,b)\text{ for some }a\in A\text{ and }b\in B\text{ and }x\neq(a,b)\text{ for any }a\in A\text{ and }b\in C) \tag*{Axiom \ref{axm:specification}}\\
            &\Longleftrightarrow (x\in A\times B\text{ and }x\notin A\times C) \tag*{Definition \ref{dfn:cartesianProduct}}\\
            &\Longleftrightarrow x\in\{y\in A\times B:y\notin A\times C\} \tag*{Axiom \ref{axm:specification}}\\
            &\Longleftrightarrow x\in(A\times B)\setminus(A\times C) \tag*{Definition \ref{dfn:differenceSets}}
        \end{align*}
    \end{proof}
    \item \label{exr:3.5.5}Let $A,B,C,D$ be sets. Show that $(A\times B)\cap(C\times D)=(A\cap C)\times(B\cap D)$. Is it true that $(A\times B)\cup(C\times D)=(A\cup C)\times(B\cup D)$? Is it true that $(A\times B)\setminus(C\times D)=(A\setminus C)\times(B\setminus D)$?
    \begin{proof}
        To prove that $(A\times B)\cap(C\times D)=(A\cap C)\times(B\cap D)$, Definition \ref{dfn:setEquality} tells us that it will suffice to show that every element $x$ of $(A\times B)\cap(C\times D)$ is an element of $(A\cap C)\times(B\cap D)$ and vice versa. Suppose first that $x\in(A\times B)\cap(C\times D)$. Then by Definition \ref{dfn:intersection}, $x\in(A\times B)$ and $x\in(C\times D)$. By Definition \ref{dfn:cartesianProduct}, this implies that $x=(a,b)$ for some $a\in A$ and $b\in B$, and $x=(c,d)$ for some $c\in C$ and $d\in D$. By transitivity, we have that $(a,b)=(c,d)$, implying by Definition \ref{dfn:orderedPair} that $a=c$ and $b=d$. Thus, we know that $x=(a,b)$ for some element $a$ of $A$ that is also an element of $C$ and for some element $b$ of $B$ that is also an element of $D$. In fact, this implies by Definition \ref{dfn:intersection} that $x=(a,b)$ for some $a\in A\cap C$ and $b\in B\cap D$. Therefore, by Definition \ref{dfn:cartesianProduct}, we have $x\in(A\cap C)\times(B\cap D)$. A similar argument works in the reverse direction.\par
        It is \emph{not} necessarily true\footnote{Note, however, that it \emph{is} true that $(A\times B)\cup(C\times D)\subseteq(A\cup C)\times(B\cup D)$: to verify as much will require (Definition \ref{dfn:subsets}) showing that every element $x$ of $(A\times B)\cup(C\times D)$ is an element of $(A\cup C)\times(B\cup D)$. Suppose that $x\in(A\times B)\cup(C\times D)$. Then by Axiom \ref{axm:pairwiseUnion}, $x\in(A\times B)$ or $x\in(C\times D)$. We now divide into two cases. If $x\in(A\times B)$, then $x=(a,b)$ for some $a\in A$ and $b\in B$ (Definition \ref{dfn:cartesianProduct}). But by Axiom \ref{axm:pairwiseUnion}, $a\in A \Longrightarrow a\in A\cup C$ and $b\in B \Longrightarrow b\in B\cup D$. Thus, we have $x=(a,b)$ for some $a\in A\cup C$ and $b\in B\cup D$, implying by Definition \ref{dfn:cartesianProduct} that $x\in(A\cup C)\times(B\cup D)$. A similar argument can treat the other case.} that $(A\times B)\cup(C\times D)=(A\cup C)\times(B\cup D)$; we may verify this through the following counterexample. Let $A=\{1\}$, $B=\{2\}$, $C=\{3\}$, and $D=\{4\}$. Then $A\times B=\{(1,2)\}$ and $C\times D=\{(3,4)\}$, so $(A\times B)\cup(C\times D)=\{(1,2),(3,4)\}$. However, $A\cup C=\{1,3\}$ and $B\cup D=\{2,4\}$, so $(A\cup C)\times(B\cup D)=\{(1,2),(1,4),(3,2),(3,4)\}$. Thus, for example, $(1,4)\in(A\cup C)\times(B\cup D)$, but $(1,4)\notin(A\times B)\cup(C\times D)$.\par
        Similarly, it is \emph{not} necessarily true\footnote{Note, however, that it \emph{is} true that $(A\setminus C)\times(B\setminus D)\subseteq(A\times B)\setminus(C\times D)$: to verify as much will require (Definition \ref{dfn:subsets}) showing that every element $x$ of $(A\setminus C)\times(B\setminus D)$ is an element of $(A\times B)\setminus(C\times D)$. Suppose that $x\in(A\setminus C)\times(B\setminus D)$. Then by Definition \ref{dfn:cartesianProduct}, $x=(a,b)$ for some $a\in A\setminus C$ and $b\in B\setminus D$. By Definition \ref{dfn:differenceSets} and, subsequently, Axiom \ref{axm:specification}, $a\in A\setminus C \Longrightarrow (a\in A\text{ and }a\notin C)$. Similarly, $b\in B\setminus D \Longrightarrow (b\in B\text{ and }b\notin D)$. Thus, we know that $x=(a,b)$ for some $a\in A$ and $b\in B$, and $x\neq(c,d)$ for any combination of elements $c\in C$ and $d\in D$. Therefore, by Definition \ref{dfn:cartesianProduct}, $x\in(A\times B)$ and $x\notin(C\times D)$, which implies by Axiom \ref{axm:specification} and, subsequently, Definition \ref{dfn:differenceSets} that $x\in(A\times B)\setminus(C\times D)$.} that $(A\times B)\setminus(C\times D)=(A\setminus C)\times(B\setminus D)$; we may verify this through the following counterexample. Let $A=\{1\}$, $B=\{2\}$, $C=\{1\}$, and $D=\{4\}$. Then $A\times B=\{(1,2)\}$ and $C\times D=\{(1,4)\}$, so $(A\times B)\setminus(C\times D)=\{(1,2)\}$. However, $A\setminus C=\emptyset$, so $(A\setminus C)\times(B\setminus D)=\emptyset$ as well. Thus, for example, $(1,2)\in(A\times B)\setminus(C\times D)$, but $(1,2)\notin(A\setminus C)\times(B\setminus D)$.
    \end{proof}
    \item \label{exr:3.5.6}Let $A,B,C,D$ be non-empty sets. Show that $A\times B\subseteq C\times D$ if and only if $A\subseteq C$ and $B\subseteq D$, and that $A\times B=C\times D$ if and only if $A=C$ and $B=D$. What happens if the hypotheses that the $A,B,C,D$ are all non-empty are removed?
    \begin{proof}
        To prove that $A\times B\subseteq C\times D$ if and only if $A\subseteq C$ and $B\subseteq D$, we must verify that the implication "if $A\times B\subseteq C\times D$, then $A\subseteq C$ and $B\subseteq D$" and its converse are both true. Suppose first that $A\times B\subseteq C\times D$. To prove that $A\subseteq C$ and $B\subseteq D$, Definition \ref{dfn:subsets} tells us that it will suffice to show that every element $a$ of $A$ is an element of $C$, and that every element $b$ of $B$ is an element of $D$, respectively. Let $a$ be an arbitrary element of $A$, and let $b$ be an arbitrary element of $B$. Consider $(a,b)$, the ordered pair whose first component is $a$ and and whose second component is $b$. By Definition \ref{dfn:cartesianProduct}, $(a,b)\in(A\times B)$, implying by Definition \ref{dfn:subsets} that $(a,b)\in(C\times D)$. Therefore, by Definition \ref{dfn:cartesianProduct}, $a\in C$ and $b\in D$. Now suppose that $A\subseteq C$ and $B\subseteq D$. To prove that $A\times B\subseteq C\times D$, Definition \ref{dfn:subsets} tells us that it will suffice to show that every element $x$ of $A\times B$ is an element of $C\times D$. Let $x$ be an arbitrary element of $A\times B$. By Definition \ref{dfn:cartesianProduct}, $x=(a,b)$ for some $a\in A$ and $b\in B$. It follows by Definition \ref{dfn:subsets} that $a\in C$ and $b\in D$. Therefore, $x=(a,b)$ for some $a\in C$ and $b\in D$, so by Definition \ref{dfn:cartesianProduct}, $x\in C\times D$.\par
        To prove that $A\times B=C\times D$ if and only if $A=C$ and $B=D$, we must verify that the implication "if $A\times B=C\times D$, then $A=C$ and $B=D$" and its converse are both true. Suppose first that $A\times B=C\times D$. To prove that $A=C$ and $B=D$, Definition \ref{dfn:setEquality} tells us that it will suffice to show that every element $a$ of $A$ is an element of $C$ and vice versa, and that every element $b$ of $B$ is an element of $D$ and vice versa, respectively. Let $a$ be an arbitrary element of $A$, and let $b$ be an arbitrary element of $B$. Consider $(a,b)$, the ordered pair whose first component is $a$ and and whose second component is $b$. By Definition \ref{dfn:cartesianProduct}, $(a,b)\in(A\times B)$, implying by Definition \ref{dfn:setEquality} that $(a,b)\in(C\times D)$. Therefore, by Definition \ref{dfn:cartesianProduct}, $a\in C$ and $b\in D$. A similar argument works in the other direction. Now suppose that $A=C$ and $B=D$. To prove that $A\times B=C\times D$, Definition \ref{dfn:setEquality} tells us that it will suffice to show that every element $x$ of $A\times B$ is an element of $C\times D$ and vice versa. Let $x$ be an arbitrary element of $A\times B$. By Definition \ref{dfn:cartesianProduct}, $x=(a,b)$ for some $a\in A$ and $b\in B$. It follows by Definition \ref{dfn:setEquality} that $a\in C$ and $b\in D$. Therefore, $x=(a,b)$ for some $a\in C$ and $b\in D$, so by Definition \ref{dfn:cartesianProduct}, $x\in C\times D$. A similar argument works in the other direction.\par
        If the hypotheses that $A,B,C,D$ are all non-empty are removed, not much happens. If $C$ or $D$ is empty, though, then $A\subseteq C$ and $B\subseteq D$ becomes a strong enough condition to prove $A\times B=C\times D$ (since the only subset of the empty set is the empty set, $A\subseteq C=\emptyset \Longrightarrow A=\emptyset$, and similarly for $B$ and $D$).
    \end{proof}
    \item \label{exr:3.5.7}Let $X,Y$ be sets, and let $\pi_{X\times Y\to X}:X\times Y\to X$ and $\pi_{X\times Y\to Y}:X\times Y\to Y$ be the maps $\pi_{X\times Y\to X}(x,y):=x$ and $\pi_{X\times Y\to Y}(x,y):=y$; these maps are known as the \textbf{co-ordinate functions} on $X\times Y$. Show that for any functions $f:Z\to X$ and $g:Z\to Y$, there exists a unique function $h:Z\to X\times Y$ such that $\pi_{X\times Y\to X}\circ h=f$ and $\pi_{X\times Y\to Y}\circ h=g$. (Compare this to the last part of Exercise \ref{exr:3.3.8}, and to Exercise \ref{exr:3.1.7}.) This function $h$ is known as the \textbf{direct sum} of $f$ and $g$ and is denoted $h=f\oplus g$.
    \begin{proof}
        Let $h:Z\to X\times Y$ be defined by the following.
        \begin{equation*}
            h(z) := (f(z),g(z))
        \end{equation*}
        To prove that $\pi_{X\times Y\to X}\circ h=f$, Definition \ref{dfn:functionEquality} tells us that it will suffice to show that $\pi_{X\times Y\to X}\circ h$ and $f$ have the same domain and range, and that $(\pi_{X\times Y\to X}\circ h)(z)=f(z)$ for all $z\in Z$. By Definition \ref{dfn:composition}, $\pi_{X\times Y\to X}\circ h:Z\to X$, so $\pi_{X\times Y\to X}\circ h$ and $f$ have the same domain and range. Moving on to the next test, we can see by Definition \ref{dfn:functions} that $f(z)\in X$ and $g(z)\in Y$; thus, $(f(z),g(z))\in X\times Y$ by Definition \ref{dfn:cartesianProduct}. Therefore, by Definition \ref{dfn:composition}, for any $z\in Z$, $(\pi_{X\times Y\to X}\circ h)(z)=\pi_{X\times Y\to X}(h(z))=\pi_{X\times Y\to X}(f(z),g(z))=f(z)$. A similar argument holds for the other case.
        % Suppose for the sake of contradiction that $\tilde{h}$ is a different function satisfying all requirements that $h$ must. Since any candidate for $h$ must map the set $Z$ into the set $X\times Y$, the domain of $\tilde{h}$ is $Z$, and its range is $X\times Y$. Now we must also have $\pi_{X\times Y\to X}\circ\tilde{h}=f$. Since we know that $\pi_{X\times Y\to X}\circ h=f$, we have by transitivity that $\pi_{X\times Y\to X}\circ\tilde{h}=\pi_{X\times Y\to X}\circ h$. We also know that $\pi_{X\times Y\to X}$ is injective with respect to $X$ (if $x\neq x'$, then $\pi_{X\times Y\to X}(x,y)=x\neq x'=\pi_{X\times Y\to X}(x',y')$). Therefore, by Exercise \ref{exr:3.3.4}, $h=\tilde{h}$, a contradiction.\par
    \end{proof}
    \item \label{exr:3.5.8}Let $X_1,\dots,X_n$ be sets. Show that the Cartesian product $\prod_{i=1}^nX_i$ is empty if and only if at least one of the $X_i$ is empty.
    \begin{proof}
        To prove that the Cartesian product $\prod_{i=1}^nX_i$ is empty if and only if at least one of the $X_i$ is empty, we must verify that the implication "if the Cartesian product $\prod_{i=1}^nX_i$ is empty, then at least one of the $X_i$ is empty" and its converse are both true. Suppose first that the Cartesian product $\prod_{i=1}^nX_i$ is empty. Then by the contrapositive of Lemma \ref{lem:finiteChoice}, at least one of the $X_i$ is empty. Now suppose that at least one of the $X_i$ is empty, say $X_j$ for some $j\in\{1\leq i\leq n\}$. Thus, there exists no ordered $n$-tuple $(x_i)_{1\leq i\leq n}$ such that $x_i\in X_i$ for \emph{all} $1\leq i\leq n$ (because there will always be no $x_j\in X_j$, by Axiom \ref{axm:emptyset}). Consequently, there are no elements in the set $\{(x_i)_{1\leq i\leq n}:x_i\in X_i\text{ for all }1\leq i\leq n\}$ since the property "$P((x_i)_{1\leq i\leq n})=x_i\in X_i\text{ for all }1\leq i\leq n$" is false for all $(x_i)_{1\leq i\leq n}$, as previously established (also see Axiom \ref{axm:specification}). Since $\prod_{i=1}^nX_i=\{(x_i)_{1\leq i\leq n}:x_i\in X_i\text{ for all }1\leq i\leq n\}$, the uniqueness of the empty set guarantees that $\prod_{i=1}^nX_i$ is empty.
        % and suppose for the sake of contradiction that all of the $X_i$ are non-empty. Since $n\geq 1$ is a natural number by definition, and for each natural number $1\leq i\leq n$, there exists a nonempty set $X_i$ by hypothesis, Lemma \ref{lem:finiteChoice} asserts there exists an $n$-tuple $(x_i)_{1\leq i\leq n}$ such that $x_i\in X_i$ for all $1\leq i\leq n$. By Axiom \ref{axm:specification}, this $n$-tuple would clearly be an element of the set $\{(x_i)_{1\leq i\leq n}:x_i\in X_i\text{ for all }1\leq i\leq n\}$, thus making it (Definition \ref{dfn:nTuple}) an element of $\prod_{i=1}^nX_i$. But this contradicts Axiom \ref{axm:emptyset}, which asserts that $x\notin\prod_{i=1}^nX_i$ for all objects $x$, including $(x_i)_{1\leq i\leq n}$. Therefore, at least one of the $X_i$ is empty.
        % Then by Definition \ref{dfn:nTuple}, the set $\{(x_i)_{1\leq i\leq n}:x_i\in X_i\text{ for all }1\leq i\leq n\}$ is empty. By Axiom \ref{axm:specification}, this implies that there exists no ordered $n$-tuple $(x_i)_{1\leq i\leq n}$ such that $x_i\in X_i$ for all $1\leq i\leq n$. In other words, for every ordered $n$-tuple $(x_i)_{1\leq i\leq n}$, there is some $x_i$ which is not an element of $X_i$. The only way this is possible is if there is at least one $X_i$ which 
    \end{proof}
    \item \label{exr:3.5.9}\marginnote{8/5:}Suppose that $I$ and $J$ are two sets, and for all $\alpha\in I$ let $A_\alpha$ be a set, and for all $\beta\in J$ let $B_\beta$ be a set. Show that $(\bigcup_{\alpha\in I}A_\alpha)\cap(\bigcup_{\beta\in J}B_\beta)=\bigcup_{(\alpha,\beta)\in I\times J}(A_\alpha\cap B_\beta)$.
    \begin{proof}
        To prove that $(\bigcup_{\alpha\in I}A_\alpha)\cap(\bigcup_{\beta\in J}B_\beta)=\bigcup_{(\alpha,\beta)\in I\times J}(A_\alpha\cap B_\beta)$, Definition \ref{dfn:setEquality} tells us that it will suffice to show that every element $x$ of $(\bigcup_{\alpha\in I}A_\alpha)\cap(\bigcup_{\beta\in J}B_\beta)$ is an element of $\bigcup_{(\alpha,\beta)\in I\times J}(A_\alpha\cap B_\beta)$ and vice versa. Suppose first that $x$ is any element of $(\bigcup_{\alpha\in I}A_\alpha)\cap(\bigcup_{\beta\in J}B_\beta)$. Then by Definition \ref{dfn:intersection}, $x\in\bigcup_{\alpha\in I}A_\alpha$ and $x\in\bigcup_{\beta\in J}B_\beta$. By Equation \ref{eqn:unionElements}, this implies that $x\in A_\alpha$ for some $\alpha\in I$ and $x\in B_\beta$ for some $\beta\in J$. Since $\alpha\in I$ and $\beta\in J$, Definition \ref{dfn:cartesianProduct} implies that the ordered pair $(\alpha,\beta)$ is an element of $I\times J$. Additionally, since $x$ is an element of some $A_\alpha$ and some $B_\beta$, Definition \ref{dfn:intersection} implies that $x\in A_\alpha\cap B_\beta$ for some $(\alpha,\beta)\in I\times J$. Although we officially take an agnostic position as to whether or not the family of sets $A_\alpha\cap B_\beta$ is indexed by one or two labels, if we consider $(\alpha,\beta)$ to be a single label and $I\times J$ to be a single index set, then we have by Equation \ref{eqn:unionElements} that $x\in\bigcup_{(\alpha,\beta)\in I\times J}(A_\alpha\cap B_\beta)$. A similar argument works in the reverse direction.
    \end{proof}
    \item \label{exr:3.5.10}If $f:X\to Y$ is a function, define the \textbf{graph} of $f$ to be the subset of $X\times Y$ defined by $\{(x,f(x)):x\in X\}$. Show that two functions $f:X\to Y$, $\tilde{f}:X\to Y$ are equal if and only if they have the same graph. Conversely, if $G$ is any subset of $X\times Y$ with the property that for each $x\in X$, the set $\{y\in Y:(x,y)\in G\}$ has exactly one element (or in other words, $G$ obeys the \textbf{vertical line test}), show that there is exactly one function $f:X\to Y$ whose graph is equal to $G$.
    \begin{proof}
        To prove that two functions $f:X\to Y$, $\tilde{f}:X\to Y$ are equal if and only if they have the same graph, we must verify that the implication "if $f=\tilde{f}$, then they have the same graph" and its converse are both true. Suppose first that $f=\tilde{f}$. By definition, the graph of $f$ is the subset of $X\times Y$ defined by $\{(x,f(x)):x\in X\}$, and the graph of $\tilde{f}$ is the subset of $X\times Y$ defined by $\{(x,\tilde{f}(x)):x\in X\}$. Thus, the question becomes the problem of demonstrating that $\{(x,f(x)):x\in X\}=\{(x,\tilde{f}(x)):x\in X\}$. By Definition \ref{dfn:setEquality}, this will require showing that every element of $\{(x,f(x)):x\in X\}$ is an element of $\{(x,\tilde{f}(x)):x\in X\}$ and vice versa. Let $(x,f(x))$ be an arbitrary element of $\{(x,f(x)):x\in X\}$. Since $x=x$ reflexively and $f(x)=\tilde{f}(x)$ by Definition \ref{dfn:functionEquality}, Definition \ref{dfn:orderedPair} implies that $(x,f(x))=(x,\tilde{f}(x))$ for some $(x,\tilde{f}(x))$ (such that $x\in X$). Therefore, $(x,f(x))\in\{(x,\tilde{f}(x)):x\in X\}$. A similar argument works in the reverse direction. Now suppose that $f$ and $\tilde{f}$ have the same graph, i.e., that $\{(x,f(x)):x\in X\}=\{(x,\tilde{f}(x)):x\in X\}$. To prove that $f=\tilde{f}$, already knowing that they have the same domain and range, Definition \ref{dfn:functionEquality} tells us that we need only show in addition that $f(x)=\tilde{f}(x)$ for all $x\in X$. Let $x$ be an arbitrary element of $X$. To $x$ corresponds the unique ordered pair $(x,f(x))$, an object which satisfies all requirements necessary to be an element of $\{(x,f(x)):x\in X\}$. By Definition \ref{dfn:setEquality}, this implies that $(x,f(x))\in\{(x,\tilde{f}(x)):x\in X\}$, i.e., that $(x,f(x))=(x,\tilde{f}(x))$ for some $(x,\tilde{f}(x))\in\{(x,\tilde{f}(x)):x\in X\}$. But by Definition \ref{dfn:orderedPair}, this implies that $f(x)=\tilde{f}(x)$.\par
        As to the other part of the question, we must first guarantee that for every suitable $G$, there exists a function $f:X\to Y$ whose graph is equal to $G$. We must then guarantee the uniqueness of this $f$. Let's begin. First off, we are given that $X$ and $Y$ are sets. Now for any $x\in X$, let $P(x,y)$ be a property pertaining to an object $x\in X$ and an object $y\in Y$. More specifically, let $P(x,y)$ be the property $y\in\{y'\in Y:(x,y')\in G\}$. By the definition of $G$, we know that for each $x\in X$, there is exactly one $y\in Y$ for which $P(x,y)$ is true (namely the element $y\in Y$ such that $(x,y)\in G$ --- see Axiom \ref{axm:specification}). Therefore, by Definition \ref{dfn:functions}, we can define a function $f:X\to Y$ defined by $P$ on the domain $X$ and range $Y$. To show that $f$ has $G$ as its graph, we must show that $G=\{(x',f(x')):x'\in X\}$; showing as much will require (Definition \ref{dfn:setEquality}) verifying that every element of $G$ is an element of $\{(x',f(x')):x'\in X\}$ and vice versa. Suppose first that $(x,f(x))$ is an arbitrary element of $G$. Since $x\in X$, Axiom \ref{axm:specification} implies that $(x,f(x))\in\{(x',f(x')):x'\in X\}$. On the other hand, let $(x,f(x))$ be an arbitrary element of $\{(x',f(x')):x'\in X\}$. By the definition of $f$ (specifically, the property $P$), $f(x)\in\{y'\in Y:(x,y')\in G\}$, implying by Axiom \ref{axm:specification} that $(x,f(x))\in G$. Now suppose for the sake of contradiction that $f:X\to Y$ and $\tilde{f}:X\to Y$ are two different functions whose graphs are both equal to $G$, as defined. Since the functions are distinct but have the same domain and range, Definition \ref{dfn:functionEquality} implies the existence of at least one element $x\in X$ such that $f(x)\neq\tilde{f}(x)$. Now by the definition of a graph, we have $(x,f(x))\in G$ and $(x,\tilde{f}(x))\in G$. But since $f(x)\neq\tilde{f}(x)$ (and $f(x)\in Y$ and $\tilde{f}(x)\in Y$), we thus have two distinct elements in the set $\{y\in Y:(x,y)\in G\}$ pertaining to the $x$ in question, a contradiction. Therefore, either $f$ and $\tilde{f}$ have distinct graphs, or they are not different functions, implying that if the graph of both $f$ and $\tilde{f}$ \emph{is} $G$, then $f=\tilde{f}$. All of this together implies that there is exactly one function $f:X\to Y$ corresponding to any suitable $G$.
    \end{proof}
    \item \label{exr:3.5.11}Show that Axiom \ref{axm:powerSets} can in fact be deduced from Exercise \ref{exr:3.4.6}\footnote{Isn't this circular since Exercise \ref{exr:3.4.6} was deduced from Axiom \ref{axm:powerSets}? I suppose it doesn't matter --- it just means that we must postulate either Axiom \ref{axm:powerSets} or Exercise \ref{exr:3.4.6} (we cannot forego axiomatizing both in the same way that we could forego axiomatizing specification [Exercise \ref{exr:3.1.11}]).} and the other axioms of set theory, and thus Exercise \ref{exr:3.4.6} can be used as an alternate formulation of the power set axiom. (Hint: for any two sets $X$ and $Y$, use Exercise \ref{exr:3.4.6} and the axiom of specification to construct the set of all subsets of $X\times Y$ which obey the vertical line test. Then use Exercise \ref{exr:3.5.10} and the axiom of replacement.)
    \begin{proof}
        To verify Axiom \ref{axm:powerSets}, we must show that the collection of all the functions from $X$ to $Y$ is in fact a set. To begin, let $X$ and $Y$ be sets. By Definition \ref{dfn:cartesianProduct} and Exercise \ref{exr:3.5.1}, the Cartesian product $X\times Y$ is a set. Thus, by Exercise \ref{exr:3.4.6}, there exists a set $2^{X\times Y}=\{Z:Z\text{ is a subset of }X\times Y\}$ of all subsets of $X\times Y$ (note that any subset of $X\times Y$ is a set of ordered pairs whose first component is an element $x\in X$ and whose second component is an element $y\in Y$). Now for every $Z\in 2^{X\times Y}$, let $P(Z)$ be a property pertaining to $Z$. More specifically, let $P(Z)$ be the statement "for every $x\in X$, exactly one ordered pair $(x,y)$ is an element of $Z$, where $y$ is some element of $Y$." Thus, by Axiom \ref{axm:specification}, we can create the set $\{Z\in 2^{X\times Y}:P(Z)\text{ is true}\}$, which by definition is the set of all subsets of $X\times Y$ which obey the vertical line test. But by Exercise \ref{exr:3.5.10}, for every set that obeys the vertical line test, there is exactly one function $f:X\to Y$ whose graph is equal to that set. Thus, if we let $P(Z,f)$ be the statement "$Z$ is the graph of $f$," then we know that for each $Z\in\{Z\in 2^{X\times Y}:P(Z)\text{ is true}\}$, there is exactly one, i.e., at most one $f$ for which $P(Z,f)$ is true. Therefore, Axiom \ref{axm:replacement} implies the existence of the set $Y^X=\{f:P(Z,f)\text{ is true for some }Z\in\{Z\in 2^{X\times Y}:P(Z)\text{ is true}\}\}$, which is the set of all functions from $X$ to $Y$.
    \end{proof}
    \item \label{exr:3.5.12}This exercise will establish a rigorous version of Proposition \ref{prp:recursiveDefinitions}. Let $f:\N\times\N\to\N$ be a function, and let $c$ be a natural number. Show that there exists a function $a:\N\to\N$ such that
    \begin{equation*}
        a(0) = c
    \end{equation*}
    and
    \begin{equation*}
        a(n\pplus) = f(n,a(n))\text{ for all }n\in\N
    \end{equation*}
    and furthermore that this function is unique. (Hint: first show inductively, by a modification of the proof of Lemma \ref{lem:finiteChoice}, that for every natural number $N\in\N$, there exists a unique function $a_N:\{n\in\N:n\leq N\}\to\N$ such that $a_N(0)=c$ and $a_N(n\pplus)=f(n,a_N(n))$ for all $n\in\N$ such that $n<N$.) For an additional challenge, prove this result without using any properties of the natural numbers other than the Peano axioms directly (in particular, without using the ordering of the natural numbers, and without appealing to Proposition \ref{prp:recursiveDefinitions}). (Hint: first show inductively, using only the Peano axioms and basic set theory, that for every natural number $N\in\N$, there exists a unique pair $A_N,B_N$ of subsets of $\N$ which obeys the following properties: (a) $A_N\cap B_N=\emptyset$, (b) $A_N\cup B_N=\N$, (c) $0\in A_N$, (d) $N\pplus\in B_N$, (e) Whenever $n\in B_N$, we have $n\pplus\in B_N$, and (f) Whenever $n\in A_N$ and $n\neq N$, we have $n\pplus\in A_N$. Once one obtains these sets, use $A_N$ as a substitute for $\{n\in\N:n\leq N\}$ in the previous argument.)
    \begin{proof}
        We first prove that for every natural number $N\in\N$, there exists a unique function $a_N:\{n\in\N:n\leq N\}\to\N$ such that $a_N(0)=c$ and $a_N(n\pplus)=f(n,a_N(n))$ for all $n\in\N$ such that $n<N$. We induct on $N$ (starting with the base case $N=1$; the claim is also vacuously true with $N=0$ but is not particularly interesting in that case). Let's begin.\par
        When $N=1$, we seek to prove that there exists a unique function $a_1:\{n\in\N:n\leq 1\}\to\N$ such that $a_1(0)=c$ and $a_1(n\pplus)=f(n,a_1(n))$ for all $n\in\N$ such that $n<1$. Since 0 is the only natural number less than 1, we need only show that there exists a unique function $a_1:\{n\in\N:n\leq 1\}\to\N$ such that $a_1(0)=c$ and $a_1(1)=f(0,a_1(0))$. We begin by showing that such a function exists. Let the function $a_1:\{n\in\N:n\leq 1\}\to\N$ be defined by $a_1(0):=c$ and $a_1(1):=f(0,a_1(0))$ (note that 0 and 1 are the only elements in the domain of $a_1$, so defining mappings for only them suffices to define $a_1$). Since $f$ is defined for all ordered pairs whose components are natural numbers, and 0 and $a_1(0)=c$ are natural numbers, $f(0,a_1(0))$ is defined (and is a natural number, i.e., is an element of the defined range of $a_1$). Thus, for the base case $N=1$, there exists a function $a_1:\{n\in\N:n\leq 1\}\to\N$ such that $a_1(0)=c$ and $a_1(n\pplus)=f(n,a_1(n))$ for all $n\in\N$ such that $n<1$. Now we must show that $a_1$ is unique. Suppose for the sake of contradiction that $\tilde{a}_1:\{n\in\N:n\leq 1\}\to\N$ is another function such that $\tilde{a}_1(0)=c$ and $\tilde{a}_1(n\pplus)=f(n,\tilde{a}_1(n))$ for all $n\in\N$ such that $n<1$ and $a_1\neq\tilde{a}_1$. Since the functions are distinct but have the same domain and range, Definition \ref{dfn:functionEquality} implies the existence of at least one element $n\in\{n\in\N:n\leq 1\}$ such that $a_1(n)\neq\tilde{a}_1(n)$. As before, $\tilde{a}_1$ is only defined for 0 and 1, so either $a_1(0)\neq\tilde{a}_1(0)$ or $a_1(1)\neq\tilde{a}_1(1)$. Since both $a_1(0)$ and $\tilde{a}_1(0)$ are equal to $c$ by definition, transitivity implies that $a_1(0)=\tilde{a}_1(0)$, so we must have $a_1(1)\neq\tilde{a}_1(1)$. But $a_1(1)=f(0,a_1(0))=f(0,\tilde{a}_1(0))=\tilde{a}_1(1)$, a contradiction. Therefore, if we assert that $\tilde{a}_1$ is defined like $a_1$, then $a_1=\tilde{a}_1$, proving the uniqueness of $a_1$ and, hence, the claim for the base case $N=1$.\par
        Now suppose inductively that we have proven the claim for $N$, i.e., we know that there exists a unique function $a_N:\{n\in\N:n\leq N\}\to\N$ such that $a_N(0)=c$ and $a_N(n\pplus)=f(n,a_N(n))$ for all $n\in\N$ such that $n<N$. We seek to prove that there exists a unique function $a_{N\pplus}:\{n\in\N:n\leq N\pplus\}\to\N$ such that $a_{N\pplus}(0)=c$ and $a_{N\pplus}(n\pplus)=f(n,a_{N\pplus}(n))$ for all $n\in\N$ such that $n<N\pplus$. We begin by showing that such a function exists. If we define $a_{N\pplus}$ by setting $a_{N\pplus}(n):=a_N(n)$ when $n\in\{n\in\N:n\leq N\}$ and $a_{N\pplus}(N\pplus):=f(N,a_{N\pplus}(N))$ (which we know exists for the same reasons detailed in the proof of the base case), it is clear that $a_{N\pplus}(0)=c$ and $a_{N\pplus}(n\pplus)=f(n,a_{N\pplus}(n))$ for all $n\in\N$ such that $n<N\pplus$, thus closing the induction. The uniqueness of $a_{N\pplus}$ can be verified in much the same way that the uniqueness of $a_1$ was demonstrated for the base case.\par
        Now we tie it all together. Let $a:\N\to\N$ be a function defined by $a(n):=a_n(n)$, where $a_n$ is the appropriate function defined above. Thus, we have
        \begin{equation*}
            a(0) = a_0(0)
            = c
        \end{equation*}
        and
        \begin{equation*}
            a(n\pplus) = a_{n\pplus}(n\pplus)
            = f(n,a_{n\pplus}(n))
            = f(n,a_n(n))
            = f(n,a(n))\text{ for all }n\in\N
        \end{equation*}
        as desired. Now we must verify the uniqueness of $a$. Suppose for the sake of contradiction that $\tilde{a}:\N\to\N$ is another function such that $\tilde{a}(0)=c$ and $\tilde{a}(n\pplus)=f(n,\tilde{a}(n))$ for all $n\in\N$ and $a\neq\tilde{a}$. Since the functions are distinct but have the same domain and range, Definition \ref{dfn:functionEquality} implies the existence of at least one element $n\in\N$ such that $a(n)\neq\tilde{a}(n)$. Thus, to find a contradiction, we must show that $a(n)=\tilde{a}(n)$ for all $n\in\N$, which we may do by induction. For the base case $n=0$, we have $a(0)=c=\tilde{a}(0)$. Now suppose inductively that $a(n)=\tilde{a}(n)$. Then $a(n\pplus)=f(n,a(n))=f(n,\tilde{a}(n))=\tilde{a}(n\pplus)$, which closes the induction. Thus, we have found the desired contradiction. Therefore, if we assert that $\tilde{a}$ is defined like $a$, then $a=\tilde{a}$, proving the uniqueness of $a$ and, hence, completing the proof.\par
        \medskip
        To treat the additional challenge, a couple of lemmas will be helpful.
        \begin{lem}\label{lem:differenceSetsSubsets}
            Let $A,B$ be sets. Then we have $A\setminus B\subseteq A$.
            \begin{proof}
                To prove that $A\setminus B\subseteq A$, Definition \ref{dfn:subsets} tells us that it will suffice to show that every element $x$ of $A\setminus B$ is an element of $A$. Let $x$ be an arbitrary element of $A\setminus B$. By Definition \ref{dfn:differenceSets}, this implies that $x\in\{y\in A:y\notin B\}$. Thus, by Axiom \ref{axm:specification}, $x\in A$ (and $x\notin B$, but this is not relevant to our present discussion).
            \end{proof}
        \end{lem}
        \begin{lem}\label{lem:unionSubsetsNeq}
            \marginnote{\emph{8/6:}}Let $A,B,C,D$ be sets. Then if $A\subsetneq C$, $B\subsetneq D$, and $C\cap D=\emptyset$, we have $A\cup B\subsetneq C\cup D$.
            \begin{proof}
                Suppose $A\subsetneq C$ and $B\subsetneq D$. To show that $A\cup B\subsetneq C\cup D$, Definition \ref{dfn:subsets} tells us that we must show that $A\cup B\subseteq C\cup D$ and $A\cup B\neq C\cup D$. We begin by showing that $A\cup B\subseteq C\cup D$. To do so, Definition \ref{dfn:subsets} tells us that we must verify that every element $x$ of $A\cup B$ is an element $C\cup D$. Let $x$ be an arbitrary element of $A\cup B$. Then by Axiom \ref{axm:pairwiseUnion}, $x\in A$ or $x\in B$. We divide into two cases. If $x\in A$, then Definition \ref{dfn:subsets} implies that $x\in C$, and Axiom \ref{axm:pairwiseUnion} implies that $x\in C\cup D$. Similarly, if $x\in B$, then Definition \ref{dfn:subsets} implies that $x\in D$, and Axiom \ref{axm:pairwiseUnion} implies that $x\in C\cup D$. Therefore, $A\cup B\subseteq C\cup D$. Now to show that $A\cup B\neq C\cup D$, Definition \ref{dfn:setEquality} tells us that we must find an element of $A\cup B$ that is not an element of $C\cup D$, or an element of $C\cup D$ that is not an element of $A\cup B$. Since every element of $A\cup B$ is an element of $C\cup D$ as previously established, we must find an element of $C\cup D$ that is not an element of $A\cup B$. First off, there exists an element (which we shall call $x$) of $C$ that is not an element of $A$\footnote{Note that we could also choose to study an element of $D$ that is not an element of $B$; the choice to study an element of $C$ that is not an element of $A$ is wholly arbitrary.} (if such an element did not exist, then then every element of $C$ would be an element of $A$, which, when combined with the fact that every element of $A$ is an element of $C$, would imply that $A=C$, contradicting the fact that $A\subsetneq C$). We wish to show that this object $x$ is an element of $C\cup D$ but not an element of $A\cup B$. Clearly, since $x\in C$, $x\in C\cup D$ (Axiom \ref{axm:pairwiseUnion}). On the other hand, if we suppose for the sake of contradiction that $x\in B$, we will find that $x\in D$ (Definition \ref{dfn:subsets}), implying by Definition \ref{dfn:intersection} that $x\in C\cap D$ and contradicting the fact that $C\cap D=\emptyset$. Thus, $x\notin B$. Since it is also not an element of $A$ by hypothesis, Axiom \ref{axm:pairwiseUnion} implies that $x\notin A\cup B$. Therefore, $A\cup B\neq C\cup D$ and, hence, $A\cup B\subsetneq C\cup D$.
            \end{proof}
        \end{lem}
        \marginnote{8/5:}Now for the primary proof (of the additional challenge). We first prove that for every natural number $N\in\N$, there exists a unique pair $A_N,B_N$ of subsets of $\N$ which obeys the following properties: (a) $A_N\cap B_N=\emptyset$, (b) $A_N\cup B_N=\N$, (c) $0\in A_N$, (d) $N\pplus\in B_N$, (e) Whenever $n\in B_N$, we have $n\pplus\in B_N$, and (f) Whenever $n\in A_N$ and $n\neq N$, we have $n\pplus\in A_N$. We induct on $N$. Let's begin.\par
        For the base case $N=0$, let $A_0=\{0\}$ and let $B_0=\N\setminus A_0$. First, we show that $A_0,B_0$ are in fact subsets of $\N$. By Axiom \ref{axm:singletonPair}, 0 is the only element of $A_0$. Thus, by Definition \ref{dfn:subsets}, to show that $A_0\subseteq\N$, we need only show that $0\in\N$. But by Axiom \ref{axm:infinity}, $0\in\N$. On the other hand, Lemma \ref{lem:differenceSetsSubsets} implies that $B_0=\N\setminus A_0\subseteq\N$. Next, we check that $A_0,B_0$ satisfy the properties (a)-(f).
        \begin{enumerate}
            \item By Exercise \ref{exr:3.1.6g}, we have $A_0\cap B_0=A_0\cap(\N\setminus A_0)=\emptyset$.
            \item By Exercise \ref{exr:3.1.6g}, we also have $A_0\cup B_0=A_0\cup(\N\setminus A_0)=\N$.
            \item By Axiom \ref{axm:singletonPair}, $0\in A_0$.
            \item By Axiom \ref{axm:npplus}, $0\pplus$ is a natural number, so Axiom \ref{axm:infinity} implies that $0\pplus\in\N$. Since Axiom \ref{axm:0NotSuccessor} asserts that $0\pplus\neq 0$, $0\pplus\notin A_0$ by Axiom \ref{axm:singletonPair}. Thus, by Axiom \ref{axm:specification}, $0\pplus\in\{n\in\N:n\notin A_0\}$. Consequently, by Definition \ref{dfn:differenceSets}, $0\pplus\in\N\setminus A_0$, implying that $0\pplus\in B_0$.
            \item For any $n\in B_0$, $n\in\N$ (Definition \ref{dfn:subsets}). Thus, by Axiom \ref{axm:infinity}, $n\pplus\in\N$. Since $n\pplus\neq 0$ (Axiom \ref{axm:0NotSuccessor}), $n\pplus\notin A_0$ by Axiom \ref{axm:singletonPair}. Thus, by Axiom \ref{axm:specification} and subsequently Definition \ref{dfn:differenceSets}, $n\pplus\in\N\setminus A_0$, which implies that $n\pplus\in B_0$.
            \item By Axiom \ref{axm:singletonPair}, 0 is the only element of $A_0$, so there is no $n\in A_0$ such that $n\neq 0$. Thus, the statement "whenever $n\in A_0$ and $n\neq 0$, we have $n\pplus\in A_0$" is vacuously true.
        \end{enumerate}
        Last, we prove the uniqueness of $A_0$ and $B_0$. Suppose for the sake of contradiction that there exist a pair of subsets ${A_0}'$ and ${B_0}'$ of $\N$ that satisfy all the same properties as $A_0$ and $B_0$ and such that ${A_0}'\neq A_0$ and ${B_0}'\neq B_0$. We begin by examining the construction of ${B_0}'$. By property (d), $0\pplus\in{B_0}'$, and by property (e), whenever $n\in{B_0}'$, we have $n\pplus\in{B_0}'$; thus, ${B_0}'$ contains the successors of every natural number. Now consider ${A_0}'$. By property (c), $0\in{A_0}'$. Additionally, since $A_0\neq{A_0}'$ by hypothesis, Definition \ref{dfn:setEquality} implies that there is some element of $A_0$ that is not in ${A_0}'$ or there is some element of ${A_0}'$ that is not in $A_0$. However, since 0 is the only element of $A_0$ (Axiom \ref{axm:singletonPair}) and $0\in{A_0}'$ as previously established, there is no element of $A_0$ that is not in ${A_0}'$. Thus, there must be some element $n\neq 0$ of ${A_0}'$ that is not in $A_0$. Let's take a closer look at this $n$: since ${A_0}'$ is a subset of $\N$, $n$ is a natural number. Since $n\neq 0$, $n$ is the successor of some natural number (Axiom \ref{axm:0NotSuccessor}); this implies that $n$ is an element of ${B_0}'$. Thus, $n\in{A_0}'\cap{B_0}'$ (Definition \ref{dfn:intersection}), contradicting property (a). Therefore, if we assert that ${A_0}'$ and ${B_0}'$ satisfy all the same properties as $A_0$ and $B_0$, then ${A_0}'=A_0$ and ${B_0}'=B_0$, proving the uniqueness of $A_0$ and $B_0$.\par
        % Now suppose inductively that we have proven that for some natural number $N$, there exists a unique pair $A_N,B_N$ of subsets of $\N$ which obeys properties (a)-(f) above. We seek to prove the same for $N\pplus$. Let $A_{N\pplus}=A_N\cup\{N\pplus\}$ and let $B_{N\pplus}=B_N\setminus\{N\pplus\}$. First, we show that $A_{N\pplus},B_{N\pplus}$ are in fact subsets of $\N$. Since $N$ is a natural number, by Axiom \ref{axm:npplus}, $N\pplus$ is a natural number. Thus, since the singleton set $\{N\pplus\}$ contains only $N\pplus$ (Axiom \ref{axm:singletonPair}), $\{N\pplus\}\subseteq\N$. This, combined with the fact that $A_N\subseteq\N$ by hypothesis, implies that $A_{N\pplus}=A_N\cup\{N\pplus\}\subseteq\N$ (Exercise \ref{exr:3.1.7}). On the other hand, Lemma \ref{lem:differenceSetsSubsets} implies that $B_{N\pplus}=B_N\setminus\{N\pplus\}\subseteq B_N$. Since $B_N\subseteq\N$ by hypothesis, Proposition \ref{prp:subsetTransitive} implies that $B_{N\pplus}\subseteq\N$. Next, we check that $A_N,B_N$ satisfy the properties (a)-(f).
        % \begin{enumerate}
        %     \item To prove that $A_{N\pplus}\cap B_{N\pplus}=\emptyset$, it will suffice to show that $x\notin A_{N\pplus}\cap B_{N\pplus}$ for all objects $x$ (because of the uniqueness of the empty set). Suppose for the sake of contradiction that there exists some object $x$ such that $x\in A_{N\pplus}\cap B_{N\pplus}$. Then by Definition \ref{dfn:intersection}, $x\in A_{N\pplus}$ and $x\in B_{N\pplus}$. Thus, $x\in A_N\cup\{N\pplus\}$ and $x\in B_N\setminus\{N\pplus\}$. By Axiom \ref{axm:pairwiseUnion}, and Definition \ref{dfn:differenceSets} followed by Axiom \ref{axm:specification}, respectively, $x\in A_N$ or $x\in\{N\pplus\}$, and $x\in B_N$ and $x\notin\{N\pplus\}$. Since $x\notin\{N\pplus\}$, $x\in A_N$. But $x\in A_N$ and $x\in B_N$ imply by Definition \ref{dfn:intersection} that $x\in A_N\cap B_N$, contradicting the fact that $A_N\cap B_N=\emptyset$, which we know to be a true statement for $A_N$ and $B_N$ since $A_N$ and $B_N$ obey property (a) by the induction hypothesis. Therefore, there exists no object $x\in A_{N\pplus}\cap B_{N\pplus}$, implying that $A_{N\pplus}\cap B_{N\pplus}=\emptyset$.
        %     \item To prove that $A_{N\pplus}\cup B_{N\pplus}=\N$, it will suffice to show that $A_{N\pplus}\cup B_{N\pplus}=A_N\cup B_N$ since $A_N\cup B_N=\N$ by the induction hypothesis. To show that $A_{N\pplus}\cup B_{N\pplus}=A_N\cup B_N$, we must verify (Definition \ref{dfn:setEquality}) that every element $x$ of $A_{N\pplus}\cup B_{N\pplus}$ is an element of $A_N\cup B_N$ and vice versa. Suppose first that $x\in A_{N\pplus}\cup B_{N\pplus}$. Then by Axiom \ref{axm:pairwiseUnion}, $x\in A_{N\pplus}$ or $x\in B_{N\pplus}$. Thus, $x\in A_N\cup\{N\pplus\}$ or $x\in B_N\setminus\{N\pplus\}$. By Axiom \ref{axm:pairwiseUnion}, and Definition \ref{dfn:differenceSets} followed by Axiom \ref{axm:specification}, respectively, $x\in A_N$ or $x\in\{N\pplus\}$, or $x\in B_N$ and $x\notin\{N\pplus\}$. We now divide into two cases. Suppose that $x\in A_N$ or $x\in\{N\pplus\}$. We now divide into two cases again. If $x\in A_N$, then Axiom \ref{axm:pairwiseUnion} implies that $x\in A_N\cup B_N$. On the other hand, if $x\in\{N\pplus\}$, then since $\{N\pplus\}\subseteq\N$ as previously established, Definition \ref{dfn:subsets} tells us that $x\in\N$. But since $\N=A_N\cup B_N$, $x\in A_N\cup B_N$. Now suppose that $x\in B_N$ and $x\notin\{N\pplus\}$. Since $x\in B_N$, by Axiom \ref{axm:pairwiseUnion}, $x\in A_N\cup B_N$. A similar argument works in the reverse direction.
        %     \item Since $0\in A_N$, Axiom \ref{axm:pairwiseUnion} implies that $0\in A_N\cup\{N\pplus\}=A_{N\pplus}$.
        %     \item Since $B_N$ obeys property (d) by hypothesis, we know that $N\pplus\in B_N$. Thus, since $B_N$ also obeys property (e) by hypothesis, $(N\pplus)\pplus\in B_N$. Now by Axiom \ref{axm:successorDistinctness}, $(N\pplus)\pplus\neq N\pplus$, so by Axiom \ref{axm:singletonPair}, $(N\pplus)\pplus\notin\{N\pplus\}$. Thus, by Axiom \ref{axm:specification} and subsequently Definition \ref{dfn:differenceSets}, $(N\pplus)\pplus\in B_N\setminus\{N\pplus\}=B_{N\pplus}$.
        %     \item For any $n\in B_{N\pplus}$, $n\in B_N$ (Definition \ref{dfn:differenceSets} and Axiom \ref{axm:specification}). Since $B_N$ obeys property (e), $n\pplus\in B_N$. Since $n\neq N$, $n\pplus\notin\{N\pplus\}$ (Axioms \ref{axm:successorDistinctness} and \ref{axm:singletonPair}). Thus, by Axiom \ref{axm:specification} and subsequently Definition \ref{dfn:differenceSets}, $n\pplus\in B_N\setminus\{N\pplus\}=B_{N\pplus}$.
        % \end{enumerate}
        Now suppose inductively that we have proven that for some natural number $N$, there exists a unique pair $A_N,B_N$ of subsets of $\N$ which obeys properties (a)-(f) above. We seek to prove the same for $N\pplus$. Let $A_{N\pplus}=A_N\cup\{N\pplus\}$ and let $B_{N\pplus}=\N\setminus A_{N\pplus}$. First, we show that $A_{N\pplus},B_{N\pplus}$ are in fact subsets of $\N$. Since $N$ is a natural number, by Axiom \ref{axm:npplus}, $N\pplus$ is a natural number, i.e., is an element of $\N$ (Axiom \ref{axm:infinity}). Thus, since the singleton set $\{N\pplus\}$ contains only $N\pplus$ (Axiom \ref{axm:singletonPair}), $\{N\pplus\}\subseteq\N$ (Definition \ref{dfn:subsets}). This, combined with the fact that $A_N\subseteq\N$ by hypothesis, implies that $A_{N\pplus}=A_N\cup\{N\pplus\}\subseteq\N$ (Exercise \ref{exr:3.1.7}). On the other hand, Lemma \ref{lem:differenceSetsSubsets} implies that $B_{N\pplus}=\N\setminus A_N\subseteq\N$. Next, we check that $A_N,B_N$ satisfy the properties (a)-(f).
        \begin{enumerate}
            \item By Exercise \ref{exr:3.1.6g}, we have $A_{N\pplus}\cap B_{N\pplus}=A_{N\pplus}\cap(\N\setminus A_{N\pplus})=\emptyset$.
            \item By Exercise \ref{exr:3.1.6g}, we also have $A_{N\pplus}\cup B_{N\pplus}=A_{N\pplus}\cup(\N\setminus A_{N\pplus})=\N$.
            \item Since $0\in A_N$, Axiom \ref{axm:pairwiseUnion} implies that $0\in A_N\cup\{N\pplus\}=A_{N\pplus}$.
            \item As previously established, $N\pplus$ is a natural number. Thus, by Axiom \ref{axm:npplus}, $(N\pplus)\pplus$ is a natural number. Consequently, Axiom \ref{axm:infinity} implies that $(N\pplus)\pplus\in\N$. Now suppose for the sake of contradiction that $(N\pplus)\pplus\in A_{N\pplus}$. Then $(N\pplus)\pplus\in A_N\cup\{N\pplus\}$, so by Axiom \ref{axm:pairwiseUnion}, $(N\pplus)\pplus\in A_N$ or $(N\pplus)\pplus\in\{N\pplus\}$. Since $(N\pplus)\pplus\neq N\pplus$ (Axiom \ref{axm:successorDistinctness}), $(N\pplus)\pplus\notin\{N\pplus\}$ (Axiom \ref{axm:singletonPair}); thus, we must have $(N\pplus)\pplus\in A_N$. But by property (d), $N\pplus\in B_N$, implying by property (e) that $(N\pplus)\pplus\in B_N$. Thus, $(N\pplus)\pplus\in A_N\cap B_N$ by Definition \ref{dfn:intersection}, implying that $A_N\cap B_N\neq\emptyset$, which contradicts property (a). Thus, $(N\pplus)\pplus\notin A_{N\pplus}$. Therefore, by Axiom \ref{axm:specification} and Definition \ref{dfn:differenceSets}, $(N\pplus)\pplus\in\N\setminus A_{N\pplus}=B_{N\pplus}$.
            \item We want to show that for any $n\in B_{N\pplus}$, we have $n\pplus\in B_{N\pplus}$. As such, let $n$ be an arbitrary element of $B_{N\pplus}$. By Definition \ref{dfn:differenceSets} and Axiom \ref{axm:specification}, to show that $n\pplus\in B_{N\pplus}$, it will suffice to show that $n\pplus\in\N$ and $n\pplus\notin A_{N\pplus}$. We first show that $n\pplus\in\N$. Since $n\in B_{N\pplus}$, Definition \ref{dfn:differenceSets} and Axiom \ref{axm:specification} imply that $n\in\N$. Thus, by Axiom \ref{axm:infinity}, $n\pplus\in\N$. We now turn our attention to demonstrating that $n\pplus\notin A_{N\pplus}$, which we may do by the following contradiction argument. Suppose for the sake of contradiction that $n\pplus\in A_{N\pplus}$. Then by Axiom \ref{axm:pairwiseUnion}, $n\pplus\in A_N$ or $n\pplus\in\{N\pplus\}$. We now divide into two cases. Suppose first that $n\pplus\in A_N$. Since $n\in B_{N\pplus}=\N\setminus A_{N\pplus}=\N\setminus(A_N\cup\{N\pplus\})$ by hypothesis, $n\in(\N\setminus A_N)\cap(\N\setminus\{N\pplus\})=B_N\cap(\N\setminus\{N\pplus\})$ by Exercise \ref{exr:3.1.6h}. Thus, by Definition \ref{dfn:intersection}, $n\in B_N$, so by property (e), $n\pplus\in B_N$. Since $n\pplus$ is also an element of $A_{N\pplus}$ by hypothesis, this implies by Definition \ref{dfn:intersection} that $n\in A_{N\pplus}\cap B_{N\pplus}$, contradicting the fact that $A_{N\pplus}\cap B_{N\pplus}=\emptyset$ by property (a). Now suppose that $n\pplus\in\{N\pplus\}$. Then Axiom \ref{axm:singletonPair} implies that $n\pplus=N\pplus$. Thus, Axiom \ref{axm:successorDistinctness} implies that $n=N$. Since $N\in A_N$ (if $N$ were an element of $B_N$ then either $N=N\pplus$ or $N$ equals some iterated successor of $N\pplus$; both eventualities are forebidden by Axioms \ref{axm:0NotSuccessor} and \ref{axm:successorDistinctness}), we have that $n\in A_N$. Thus, by Axiom \ref{axm:pairwiseUnion}, $n\in A_{N\pplus}$. Since $n$ is also an element of $B_{N\pplus}$ by hypothesis, this implies by Definition \ref{dfn:intersection} that $n\in A_{N\pplus}\cap B_{N\pplus}$, contradicting the fact that $A_{N\pplus}\cap B_{N\pplus}=\emptyset$ by property (a). Therefore, $n\pplus\notin A_{N\pplus}$.
            % \item For any $n\in B_{N\pplus}$, $n\in\N$ (Definition \ref{dfn:subsets}). Thus, by Axiom \ref{axm:infinity}, $n\pplus\in\N$. Additionally, $n\in B_{N\pplus}$ implies that $n\notin A_{N\pplus}$ (property (a)). Thus, property (f) implies that $n\pplus\notin A_{N\pplus}$. Therefore, by Axiom \ref{axm:specification} and Definition \ref{dfn:differenceSets}, $n\pplus\in\N\setminus A_{N\pplus}=B_{N\pplus}$.
            \item For any $n\in A_{N\pplus}$, Axiom \ref{axm:pairwiseUnion} implies that $n\in A_N$ or $n\in\{N\pplus\}$. We divide into two cases. Suppose first that $n\in A_N$. Then either $n=N$ or $n\neq N$. We divide into two cases again. If $n=N$, then $n\pplus=N\pplus\in\{N\pplus\}$ as previously established, implying that $n\pplus\in A_N\cup\{N\pplus\}=A_{N\pplus}$ by Axiom \ref{axm:pairwiseUnion}. On the other hand, if $n\neq N$, then $n\pplus\in A_N$ since $A_N$ obeys property (f), implying by Axiom \ref{axm:pairwiseUnion} again that $n\pplus\in A_{N\pplus}$. Now suppose that $n\in\{N\pplus\}$. Then by Axiom \ref{axm:singletonPair}, $n=N\pplus$, and property (f) does not apply. Therefore, whenever $n\in A_{N\pplus}$ and $n\neq N\pplus$, we have $n\pplus\in A_{N\pplus}$.
        \end{enumerate}
        \marginnote{8/6:}Last, we prove the uniqueness of $A_{N\pplus}$ and $B_{N\pplus}$. Suppose for the sake of contradiction that there exist a pair of subsets ${A_{N\pplus}}'$ and ${B_{N\pplus}}'$ of $\N$ that satisfy all the same properties as $A_{N\pplus}$ and $B_{N\pplus}$ and such that ${A_{N\pplus}}'\neq A_{N\pplus}$ and ${B_{N\pplus}}'\neq B_{N\pplus}$. By property (c), $0\in{A_{N\pplus}}'$, and by property (f), whenever $n\in{A_{N\pplus}}'$ and $n\neq N\pplus$, $n\pplus\in{A_{N\pplus}}'$. Since this also defines exactly the elements of $A_{N\pplus}$, we know that at least every element of $A_{N\pplus}$ is an element of ${A_{N\pplus}}'$. This, combined with the fact that ${A_{N\pplus}}'\neq A_{N\pplus}$, implies that $A_{N\pplus}\subsetneq{A_{N\pplus}}'$ (Definition \ref{dfn:subsets}). Similarly, we can use properties (d) and (e) to show that $B_{N\pplus}\subsetneq{B_{N\pplus}}'$. These two proper subset relations combined with the fact that ${A_{N\pplus}}'\cap{B_{N\pplus}}'=\emptyset$ (property (a)) imply by Lemma \ref{lem:unionSubsetsNeq} that $A_{N\pplus}\cup B_{N\pplus}\subsetneq{A_{N\pplus}}'\cup{B_{N\pplus}}'$. But by property (b), this implies that $\N\subsetneq\N$, i.e., that $\N\neq\N$, a contradiction. Therefore, if we assert that ${A_{N\pplus}}'$ and ${B_{N\pplus}}'$ have all the same properties as $A_{N\pplus}$ and $B_{N\pplus}$, then ${A_{N\pplus}}'=A_{N\pplus}$ and ${B_{N\pplus}}'=B_{N\pplus}$, proving the uniqueness of $A_{N\pplus}$ and $B_{N\pplus}$.\par
        We now prove that for every natural number $N\in\N$, there exists a unique function $a_N:A_N\to\N$ such that $a_N(0)=c$ and $a_N(n\pplus)=f(n,a_N(n))$ for all $n\in A_N$ such that $n\neq N$. We induct on $N$ (starting with the base case $N=1$; the claim is also vacuously true with $N=0$ but is not particularly interesting in that case). Let's begin.\par
        When $N=1$, we seek to prove that there exists a unique function $a_1:A_1\to\N$ such that $a_1(0)=c$ and $a_1(n\pplus)=f(n,a_1(n))$ for all $n\in A_1$ such that $n\neq 1$. By the preceding proof of the existence and uniqueness of $A_N,B_N$, we know that $A_1=A_0\cup\{1\}=\{0\}\cup\{1\}$. By Exercise \ref{exr:3.1.3}, $\{0\}\cup\{1\}=\{0,1\}$, so $A_1$ is the pair set formed by 0 and 1. Now since 0 is the only element of $A_1$ that is not equal to 1 (we know that $0\neq 1$ by Axiom \ref{axm:0NotSuccessor}), we need only show that there exists a unique function $a_1:A_1\to\N$ such that $a_1(0)=c$ and $a_1(1)=f(0,a_1(0))$. We begin by showing that such a function exists. Let the function $a_1:A_1\to\N$ be defined by $a_1(0):=c$ and $a_1(1):=f(0,a_1(0))$ (since 0 and 1 are the only elements of $A_1$ by Axiom \ref{axm:singletonPair}, defining mappings for only them suffices to define $a_1$). Since $f$ is defined for all ordered pairs whose components are natural numbers, and 0 and $a_1(0)=c$ are natural numbers, $f(0,a_1(0))$ is defined (and is a natural number, i.e., is an element of the defined range of $a_1$). Thus, for the base case $N=1$, there exists a function $a_1:A_1\to\N$ such that $a_1(0)=c$ and $a_1(n\pplus)=f(n,a_1(n))$ for all $n\in A_1$ such that $n\neq 1$. Now we must show that $a_1$ is unique. Suppose for the sake of contradiction that $\tilde{a}_1:A_1\to\N$ is another function such that $\tilde{a}_1(0)=c$ and $\tilde{a}_1(n\pplus)=f(n,\tilde{a}_1(n))$ for all $n\in A_1$ such that $n\neq 1$ and $a_1\neq\tilde{a}_1$. Since the functions are distinct but have the same domain and range, Definition \ref{dfn:functionEquality} implies the existence of at least one element $n\in A_1$ such that $a_1(n)\neq\tilde{a}_1(n)$. As before, $\tilde{a}_1$ is only defined for 0 and 1, so either $a_1(0)\neq\tilde{a}_1(0)$ or $a_1(1)\neq\tilde{a}_1(1)$. Since both $a_1(0)$ and $\tilde{a}_1(0)$ are equal to $c$ by definition, transitivity implies that $a_1(0)=\tilde{a}_1(0)$, so we must have $a_1(1)\neq\tilde{a}_1(1)$. But $a_1(1)=f(0,a_1(0))=f(0,\tilde{a}_1(0))=\tilde{a}_1(1)$, a contradiction. Therefore, if we assert that $\tilde{a}_1$ is defined like $a_1$, then $a_1=\tilde{a}_1$, proving the uniqueness of $a_1$ and, hence, the claim for the base case $N=1$.\par
        Now suppose inductively that we have proven the claim for $N$, i.e., we know that there exists a unique function $a_N:A_N\to\N$ such that $a_N(0)=c$ and $a_N(n\pplus)=f(n,a_N(n))$ for all $n\in A_N$ such that $n\neq N$. We seek to prove that there exists a unique function $a_{N\pplus}:A_{N\pplus}\to\N$ such that $a_{N\pplus}(0)=c$ and $a_{N\pplus}(n\pplus)=f(n,a_{N\pplus}(n))$ for all $n\in A_{N\pplus}$ such that $n\neq N\pplus$. We begin by showing that such a funciton exists. If we define $a_{N\pplus}$ by setting $a_{N\pplus}(n):=a_N(n)$ when $n\in A_N$ (recall that since $A_{N\pplus}=A_N\cup\{N\pplus\}$, every element save one of $A_{N\pplus}$ is an element of $A_N$ and, thus, is mapped by $a_N$ by hypothesis) and $a_{N\pplus}(N\pplus):=f(N,a_{N\pplus}(N))$ (which we know exists for the same reasons detailed in the proof of the base case), it is clear that $a_{N\pplus}(0)=c$ and $a_{N\pplus}(n\pplus)=f(n,a_{N\pplus}(n))$ for all $n\in A_{N\pplus}$ such that $n\neq N\pplus$, thus closing the induction. The uniqueness of $a_{N\pplus}$ can be verified in much the same way that the uniqueness of $a_1$ was demonstrated for the base case.\par
        Now we tie it all together. Let $a:\N\to\N$ be a function defined by $a(n):=a_n(n)$, where $a_n$ is the appropriate function defined above. Thus, we have
        \begin{equation*}
            a(0) = a_0(0) = c
        \end{equation*}
        and
        \begin{equation*}
            a(n\pplus) = a_{n\pplus}(n\pplus)
            = f(n,a_{n\pplus}(n))
            = f(n,a_n(n))
            = f(n,a(n))\text{ for all }n\in\N
        \end{equation*}
        as desired. Now we must verify the uniqueness of $a$. Suppose for the sake of contradiction that $\tilde{a}:\N\to\N$ is another function such that $\tilde{a}(0)=c$ and $\tilde{a}(n\pplus)=f(n,\tilde{a}(n))$ for all $n\in\N$ and $a\neq\tilde{a}$. Since the functions are distinct but have the same domain and range, Definition \ref{dfn:functionEquality} implies the existence of at least one element $n\in\N$ such that $a(n)\neq\tilde{a}(n)$. Thus, to find a contradiction, we must show that $a(n)=\tilde{a}(n)$ for all $n\in\N$, which we may do by induction. For the base case $n=0$, we have $a(0)=c=\tilde{a}(0)$. Now suppose inductively that $a(n)=\tilde{a}(n)$. Then $a(n\pplus)=f(n,a(n))=f(n,\tilde{a}(n))=\tilde{a}(n\pplus)$, which closes the induction. Thus, we have found the desired contradiction. Therefore, if we assert that $\tilde{a}$ is defined like $a$, then $a=\tilde{a}$, proving the uniqueness of $a$ and, hence, completing the additional challenge.
    \end{proof}
    \item \label{exr:3.5.13}The purpose of this exercise is to show that there is essentially only one version of the natural number system in set theory (cf. the note in Section \ref{sse:2.1} on isomorphic number systems). Suppose we have a set $\N'$ of "alternative natural numbers," an "alternative zero" $0'$, and an "alternative increment operation" which takes any alternative natural number $n'\in\N'$ and returns another alternative natural number $n'\pplus'\in\N'$, such that the Peano axioms (Axioms 2.1-2.5) all hold with the natural numbers, zero, and increment replaced by their alternative counterparts. Show that there exists a bijection $f:\N\to\N'$ from the natural numbers to the alternative natural numbers such that $f(0)=0'$, and such that for any $n\in\N$ and $n'\in\N'$, we have $f(n)=n'$ if and only if $f(n\pplus)=n'\pplus'$. (Hint: use Exercise \ref{exr:3.5.12}.)
    \begin{proof}
        Let $g:\N\times\N'\to\N'$ be defined by $g(a,b)=b\pplus'$. Then by Exercise \ref{exr:3.5.12}, there exists a function $f:\N\to\N'$ such that $f(0)=c$ and $f(n\pplus)=g(n,f(n))$ for all $n\in\N$. If we let $c=0'$, then we have $f(0)=0'$. On the other hand, if $f(n)=n'$, then we have $f(n\pplus)=g(n,f(n))=g(n,n')=n'\pplus'$ and vice versa.
    \end{proof}
\end{enumerate}




\end{document}