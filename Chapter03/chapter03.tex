\documentclass[../main.tex]{subfiles}

\pagestyle{main}
\renewcommand{\chaptermark}[1]{\markboth{\chaptername\ \thechapter: #1}{}}
\setcounter{chapter}{2}
\externaldocument{../main}

\begin{document}




\chapter{Set Theory}
\begin{itemize}
    \item \marginnote{7/4:}Set theory will be frequently used in the subsequent chapters; it is part of the foundation of almost every other branch of mathematics.
    \begin{itemize}
        \item Note that Euclidean geometry will not be defined --- we will use the Cartesian coordinate system's parallel with the real numbers instead.
    \end{itemize}
    \item This chapter covers the elementary aspects, Chapter \ref{chr:8} covers more advanced topics, and the finer subtleties are well beyond the scope of this text.
\end{itemize}



\section{Fundamentals}\label{sse:3.1}
\begin{itemize}
    \item We define sets axiomatically, as we did with the natural numbers\footnote{Note that the following list of axioms will be somewhat overcomplete, as some axioms may be derived from others. However, this is helpful for pedagogical reasons, and there is no real harm being done.}.
    \begin{axm}[Sets are objects]\label{axm:setsAreObjects}
        If $A$ is a set, then $A$ is also an object. In particular, given two sets $A$ and $B$, it is meaningful to ask whether $A$ is also an element of $B$.
    \end{axm}
    \item Note that while all sets are objects, not all objects are sets.
    \begin{itemize}
        \item For example, $1$ is not a set while $\{1\}$ is.
        \item Note, though, that \textbf{pure set theory} considers all objects to be sets. However, impure set theory (where some objects are not sets) is conceptually easier to deal with.
        \begin{itemize}
            \item Since both types are equal for the purposes of mathematics, we will take a middle-ground approach.
        \end{itemize}
    \end{itemize}
    \item If $x,y$ are objects and $A$ a set, then the statement $x\in A$ is either true or false. Note that $x\in y$ is neither true nor false, simply meaningless.
    \item We now define equality for sets.
    \begin{dfn}[Equality of sets]\label{dfn:setEquality}
        Two sets $A$ and $B$ are equal, $A=B$, iff every element of $A$ is an element of $B$ and vice versa. To put it another way, $A=B$ if and only if every element $x$ of $A$ belongs also to $B$, and every element $y$ of $B$ belongs also to $A$.
    \end{dfn}
    \item Note that this implies that repetition of elements does not effect equality ($\{3,3\}=\{3\}$, for example).
    \item It can be proven that this notion of equality is reflexive, symmetric, and transitive (see Exercise \ref{exr:3.1.1}).
    \item \marginnote{7/14:}Since $x\in A$ and $A=B$ implies $x\in B$, the $\in$ relation obeys the axiom of substitution as well.
    \begin{itemize}
        \item Thus, any operation defined in terms of the $\in$ relation obeys the axiom of substitution.
    \end{itemize}
    \item We define sets in an analogous way to how we defined natural numbers from 0, onward.
    \begin{axm}[Empty set]\label{axm:emptyset}
        There exists a set $\emptyset$ \textup{(}also denoted $\{\}$\textup{)}, known as the empty set, which contains no elements, i.e., for every object $x$, we have $x\notin\emptyset$.
    \end{axm}
    \begin{itemize}
        \item Note that there is only one empty set --- if $\emptyset$ and $\emptyset'$ were supposedly distinct empty sets, then Definition \ref{dfn:setEquality} would prove that $\emptyset=\emptyset'$.
    \end{itemize}
    \item \textbf{Non-empty set}: A set that \dq{is not equal to the empty set}{36}
    \item Non-empty sets must contain at least one object.
    \begin{lem}[Single choice]
        Let $A$ be a non-empty set. Then there exists an object $x$ such that $x\in A$.
        \begin{proof}
            Suppose for the sake of contradiction that no object $x$ exists such that $x\in A$, i.e., for all objects $x$, we have $x\notin A$. By Axiom \ref{axm:emptyset}, we have $x\notin\emptyset$ either. Thus, $x\in A \Longleftrightarrow x\in\emptyset$ (denoting logical equivalence; both statements are equally false), so, by Definition \ref{dfn:setEquality}, $A=\emptyset$, a contradiction.
        \end{proof}
    \end{lem}
    \item There exist more sets than just the empty set.
    \begin{axm}[Singleton sets and pair sets]\label{axm:singletonPair}
        If $a$ is an object, then there exists a set $\{a\}$ whose only element is $a$, i.e., for every object $y$, we have $y\in\{a\}$ if and only if $y=a$; we refer to $\{a\}$ as the \textbf{singleton set} whose element is $a$. Furthermore, if $a$ and $b$ are objects, then there exists a set $\{a,b\}$ whose elements are $a$ and $b$; i.e., for every object $y$, we have $y\in\{a,b\}$ if and only if $y=a$ or $y=b$; we refer to this set as the \textbf{pair set} formed by $a$ and $b$.
    \end{axm}
    \item By Definition \ref{dfn:setEquality}, there exists only one singleton set for each object $a$ and only one pair set for any two objects $a,b$.
    \item Note that the singleton set axiom follows from the pair set axiom, and the pair set axiom follows from the singleton set axiom and the pairwise union axiom, below.
    \item As alluded to, the pairwise union axiom allows us to build sets with more than two elements.
    \begin{axm}[Pairwise union]\label{axm:pairwiseUnion}
        Given any two sets $A,B$, there exists a set $A\cup B$ called the \textbf{union} $A\cup B$ of $A$ and $B$, whose elements consist of all the elements which belong to $A$ or $B$ or both. In other words, for any object $x$,
        \begin{equation*}
            x\in A\cup B \Longleftrightarrow (x\in A\text{ or }x\in B)
        \end{equation*}
    \end{axm}
    \item The $\cup$ operation obeys the axiom of substitution (if $A=A'$, then $A\cup B=A'\cup B$).
    \item We now prove some basic properties of unions (one below and three in Exercise \ref{exr:3.1.3}).
    \begin{lem}\label{lem:associativitySetUnion}
        If $A,B,C$ are sets, then the union operation is associative, i.e., $(A\cup B)\cup C=A\cup(B\cup C)$.
        \begin{proof}
            By Definition \ref{dfn:setEquality}, showing that every element $x$ of $(A\cup B)\cup C$ is an element of $A\cup(B\cup C)$ and vice versa will suffice to prove this lemma. Suppose first that $x\in(A\cup B)\cup C$. By Axiom \ref{axm:pairwiseUnion}, this means that at least one of $x\in A\cup B$ or $x\in C$ is true. We now divide into two cases. If $x\in C$, then by Axiom \ref{axm:pairwiseUnion}, $x\in B\cup C$, and, so, by Axiom \ref{axm:pairwiseUnion} again, we have $x\in A\cup(B\cup C)$. Now suppose instead that $x\in A\cup B$. Then by Axiom \ref{axm:pairwiseUnion}, $x\in A$ or $x\in B$. If $x\in A$, then $x\in A\cup(B\cup C)$ by Axiom \ref{axm:pairwiseUnion}, while if $x\in B$, then by consecutive applications of Axiom \ref{axm:pairwiseUnion}, we have $x\in B\cup C$ and, hence, $x\in A\cup (B\cup C)$. A similar argument shows that every element of $A\cup(B\cup C)$ lies in $(A\cup B)\cup C$, and so $(A\cup B)\cup C=A\cup(B\cup C)$, as desired.
        \end{proof}
    \end{lem}
    \item As a consequence of the above, we are free to write $A\cup B\cup C\cup\cdots$ to denote repeated unions without having to use parentheses.
    \item We can also now define triplet sets ($\{a,b,c\}:=\{a\}\cup\{b\}\cup\{c\}$), quadruplet sets, and so forth.
    \begin{itemize}
        \item However, we cannot yet define a set of $n$ objects or an infinite set.
    \end{itemize}
    \item Note that addition and union are analogous, but importantly \emph{not} identical.
    \item Some sets are "larger" than others; hence, subsets.
    \begin{dfn}[Subsets]\label{dfn:subsets}
        Let $A,B$ be sets. We say that $A$ is a \textbf{subset} of $B$, denoted $A\subseteq B$, iff every element of $A$ is also an element of $B$, i.e., for any object $x$, $x\in A \Longrightarrow x\in B$. We say that $A$ is a \textbf{proper subset} of $B$, denoted $A\subsetneq B$ if $A\subseteq B$ and $A\neq B$.
    \end{dfn}
    \item The $\subseteq$ and $\subsetneq$ operations obey the axiom of substitution (since both $=$ and $\in$, the two component operations of $\subseteq$ and $\subsetneq$, obey it).
    \item Note that $A\subseteq A$ and $\emptyset\subseteq A$ for any set $A$.
    \item Note that less than or equal to and subset are analogous, but not identical, either (see below for one related property and Exercise \ref{exr:3.1.4} for two more).
    \begin{prp}[Sets are partially ordered by set inclusion 1]\label{prp:subsetTransitive}
        Let $A,B,C$ be sets. If $A\subseteq B$ and $B\subseteq C$, then $A\subseteq C$.
        \begin{proof}
            Suppose that $A\subseteq B$ and $B\subseteq C$. To prove that $A\subseteq C$, we have to prove that every element of $A$ is an element of $C$. Let $x\in A$. Then $x\in B$ (since $A\subseteq B$), implying that $x\in C$ (since $B\subseteq C$).
        \end{proof}
    \end{prp}
    \item There exist relations between subsets and unions (see Exercise \ref{exr:3.1.7}).
    \item Note this difference between $\subsetneq$ and $<$: Given any two distinct natural numbers $n,m$, one is smaller than the other (Proposition \ref{prp:trichotomy}). However, given any two distinct sets, it is not in general true that one is a subset of the other. This is why we say that sets are \textbf{partially ordered} while the natural numbers (for example) are \textbf{totally ordered} (see Definitions \ref{dfn:partiallyOrderedSets} and \ref{dfn:totallyOrderedSet}, respectively).
    \item Note that $\in$ and $\subseteq$ are distinct ($2\in\{1,2,3\}$, but $2\nsubseteq\{1,2,3\}$; similarly, $\{2\}\subseteq\{1,2,3\}$, but $\{2\}\notin\{1,2,3\}$).
    \item It is important to distinguish sets from their elements, for they can have different properties ($\N$ is an infinite set of finite elements, and $\{\N,\Z,\Q,\R\}$ is a finite set of infinite objects).
    \item We now formally state that it is acceptable to create subsets.
    \begin{axm}[Axiom of specification]\label{axm:specification}
        Let $A$ be a set, and for each $x\in A$, let $P(x)$ be a property pertaining to $x$ \textup{(}i.e., $P(x)$ is either a true statement or a false statement\textup{)}. Then there exists a set, called $\{x\in A:P(x)\text{ is true}\}$ \textup{(}or simply $\{x\in A:P(x)\}$ for short\textup{)}, whose elements are precisely the elements $x$ in $A$ for which $P(x)$ is true. In other words, for any object $y$,
        \begin{equation*}
            y\in\{x\in A:P(x)\text{ is true}\} \Longleftrightarrow (y\in A\text{ and }P(y)\text{ is true})
        \end{equation*}
    \end{axm}
    \item \emph{Also known as} \textbf{axiom of separation}.
    \item Specification obeys the axiom of substitution (since $\in$ obeys it).
    \item Sometimes $\{x\in A:P(x)\}$ is denoted by $\{x\in A\Big|P(x)\}$ (useful when we need the colon for something else, e.g., $f:X\to Y$).
    \item We use Axiom \ref{axm:specification} to define intersections.
    \begin{dfn}[Intersections]\label{dfn:intersection}
        The \textbf{intersection} $S_1\cap S_2$ of two sets is defined to be the set
        \begin{equation*}
            S_1\cap S_2 := \{x\in S_1:x\in S_2\}
        \end{equation*}
        In other words, $S_1\cap S_2$ consists of all the elements which belong to both $S_1$ and $S_2$. Thus, for all objects $x$,
        \begin{equation*}
            x\in S_1\cap S_2 \Longleftrightarrow x\in S_1\text{ and }x\in S_2
        \end{equation*}
    \end{dfn}
    \item The $\cap$ operation obeys the axiom of substitution (since $\in$ obeys it).
    \begin{itemize}
        \item Note that since $\cap$ is defined in terms of more primitive operations, it is well-defined.
    \end{itemize}
    \item Problems with the English word, "and."
    \begin{itemize}
        \item It can mean union or intersection depending on the context.
        \begin{itemize}
            \item If $X,Y$ are sets, "the set of elements of $X$ and elements of $Y$" refers to $X\cup Y$, e.g., "the set of singles and males."
            \item If $X,Y$ are sets, "the set of objects that are elements of $X$ and elements of $Y$" refers to $X\cap Y$, e.g., "the set of people who are single and male."
        \end{itemize}
        \item It can also denote addition.
        \begin{itemize}
            \item "2 and 3 is 5" means $2+3=5$.
        \end{itemize}
        \item \dq{One reason we resort to mathematical symbols instead of English words such as `and' is that mathematical symbols always have a precise and unambiguous meaning, whereas one must often look very carefully at the context in order to work out what an English word means}{42}
    \end{itemize}
    \item \textbf{Disjoint} (sets): Two sets $A,B$ such that $A\cap B=\emptyset$.
    \item \textbf{Distinct} (sets): Two sets $A,B$ such that $A\neq B$.
    \begin{itemize}
        \item Note that $\emptyset$ and $\emptyset$ are disjoint but not distinct.
    \end{itemize}
    \item We also use Axiom \ref{axm:specification} to define difference sets.
    \begin{dfn}[Difference sets]\label{dfn:differenceSets}
        Given two sets $A$ and $B$, we define the set $A-B$ or $A\setminus B$ to be the set $A$ with any elements of $B$ removed, i.e.,
        \begin{equation*}
            A\setminus B := \{x\in A:x\notin B\}
        \end{equation*}
    \end{dfn}
    \item For example, $\{1,2,3,4\}\setminus\{2,4,6\}=\{1,3\}$ --- in many cases, $B\subseteq A$, but not necessarily.
    \item \marginnote{7/15:}See Exercise \ref{exr:3.1.6} for some basic properties of unions, intersections, and difference sets.
    \begin{itemize}
        \item The de Morgan laws are named after the logician Augustus De Morgan (1806-1871).
        \item The \textbf{laws of Boolean algebra} (the contents of Proposition \ref{prp:booleanAlgebra}) are named after the mathematician George Boole (1815-1864).
        \begin{itemize}
            \item They are applicable to a number of objects other than sets (e.g., laws of propositional logic).
        \end{itemize}
        \item Note the \textbf{duality} in Proposition \ref{prp:booleanAlgebra}, manifesting itself in the certain symmetry between $\cup$ and $\cap$, and $X$ and $\emptyset$.
    \end{itemize}
    \item \textbf{Duality}: \dq{Two distinct properties or objects being dual to each other}{43}
    \item We can do a lot, but we still wish to do more. For example, we'd like to transform sets (say, take $\{3,5,9\}$ and increment each object to yield $\{4,6,10\}$).
    \begin{axm}[Replacement]\label{axm:replacement}
        Let $A$ be a set. For any object $x\in A$, and any object $y$, suppose we have a statement $P(x,y)$ pertaining to $x$ and $y$ such that for each $x\in A$, there is at most one $y$ for which $P(x,y)$ is true. Then there exists a set $\{y:P(x,y)\text{ is true for some }x\in A\}$, such that for any object $z$,
        \begin{equation*}
            z\in\{y:P(x,y)\text{ is true for some }x\in A\} \Longleftrightarrow P(x,z)\text{ is true for some }x\in A
        \end{equation*}
    \end{axm}
    \item For example, let $A:=\{3,5,9\}$ and let $P(x,y)$ be the statement $y=x\pplus$. By Axiom \ref{axm:successorDistinctness}, for every $x\in A$, there is exactly one $y$ for which $P(x,y)$ is true (namely, the successor of $x$). Thus, by Axiom \ref{axm:replacement}, the set $\{y:y=x\pplus\text{ for some }x\in\{3,5,9\}\}$ exists. It is clearly the same set as $\{4,6,10\}$.
    % \begin{itemize}
    %     \item Let $B:=\{y:y=x\pplus\text{ for some }x\in\{3,5,9\}\}$ and $C:=\{4,6,10\}$. Show $B=C$.
    %     \begin{proof}
    %         By Definition \ref{dfn:setEquality}, it will suffice to show that every element of $B$ is an element of $C$ and vice versa. Suppose $z\in\{y:y=x\pplus\text{ for some }x\in\{3,5,9\}\}$. By Axiom \ref{axm:replacement}, $z=x\pplus$ for some $x\in\{3,5,9\}$.
    %     \end{proof}
    % \end{itemize}
    \item The set obtained can be smaller than the original set, e.g., $\{y:y=1\text{ for some }x\in\{3,5,9\}\}=\{1\}$.
    \item We often abbreviate the set specified in Axiom \ref{axm:replacement} to one of the following.
    \begin{gather*}
        \{y:y=f(x)\text{ for some }x\in A\}\\
        \{f(x):x\in A\}\\
        \{f(x)\Big|x\in A\}
    \end{gather*}
    \item We can combine Axioms \ref{axm:replacement} and \ref{axm:specification}, i.e., $\{f(x):x\in A;P(x)\text{ is true}\}$, e.g., $\{n\pplus:n\in\{3,5,9\};n<6\}=\{4,6\}$.
    \item Although we have assumed that natural numbers are objects in several examples up to this point, we must formalize this notion.
    \begin{axm}[Infinity]\label{axm:infinity}
        There exists a set $\N$, whose elements are called the natural numbers, as well as an object $0$ in $\N$, and an object $n\pplus$ assigned to every natural number $n\in N$, such that the Peano axioms hold.
    \end{axm}
    \item Axiom \ref{axm:infinity} is called the \textbf{axiom of infinity} because \dq{it introduces the most basic example of an infinite set, namely the set of natural numbers $\N$}{44}
\end{itemize}


\subsection*{Exercises}
\begin{enumerate}[ref={\thesection.\arabic*}]
    \item \label{exr:3.1.1}\marginnote{7/4:}Show that the definition of equality in Definition \ref{dfn:setEquality} is reflexive, symmetric, and transitive\footnote{Note that since Definition \ref{dfn:setEquality} should be an axiom (should axiomatize equality and all that that entails for sets), this exercise is silly (see \cite{bib:TaoErrata}).}.
    \begin{proof}
        Given a set $A$, suppose $A\neq A$. Then, by Definition \ref{dfn:setEquality}, every element of $A$ is not an element of $A$, a contradiction. Thus, $A=A$.\par
        Let sets $A=B$. Then, by Definition \ref{dfn:setEquality}, every element $x$ of $A$ belongs also to $B$, and every element $y$ of $B$ belongs also to $A$. Identically, every element $y$ of $B$ belongs also to $A$, and every element $X$ of $A$ belongs also to $B$. Thus, $B=A$.\par
        Let sets $A=B$ and $B=C$. Then, by Definition \ref{dfn:setEquality}, every element $x$ of $A$ belongs also to $B$, and every element $y$ of $B$ belongs also to $A$. Similarly, every element $y$ of $B$ belongs also to $C$, and every element $z$ of $C$ belongs also to $B$. Since $x\in A\Rightarrow x\in B\Rightarrow x\in C$, and $y\in C\Rightarrow y\in B\Rightarrow y\in A$, $A=C$.
    \end{proof}
    \item \label{exr:3.1.2}\marginnote{7/14:}Using only Definition \ref{dfn:setEquality}, Axiom \ref{axm:setsAreObjects}, Axiom \ref{axm:emptyset}, and Axiom \ref{axm:singletonPair}, prove that the sets $\emptyset$, $\{\emptyset\}$, $\{\{\emptyset\}\}$, and $\{\emptyset,\{\emptyset\}\}$ are all distinct (i.e., no two of them are equal to each other).
    \begin{proof}
        % Super rigorous: Suppose for the sake of contradiction that $\emptyset=\{\emptyset\}$. By Axiom \ref{axm:setsAreObjects}, $\emptyset$ is an object, implying by Axiom \ref{axm:singletonPair} that there exists a set $\{\emptyset\}$ (which happens to be the set on the right of the hypothesized equality). It follows that $\emptyset\in\{\emptyset\}$. Since $\emptyset=\{\emptyset\}$, by Definition \ref{dfn:setEquality}, $\emptyset\in\emptyset$ as well. But this contradicts Axiom \ref{axm:emptyset}, which asserts that $x\notin\emptyset$ for all objects $x$, including $\emptyset$. Therefore, $\emptyset\neq\{\emptyset\}$.
        First, we show that all sets are distinct from the empty set. Axiom \ref{axm:setsAreObjects} asserts that $\emptyset$ and $\{\emptyset\}$ are objects. By Axiom \ref{axm:singletonPair}, we have $\emptyset\in\{\emptyset\}$, $\emptyset\in\{\emptyset,\{\emptyset\}\}$, and $\{\emptyset\}\in\{\{\emptyset\}\}$. Since $x\notin\emptyset$ for all objects $x$ (Axiom \ref{axm:emptyset}), $\{\emptyset\}$, $\{\{\emptyset\}\}$, and $\{\emptyset,\{\emptyset\}\}$ all contain objects that $\emptyset$ does not (namely, $\emptyset$, $\{\emptyset\}$, and $\emptyset$, respectively). Thus, by Definition \ref{dfn:setEquality}, $\emptyset\neq\{\emptyset\}$, $\emptyset\neq\{\{\emptyset\}\}$, and $\emptyset\neq\{\emptyset,\{\emptyset\}\}$.\par
        Next, we show that $\{\emptyset\}\neq\{\emptyset,\{\emptyset\}\}$. By Axiom \ref{axm:singletonPair}, $\{\emptyset\}\in\{\emptyset,\{\emptyset\}\}$ and $y\in\{\emptyset\}$ iff $y=\emptyset$. Since $\{\emptyset\}\neq\emptyset$ (see above), $\{\emptyset\}\notin\{\emptyset\}$. Thus, $\{\emptyset,\{\emptyset\}\}$ contains an object that $\{\emptyset\}$ does not, implying by Definition \ref{dfn:setEquality} that $\{\emptyset\}\neq\{\emptyset,\{\emptyset\}\}$.\par
        Lastly, we show that $\{\emptyset\}\neq\{\{\emptyset\}\}$ and that $\{\emptyset,\{\emptyset\}\}\neq\{\{\emptyset\}\}$. We proceed in a similar manner to the above. By Axiom \ref{axm:singletonPair}, $\emptyset\in\{\emptyset\}$, $\emptyset\in\{\emptyset,\{\emptyset\}\}$, and $y\in\{\{\emptyset\}\}$ iff $y=\{\emptyset\}$. Since $\emptyset\neq\{\emptyset\}$ (see above), $\emptyset\notin\{\{\emptyset\}\}$. Thus, $\{\emptyset\}$ and $\{\emptyset,\{\emptyset\}\}$ both contain an object that $\{\{\emptyset\}\}$ does not, implying by Definition \ref{dfn:setEquality} that $\{\emptyset\}\neq\{\{\emptyset\}\}$ and that $\{\emptyset,\{\emptyset\}\}\neq\{\{\emptyset\}\}$.\par
        % Second, we show that $\{\emptyset\}\neq\{\{\emptyset\}\}$. By Axiom \ref{axm:singletonPair}, $\emptyset\in\{\emptyset\}$ and $y\in\{\{\emptyset\}\}$ iff $y=\{\emptyset\}$. Since $\emptyset\neq\{\emptyset\}$ (see above), $\emptyset\notin\{\{\emptyset\}\}$. Thus, $\{\emptyset\}$ contains an object that $\{\{\emptyset\}\}$ does not, implying by Definition \ref{dfn:setEquality} that $\{\emptyset\}\neq\{\{\emptyset\}\}$.\par
        % Lastly, we show that $\{\{\emptyset\}\}\neq\{\emptyset,\{\emptyset\}\}$. We proceed in a similar manner to the previous two. By Axiom \ref{axm:singletonPair}, $\emptyset\in\{\emptyset,\{\emptyset\}\}$ and $y\in\{\{\emptyset\}\}$ iff $y=\{\emptyset\}$. Since $\emptyset\neq\{\emptyset\}$ (see above), $\emptyset\notin\{\{\emptyset\}\}$. Thus, $\{\emptyset,\{\emptyset\}\}$ contains an object that $\{\{\emptyset\}\}$ does not, implying by Definition \ref{dfn:setEquality} that $\{\{\emptyset\}\}\neq\{\emptyset,\{\emptyset\}\}$.
        %  and $\{\emptyset\}\neq\{\emptyset,\{\emptyset\}\}$.
    \end{proof}
    \item \label{exr:3.1.3}Prove the following lemmas.
    \begin{lem}\label{lem:commutativityUnion}
        If $a$ and $b$ are objects, then $\{a,b\}=\{a\}\cup\{b\}$.
        \begin{proof}
            % $\{a\}$ and $\{b\}$ are sets. Thus, by Axiom \ref{axm:pairwiseUnion}, there exists a set $\{a\}\cup\{b\}$ whose elements consist of all the elements which belong to $\{a\}$ or $\{b\}$ or both. By Axiom \ref{axm:singletonPair}, $x\in\{a\}$ iff $x=a$ (implying that $a$ is the only element of $\{a\}$) and $y\in\{b\}$ iff $y=b$ (implying that $b$ is the only element of $\{b\}$). Thus, $\{a\}\cup\{b\}$ must contain exclusively $a,b$. By Axiom \ref{axm:singletonPair}, the set that contains just $a,b$ is $\{a,b\}$. Therefore, $\{a,b\}=\{a\}\cup\{b\}$.
            By Definition \ref{dfn:setEquality}, it will suffice to show that every element $x$ of $\{a,b\}$ is an element of $\{a\}\cup\{b\}$ and vice versa. By Axiom \ref{axm:singletonPair}, if $x\in\{a,b\}$, then $x=a$ or $x=b$. By Axiom \ref{axm:pairwiseUnion}, if $x\in\{a\}\cup\{b\}$, then $x\in\{a\}$ or $x\in\{b\}$, implying by Axiom \ref{axm:singletonPair} that $x=a$ or $x=b$. Thus, the elements of $\{a,b\}$ and of $\{a\}\cup\{b\}$ are both $a,b$, so by Definition \ref{dfn:setEquality}, the sets are equal.
        \end{proof}
    \end{lem}
    \begin{lem}
        If $A,B,C$ are sets, then the union operation is commutative, i.e., $A\cup B=B\cup A$.
        \begin{proof}
            By Definition \ref{dfn:setEquality}, it will suffice to show that every element $x$ of $A\cup B$ is an element of $B\cup A$ and vice versa. Let $x\in A\cup B$. By Axiom \ref{axm:pairwiseUnion}, $x\in A$ or $x\in B$. If, on the one hand, $x\in A$, then $x\in B\cup A$ (by Axiom \ref{axm:pairwiseUnion}). If, on the other hand, $x\in B$, then $x\in B\cup A$ (by Axiom \ref{axm:pairwiseUnion}). A similar argument holds if we choose an element $y\in B\cup A$ first.
        \end{proof}
    \end{lem}
    \begin{lem}
        If $A$ is a set, then $A=A\cup\emptyset=\emptyset\cup A=A\cup A$.
        \begin{proof}
            First, we show that $A\cup\emptyset=\emptyset\cup A$. This is a direct consequence of Lemma \ref{lem:commutativityUnion}.\par
            Next, we show that $A=A\cup\emptyset$. By Definition \ref{dfn:setEquality}, it will suffice to show that every element $x$ of $A$ is an element of $A\cup\emptyset$ and vice versa. By Axiom \ref{axm:pairwiseUnion}, every element $x$ of $A$ is an element of $A\cup\emptyset$. Now let $x\in A\cup\emptyset$. Then by Axiom \ref{axm:pairwiseUnion}, $x\in A$ or $x\in\emptyset$. By Axiom \ref{axm:emptyset}, $x\notin\emptyset$, so $x\in A$. Thus, every element of $A\cup\emptyset$ is an element of $A$. We now have by the transitive property that $A=A\cup\emptyset=\emptyset\cup A$.\par
            Lastly, we show that $A=A\cup A$. By Definition \ref{dfn:setEquality}, it will suffice to show that every element $x$ of $A$ is an element of $A\cup A$ and vice versa. By Axiom \ref{axm:pairwiseUnion}, every element $x$ of $A$ is an element of $A\cup A$. Now let $x\in A\cup A$. Then by Axiom \ref{axm:pairwiseUnion}, $x\in A$ or $x\in A$, implying $x\in A$. We have, at last, by the transitive property that $A=A\cup\emptyset=\emptyset\cup A=A\cup A$.
        \end{proof}
    \end{lem}
    \item \label{exr:3.1.4}Prove the following propositions.
    \begin{prp}[Sets are partially ordered by set inclusion 2]
        Let $A,B$ be sets. If $A\subseteq B$ and $B\subseteq A$, then $A=B$.
        \begin{proof}
            Suppose that $A\subseteq B$ and $B\subseteq A$. By Definition \ref{dfn:subsets}, $A\subseteq B$ implies that every element of $A$ is also an element of $B$ and $B\subseteq A$ implies that every element of $B$ is also an element of $A$. Thus, by Definition \ref{dfn:setEquality}, $A=B$.
        \end{proof}
    \end{prp}
    \begin{prp}[Sets are partially ordered by set inclusion 3]
        Let $A,B,C$ be sets. If $A\subsetneq B$ and $B\subsetneq C$, then $A\subsetneq C$.
        \begin{proof}
            Suppose that $A\subsetneq B$ and $B\subsetneq C$. Then, by Definition \ref{dfn:subsets}, $A\subseteq B$ and $B\subseteq C$, implying $A\subseteq C$ by the first claim proved. Since $A\subsetneq B$, $A\neq B$ (implying, by Definition \ref{dfn:setEquality}, that every element of $A$ is not an element of $B$ or every element of $B$ is not an element of $A$) and every element of $A$ is an element of $B$; hence, every element of $B$ is not an element of $A$. Therefore, $B$ \emph{must} contain some element that $A$ does not. Similarly, $B\subsetneq C$ implies that $C$ must contain some element that $B$ does not. Hence, $C$ contains at least two elements that $A$ does not, proving that $A\neq C$, too.
        \end{proof}
    \end{prp}
    \item \label{exr:3.1.5}Let $A,B$ be sets. Show that the three statements $A\subseteq B$, $A\cup B=B$, and $A\cap B=A$ are logically equivalent (any one of them implies the other two).
    \begin{proof}
        First, we show that $A\subseteq B \Longrightarrow (A\cup B=B\text{ and }A\cap B=A)$. Next, we show that $A\cup B=B \Longrightarrow A\subseteq B$ (which, in turn, implies $A\cap B=A$). Lastly, we show that $A\cap B=A \Longrightarrow A\subseteq B$ (which, in turn, implies $A\cup B=B$). Let's begin.\par
        Suppose that $A\subseteq B$. To prove $A\cup B=B$, Definition \ref{dfn:setEquality} tells us that it will suffice to show that every element $x$ of $A\cup B$ is an element of $B$ and vice versa. By Axiom \ref{axm:pairwiseUnion}, $x\in B \Longrightarrow x\in A\cup B$. By Axiom \ref{axm:pairwiseUnion}, $x\in A\cup B \Longrightarrow (x\in A\text{ or }x\in B)$. By Definition \ref{dfn:subsets}, $A\subseteq B$ means that $x\in A \Longrightarrow x\in B$. Thus, $x\in A\cup B \Longrightarrow (x\in A\text{ or }x\in B) \Longrightarrow x\in B$. Therefore, if $A\subseteq B$, then $A\cup B=B$. To prove that $A\cap B=A$, Definition \ref{dfn:setEquality} tells us that it will suffice to show that every element $x$ of $A\cap B$ is an element of $A$ and vice versa. By Definition \ref{dfn:intersection}, $x\in A\cap B \Longrightarrow x\in A$ (and $x\in B$). Since $x\in A \Longrightarrow x\in B$ (see above), $x\in A \Longrightarrow (x\in A\text{ and }x\in B) \Longrightarrow x\in A\cap B$ (Definition \ref{dfn:intersection}). Therefore, if $A\subseteq B$, then $A\cap B=A$.\par
        Suppose that $A\cup B=B$. To prove $A\subseteq B$, Definition \ref{dfn:subsets} tells us that it will suffice to show that every element of $A$ is also an element of $B$. By Axiom \ref{axm:pairwiseUnion}, $x\in A \Longrightarrow x\in A\cup B$. By Definition \ref{dfn:setEquality}, $y\in A\cup B \Longrightarrow y\in B$. Thus, $x\in A \Longrightarrow x\in A\cup B \Longrightarrow x\in B$. Therefore, if $A\cup B=B$, $A\subseteq B$ (and $A\cap B=A$).\par
        Suppose that $A\cap B=A$. To prove that $A\subseteq B$, Definition \ref{dfn:subsets} tells us that it will suffice to show that every element of $A$ is also an element of $B$. By Definition \ref{dfn:setEquality}, $x\in A \Longrightarrow x\in A\cap B$. By Definition \ref{dfn:intersection}, $y\in A\cap B \Longrightarrow y\in B$. Thus, $x\in A \Longrightarrow x\in A\cap B \Longrightarrow x\in B$. Therefore, if $A\cap B=A$, $A\subseteq B$ (and $A\cup B=B$).
    \end{proof}
    \item \label{exr:3.1.6}\marginnote{7/15:}Prove the following proposition. (Hint: one can use some of these claims to prove others. Some of the claims have also appeared previously in Lemma \ref{lem:associativitySetUnion} and Exercise \ref{exr:3.1.3}.)
    \begin{prp}[Sets form a boolean algebra]\label{prp:booleanAlgebra}
        Let $A,B,C$ be sets, and let $X$ be a set containing $A,B,C$ as subsets.
        \begin{enumerate}[label={\textup{(}\alph*\textup{)}},ref={\theenumi\alph*}]
            \item \label{exr:3.1.6a}(Minimal element) We have $A\cup\emptyset=A$ and $A\cap\emptyset=\emptyset$.
            \begin{proof}
                See Exercise \ref{exr:3.1.3} for the first claim.\par
                To prove $A\cap\emptyset=\emptyset$, Definition \ref{dfn:setEquality} tells us that it will suffice to show that every element $x$ of $A\cap\emptyset$ is an element of $\emptyset$ and vice versa.
                % By Axiom \ref{axm:emptyset}, $x\notin\emptyset$ for all objects $x$. By Definition \ref{dfn:intersection}, $A\cap\emptyset=\{x\in A:x\in\emptyset\}$. But $x\in\emptyset$ is necessarily false for all $x$ (see above). Thus, $x\notin A\cap\emptyset$ for all objects $x$ (specifically, all $x\in A$). Therefore, the statement "every element $x$ of $A\cap\emptyset$ is an element of $\emptyset$ and vice versa" is vacuously true.
                First off, "every element $x\in\emptyset$ is an element of $A\cap\emptyset$" is vacuously true (by Axiom \ref{axm:emptyset}, there exists no $x\in\emptyset$). In the other direction, suppose for the sake of contradiction that $x\in A\cap\emptyset$ for some object $x$. By Definition \ref{dfn:intersection}, $x\in A$ and $x\in \emptyset$. But $x\in\emptyset$ contradicts Axiom \ref{axm:emptyset}. Therefore, $x\notin A\cap\emptyset$ for all objects $x$. Thus, the statement "every element $x\in A\cap\emptyset$ is an element of $\emptyset$" is vacuously true (there exists no $x\in A\cap\emptyset$).
            \end{proof}
            \item \label{exr:3.1.6b}(Maximal element) We have $A\cup X=X$ and $A\cap X=A$.
            \begin{proof}
                See Exercise \ref{exr:3.1.5}.
            \end{proof}
            \item \label{exr:3.1.6c}(Identity) We have $A\cap A=A$ and $A\cup A=A$.
            \begin{proof}
                To prove that $A\cap A=A$, Definition \ref{dfn:setEquality} tells us that it will suffice to show that every element $x$ of $A\cap A$ is an element of $A$ and vice versa. By Definition \ref{dfn:intersection}, $x\in A\cap A \Longrightarrow (x\in A\text{ and }x\in A) \Longrightarrow x\in A$. On the other hand, $x\in A \Longrightarrow (x\in A\text{ and }x\in A)$ (idempotent law for conjunction), $(x\in A\text{ and }x\in A) \Longrightarrow x\in\{x\in A:x\in A\}$ (Axiom \ref{axm:specification}), and $x\in\{x\in A:x\in A\} \Longrightarrow x\in A\cap A$ (Definition \ref{dfn:intersection}).\par
                See Exercise \ref{exr:3.1.3} for the second claim.
            \end{proof}
            \item \label{exr:3.1.6d}(Commutativity) We have $A\cup B=B\cup A$ and $A\cap B=B\cap A$.
            \begin{proof}
                See Exercise \ref{exr:3.1.3} for the first claim.\par
                To prove $A\cap B=B\cap A$, Definition \ref{dfn:setEquality} tells us that it will suffice to show that every element $x$ of $A\cap B$ is an element of $B\cap A$ and vice versa. By two applications of Definition \ref{dfn:intersection} with the commutative law for conjunction in between, $x\in A\cap B \Longrightarrow (x\in A\text{ and }x\in B) \Longrightarrow (x\in B\text{ and }x\in A) \Longrightarrow x\in B\cap A$. A similar argument works in the opposite direction.
            \end{proof}
            \item \label{exr:3.1.6e}(Associativity) We have $(A\cup B)\cup C=A\cup(B\cup C)$ and $(A\cap B)\cap C=A\cap(B\cap C)$.
            \begin{proof}
                See Lemma \ref{lem:associativitySetUnion} for the first claim.\par
                By Definition \ref{dfn:setEquality}, showing that every element $x$ of $(A\cap B)\cap C$ is an element of $A\cap(B\cap C)$ and vice versa will suffice to prove this lemma. Suppose first that $x\in (A\cap B)\cap C$. By Definition \ref{dfn:intersection}, $x\in A\cap B$ and $x\in C$, which implies by a second application of Definition \ref{dfn:intersection} that $x\in A$ and $x\in B$ and $x\in C$. It follows by consecutive applications of Definition \ref{dfn:intersection} that $x\in A$ and $x\in B\cap C$, and that $x\in A\cap(B\cap C)$. A similar argument shows that every element of $A\cap(B\cap C)$ lies in $(A\cap B)\cap C$.
            \end{proof}
            \item \label{exr:3.1.6f}(Distributivity) We have $A\cap(B\cup C)=(A\cap B)\cup(A\cap C)$ and $A\cup(B\cap C)=(A\cup B)\cap(A\cup C)$.
            \begin{proof}
                To prove $A\cap(B\cup C)=(A\cap B)\cup(A\cap C)$, Definition \ref{dfn:setEquality} tells us that it will suffice to show that every element $x$ of $A\cap(B\cup C)$ is an element of $(A\cap B)\cup(A\cap C)$ and vice versa. By Definition \ref{dfn:intersection}, $x\in A\cap(B\cup C) \Longrightarrow (x\in A\text{ and }x\in B\cup C)$. By Axiom \ref{axm:pairwiseUnion}, $x\in B\cup C \Longrightarrow x\in B\text{ or }x\in C$. Thus, we know that $x\in A$, and we know that $x\in B$ or $x\in C$ (or both). We divide into two cases. Suppose first that $x\in B$. Since $x\in A$ as well, $x\in A\cap B$ (Definition \ref{dfn:intersection}). This implies by Axiom \ref{axm:pairwiseUnion} that $x\in(A\cap B)\cup(A\cap C)$. Now suppose that $x\in C$. Since $x\in A$ as well, $x\in A\cap C$ (Definition \ref{dfn:intersection}). This implies by Axiom \ref{axm:pairwiseUnion} that $x\in(A\cap B)\cup(A\cap C)$. A similar argument shows that every element of $(A\cap B)\cup(A\cap C)$ lies in $A\cap(B\cup C)$.\par
                The second proof is similar to the first (and similar to all the rest of the proofs written for this exercise thus far). Thus, for the sake of variety, we will do this one entirely symbolically, using the laws of propositional logic from Section \ref{sss:A.4}.
                \begin{align*}
                    x\in A\cup(B\cap C) &\Longrightarrow (x\in A) \vee (x\in B\cap C)\tag*{Axiom \ref{axm:pairwiseUnion}}\\
                    &\Longrightarrow (x\in A) \vee ((x\in B) \wedge (x\in C))\tag*{Definition \ref{dfn:intersection}}\\
                    &\Longrightarrow ((x\in A) \wedge (x\in B)) \vee ((x\in A) \wedge (x\in C))\tag*{Distributive law for disjunction}\\
                    &\Longrightarrow (x\in A\cap B) \vee (x\in A\cap C)\tag*{Definition \ref{dfn:intersection}}\\
                    &\Longrightarrow x\in (A\cap B)\cup(A\cap C)\tag*{Axiom \ref{axm:pairwiseUnion}}
                \end{align*}
                A similar argument works in reverse.
            \end{proof}
            \item \label{exr:3.1.6g}(Partition) We have $A\cup(X\setminus A)=X$ and $A\cap(X\setminus A)=\emptyset$.
            \begin{proof}
                To prove $A\cup(X\setminus A)=X$, Definition \ref{dfn:setEquality} tells us that it will suffice to show that every element $x$ of $A\cup(X\setminus A)$ is an element of $X$ and vice versa. Suppose $x\in A\cup(X\setminus A)$. Then by Axiom \ref{axm:pairwiseUnion}, $x\in A$ or $x\in X\setminus A$. We divide into two cases. Suppose first that $x\in A$. Since $A\subseteq X$, Definition \ref{dfn:subsets} asserts that $x\in X$. Now suppose that $x\in X\setminus A$. This implies by Definition \ref{dfn:differenceSets} that $x\in\{y\in X:y\notin A\}$, which means by Axiom \ref{axm:specification} that $x\in X$ (and $x\notin A$, but that's not relevant). On the other hand, suppose $x\in X$. Naturally, either $x\in A$ or $x\notin A$ ($x\in A$ is false). If $x\in A$, then $x\in A\cup(X\setminus A)$ (Axiom \ref{axm:pairwiseUnion}). If $x\notin A$, since $x\in X$ as well, Axiom \ref{axm:specification} asserts that $x\in\{x\in X:x\notin A\}$, which, by Definition \ref{dfn:differenceSets}, implies that $x\in X\setminus A$. This, in turn, implies that $x\in A\cup(X\setminus A)$ (Axiom \ref{axm:pairwiseUnion}).\par
                To prove $A\cap(X\setminus A)=\emptyset$, it suffices to prove that for every object $x$, we have $x\notin A\cap(X\setminus A)$ (because of the uniqueness of the empty set). Suppose for the sake of contradiction that $x\in A\cap(X\setminus A)$ for some object $x$. By Definition \ref{dfn:intersection}, $x\in A$ and $x\in X\setminus A$. By Definition \ref{dfn:differenceSets}, $x\in\{y\in X:y\notin A\}$. By Axiom \ref{axm:specification}, $x\in X$ and $x\notin A$. But this contradicts the previously derived fact that $x\in A$. Therefore, $x\notin A\cap(X\setminus A)$ for all objects $x$.
                % To prove $A\cap(X\setminus A)=\emptyset$, Definition \ref{dfn:setEquality} tells us that it will suffice to show that every element $x$ of $A\cap(X\setminus A)$ is an element of $\emptyset$ and vice versa. First off, "every element $x\in\emptyset$ is an element of $A\cap(X\setminus A)$" is vacuously true (by Axiom \ref{axm:emptyset}, there exists no $x\in\emptyset$). In the other direction, suppose for the sake of contradiction that $x\in A\cap(X\setminus A)$ for some object $x$. By Definition \ref{dfn:intersection}, $x\in A$ and $x\in X\setminus A$. By Definition \ref{dfn:differenceSets}, $x\in\{y\in X:y\notin A\}$. By Axiom \ref{axm:specification}, $x\in X$ and $x\notin A$. But this contradicts the previously derived fact that $x\in A$. Therefore, $x\notin A\cap(X\setminus A)$ for all objects $x$. Thus, the statement "every element $x\in A\cap(X\setminus A)$ is an element of $\emptyset$" is vacuously true (there exists no $x\in A\cap(X\setminus A)$).
            \end{proof}
            \item \label{exr:3.1.6h}(De Morgan laws) We have $X\setminus(A\cup B)=(X\setminus A)\cap(X\setminus B)$ and $X\setminus(A\cap B)=(X\setminus A)\cup(X\setminus B)$.
            \begin{proof}
                To prove $X\setminus(A\cup B)=(X\setminus A)\cap(X\setminus B)$, Definition \ref{dfn:setEquality} tells us that it will suffice to show that every element $x$ of $X\setminus(A\cup B)$ is an element of $(X\setminus A)\cap(X\setminus B)$ and vice versa. Suppose that $x\in X\setminus(A\cup B)$. Then by Definition \ref{dfn:differenceSets}, $x\in\{y\in X:y\notin A\cup B\}$. By Axiom \ref{axm:specification}, $x\in X$ and $x\notin A\cup B$. By the inverse of Axiom \ref{axm:pairwiseUnion} (which is a valid assertion since Axiom \ref{axm:pairwiseUnion} asserts the logical equivalence of "$x\in A\cup B$" and "$x\in A$ and $x\in B$"), $x\notin A\cup B \Longrightarrow (x\notin A\text{ and }x\notin B)$. By Axiom \ref{axm:specification} and Definition \ref{dfn:differenceSets}, $(x\in X\text{ and }x\notin A) \Longrightarrow x\in\{y\in X:y\notin A\} \Longrightarrow x\in X\setminus A$. Similarly, $(x\in X\text{ and }x\notin B) \Longrightarrow x\in\{y\in X:y\notin B\} \Longrightarrow x\in X\setminus B$. Thus, by Definition \ref{dfn:intersection}, $x\in(X\setminus A)\cap(X\setminus B)$. A similar, reversed argument will work in the other direction.\par
                To prove $X\setminus(A\cap B)=(X\setminus A)\cup(X\setminus B)$, Definition \ref{dfn:setEquality} tells us that it will suffice to show that every element $x$ of $X\setminus(A\cap B)$ is an element of $(X\setminus A)\cup(X\setminus B)$ and vice versa. By Definition \ref{dfn:differenceSets} and Axiom \ref{axm:specification}, $x\in X\setminus(A\cap B) \Longrightarrow x\in\{y\in X:y\notin A\cap B\} \Longrightarrow (x\in X\text{ and }x\notin A\cap B)$. By the inverse of Definition \ref{dfn:intersection} (which, again, is a valid assertion since $x\in A\cap B \leftrightarrow ((x\in A) \wedge (x\in B))$), $x\notin A\cap B \Longrightarrow (x\notin A\text{ or }x\notin B)$. We divide into two cases. Suppose $x\notin A$. Then by Axiom \ref{axm:specification} and Definition \ref{dfn:differenceSets}, $(x\in X\text{ and }x\notin A) \Longrightarrow x\in\{y\in X:y\notin A\} \Longrightarrow x\in X\setminus A$. Thus, by Axiom \ref{axm:pairwiseUnion}, $x\in(X\setminus A)\cup(X\setminus B)$. Now suppose $x\notin B$. Then by Axiom \ref{axm:specification} and Definition \ref{dfn:differenceSets}, $(x\in X\text{ and }x\notin B) \Longrightarrow x\in\{y\in X:y\notin B\} \Longrightarrow x\in X\setminus B$. Thus, by Axiom \ref{axm:pairwiseUnion}, $x\in(X\setminus A)\cup(X\setminus B)$.
            \end{proof}
        \end{enumerate}
    \end{prp}
    \item \label{exr:3.1.7}\marginnote{7/17:}Let $A,B,C$ be sets. Show that $A\cap B\subseteq A$ and $A\cap B\subseteq B$. Furthermore, show that $C\subseteq A$ and $C\subseteq B$ iff $C\subseteq A\cap B$. In a similar spirit, show that $A\subseteq A\cup B$ and $B\subseteq A\cup B$, and furthermore that $A\subseteq C$ and $B\subseteq C$ iff $A\cup B\subseteq C$.
    \begin{proof}
        To prove that $A\cap B\subseteq A$ and $A\cap B\subseteq B$, Definition \ref{dfn:subsets} tells us that it will suffice to show that every element of $A\cap B$ is an element of $A$ and $B$. By Definition \ref{dfn:intersection}, $x\in A\cap B \Longrightarrow (x\in A\text{ and }x\in B)$.\par
        \marginnote{7/19:}To prove that $C\subseteq A$ and $C\subseteq B$ iff $C\subseteq A\cap B$, it will suffice to show that $C\subseteq A$ and $C\subseteq B$ imply $C\subseteq A\cap B$ and vice versa. Suppose first that $C\subseteq A$ and $C\subseteq B$. Then by two applications of Definition \ref{dfn:subsets}, $x\in C \Longrightarrow x\in A$ and $x\in C \Longrightarrow x\in B$. By Definition \ref{dfn:intersection}, $(x\in A\text{ and }x\in B) \Longrightarrow x\in A\cap B$. Thus, every element $x$ of $C$ is an element of $A\cap B$, so by Definition \ref{dfn:subsets}, $C\subseteq A\cap B$. Now suppose that $C\subseteq A\cap B$. Then by Definition \ref{dfn:subsets}, $x\in C \Longrightarrow x\in A\cap B$. By Definition \ref{dfn:intersection}, $x\in A\cap B \Longrightarrow (x\in A\text{ and }x\in B)$. Thus, every element $x$ of $C$ is an element of $A$ and $B$, so by two applications of Definition \ref{dfn:subsets}, $C\subseteq A$ and $C\subseteq B$.\par
        To prove that $A\subseteq A\cup B$ and that $B\subseteq A\cup B$, Definition \ref{dfn:subsets} tells us that it will suffice to show that every element $x$ of $A$ is an element of $A\cup B$ and that every element $y$ of $B$ is an element of $A\cup B$, respectively. By two applications of Axiom \ref{axm:pairwiseUnion}, $x\in A \Longrightarrow x\in A\cup B$, and $y\in B \Longrightarrow y\in A\cup B$.\par
        To prove that $A\subseteq C$ and $B\subseteq C$ iff $A\cup B\subseteq C$, it will suffice to show that $A\subseteq C$ and $B\subseteq C$ imply $A\cup B\subseteq C$ and vice versa. Suppose first that $A\subseteq C$ and $B\subseteq C$. By Axiom \ref{axm:pairwiseUnion}, $x\in A\cup B \Longrightarrow (x\in A\text{ or }x\in B)$. We divide into two cases. If $x\in A$, then Definition \ref{dfn:subsets} ensures that $x\in C$. If $x\in B$, then Definition \ref{dfn:subsets} similarly ensures that $x\in C$. Thus, either way, $x\in A\cup B \Longrightarrow x\in C$, so by Definition \ref{dfn:subsets}, $A\cup B\subseteq C$. Now suppose that $A\cup B\subseteq C$. By Axiom \ref{axm:pairwiseUnion} followed by Definition \ref{dfn:subsets}, $x\in A \Longrightarrow x\in A\cup B \Longrightarrow x\in C$. A similar argument shows that $x\in B \Longrightarrow x\in C$. Thus, $A\subseteq C$ and $B\subseteq C$.
    \end{proof}
    \item \label{exr:3.1.8}Let $A,B$ be sets. Prove the \textbf{absorption laws} $A\cap(A\cup B)=A$ and $A\cup(A\cap B)=A$.
    \begin{proof}
        By Exercise \ref{exr:3.1.7}, $A\subseteq A\cup B$. By Exercise \ref{exr:3.1.6b} (with $A=A$ and $X=A\cup B$), $A\cap(A\cup B)=A$.\par
        By Exercise \ref{exr:3.1.7}, $A\cap B\subseteq A$. By Exercise \ref{exr:3.1.6b} (with $A=A\cap B$ and $X=A$), $A\cup(A\cap B)=A$.
    \end{proof}
    \item \label{exr:3.1.9}Let $A,B,X$ be sets such that $A\cup B=X$ and $A\cap B=\emptyset$. Show that $A=X\setminus B$ and $B=X\setminus A$.
    \begin{proof}
        To prove $A=X\setminus B$, Definition \ref{dfn:setEquality} tells us that it will suffice to show that every element $x$ of $A$ is an element of $X\setminus B$ and vice versa. Suppose first that $x\in A$. Then by Axiom \ref{axm:pairwiseUnion}, $x\in A\cup B$. Since $A\cup B=X$, we have by Definition \ref{dfn:setEquality} that $x\in X$. Now suppose for the sake of contradiction that $x\in B$. Since $x\in A$ as well, by Definition \ref{dfn:intersection}, $x\in A\cap B$. But $A\cap B=\emptyset$, so by Definition \ref{dfn:setEquality}, $x\in\emptyset$, which contradicts Axiom \ref{axm:emptyset}. Therefore, $x\notin B$. Since $x\in X$ and $x\notin B$, by Definition \ref{dfn:differenceSets}, $x\in X\setminus B$. Now suppose that $x\in X\setminus B$. Then by Definition \ref{dfn:differenceSets}, $x\in X$ and $x\notin B$. By Definition \ref{dfn:setEquality}, $x\in A\cup B$. Thus, by Axiom \ref{axm:pairwiseUnion}, $x\in A$ or $x\in B$. Since $x\notin B$, $x$ must be an element of $A$. A similar argument shows that $B=X\setminus A$.
    \end{proof}
    \item \label{exr:3.1.10}\marginnote{7/22:}Let $A$ and $B$ be sets. Show that the three sets $A\setminus B$, $A\cap B$, and $B\setminus A$ are disjoint, and that their union is $A\cup B$.
    \begin{proof}
        To show that $A\setminus B$, $A\cap B$, and $B\setminus A$ are disjoint, we must show that
        \begin{equation*}
            (A\setminus B)\cap(A\cap B) = (A\setminus B)\cap(B\setminus A)
            = (A\cap B)\cap(B\setminus A)
            = \emptyset
        \end{equation*}
        We may do this by showing that no object $x$ is an element of any of the left three sets above (because of the uniqueness of the empty set). We do this via three contradiction proofs, as follows.\par
        Suppose for the sake of contradiction that $x\in(A\setminus B)\cap(A\cap B)$. Then by Definition \ref{dfn:intersection}, $x\in A\setminus B$ and $x\in A\cap B$. Since $x\in A\setminus B$, Definition \ref{dfn:differenceSets} tells us that $x\notin B$ (and $x\in A$). But since $x\in A\cap B$, Definition \ref{dfn:intersection} tells us that $x\in B$ (and $x\in A$), a contradiction.\par
        A similar argument to the above can handle $(A\cap B)\cap(B\setminus A)$.\par
        Suppose for the sake of contradiction that $x\in (A\setminus B)\cap(B\setminus A)$. Then by Definition \ref{dfn:intersection}, $x\in A\setminus B$ and $x\in B\setminus A$. By the first statement, $x\in A$ and $x\notin B$, while by the second statement, $x\in B$ and $x\notin A$, two contradictions.\par
        We now turn our attention to proving the following.
        \begin{equation*}
            A\cup B = (A\setminus B)\cup(A\cap B)\cup(B\setminus A)
        \end{equation*}
        We can actually prove this solely on the basis of prior results (and one additional lemma).
        \begin{lem}\label{lem:unionMinus}
            Let $A$ and $B$ be sets. Show that $(A\cup B)\setminus A=B\setminus A$.
            \begin{proof}
                To prove $(A\cup B)\setminus A=B\setminus A$, Definition \ref{dfn:setEquality} tells us that it will suffice to show that every element $x$ of $(A\cup B)\setminus A$ is an element of $B\setminus A$ and vice versa. Suppose first that $x\in(A\cup B)\setminus A$. Then by Definition \ref{dfn:differenceSets}, $x\in A\cup B$ and $x\notin A$. Since $x\in A\cup B$, by Axiom \ref{axm:pairwiseUnion}, $x\in A$ or $x\in B$. But $x\notin A$, so $x$ must be an element of $B$. Having established that $x\in B$ and $x\notin A$, Definition \ref{dfn:differenceSets} tells us that $x\in B\setminus A$. Now suppose that $x\in B\setminus A$. Then by Definition \ref{dfn:differenceSets}, $x\in B$ and $x\notin A$. Since $x\in B$, by Axiom \ref{axm:pairwiseUnion}, $x\in A\cup B$. Consequently, by Definition \ref{dfn:differenceSets}, $x\in(A\cup B)\setminus A$.
            \end{proof}
        \end{lem}
        Now we can begin. By Exercise \ref{exr:3.1.7}, $A\cap B\subseteq A$ and $A\subseteq A\cup B$. This implies by Proposition \ref{prp:subsetTransitive} that $A\cap B\subseteq A\cup B$. Thus,
        \begin{align*}
            A\cup B &= (A\cap B)\cup((A\cup B)\setminus(A\cap B))\tag*{Exercise \ref{exr:3.1.6g}}\\
            &= (A\cap B)\cup((A\cup B)\setminus A)\cup((A\cup B)\setminus B)\tag*{Exercise \ref{exr:3.1.6h}}\\
            &= (A\cap B)\cup(B\setminus A)\cup(A\setminus B)\tag*{Lemma \ref{lem:unionMinus}}
        \end{align*}
    \end{proof}
    \item \label{exr:3.1.11}Show that the axiom of replacement implies the axiom of specification.
    \begin{proof}
        Let $P(x,y)$ be the statement "$y=x$ and $P(y)$ is true." Then by Axiom \ref{axm:replacement}, there exists a set
        \begin{align*}
            \{y:P(x,y)\text{ is true for some }x\in A\} &= \{y:y=x\text{ and }P(y)\text{ is true for some }x\in A\}\\
            &= \{y:y\in A\text{ and }P(y)\text{ is true}\}\\
            &= \{y\in A:P(y)\text{ is true}\}
        \end{align*}
        Moreover,
        \begin{align*}
            z\in \{y\in A:P(y)\text{ is true}\} &\Longrightarrow z\in\{y:P(x,y)\text{ is true for some }x\in A\}\\
            &\Longrightarrow P(x,z)\text{ is true for some }x\in A\\
            &\Longrightarrow z=x\text{ and }P(z)\text{ is true for some }x\in A\\
            &\Longrightarrow z\in A\text{ and }P(z)\text{ is true}
        \end{align*}
        The above logic also works in reverse. Thus, all the tenets of Axiom \ref{axm:specification} have been shown to follow from Axiom \ref{axm:replacement} (proof modified from \cite{bib:ReplacementToSpecification}).
    \end{proof}
\end{enumerate}



\section{Russell's Paradox}
\begin{itemize}
    \item Suppose that we could unify the multitude of axioms in Section \ref{sse:3.1} into a single axiom. The following would be a good candidate (in fact, it implies the majority of the Section \ref{sse:3.1} axioms --- see Exercise \ref{exr:3.2.1}).
    \begin{axm}[Universal specification]\label{axm:universalSpecification}
        (Dangerous!) Suppose for every object $x$ we have a property $P(x)$ pertaining to $x$ (so that for every $x$, $P(x)$ is either a true statement or a false statement). Then there exists a set $\{x:P(x)\text{ is true}\}$ such that for every object $y$,
        \begin{equation*}
            y\in\{x:P(x)\text{ is true}\} \Longleftrightarrow P(y)\text{ is true}
        \end{equation*}
    \end{axm}
    \item \emph{Also known as} \textbf{axiom of comprehension}.
    \item Basically, Axiom \ref{axm:universalSpecification} asserts that \dq{every property corresponds to a set}{46}
    \item Unfortunately, Axiom \ref{axm:universalSpecification} cannot be introduced into set theory because it creates a logical contradiction known as \textbf{Russell's paradox}.
    \begin{itemize}
        \item Discovered by philosopher and logician Bertrand Russell (1872-1970) in 1901.
    \end{itemize}
    \item \textbf{Russell's paradox}: \dq{Let $P(x)$ be the statement\dots "$x$ is a set, and $x\notin x$"; i.e., $P(x)$ is true only when $x$ is a set which does not contain itself. For instance, $P(\{2,3,4\})$ is true, since the set $\{2,3,4\}$ is not one of the three elements 2, 3, 4 of $\{2,3,4\}$. On the other hand, if we let $S$ be the set of all sets (which we would know to exist from the axiom of universal specification), then since $S$ is itself a set, it is an element of $S$, and so $P(S)$ is false. Now use the axiom of universal specification to create the set
    \begin{equation*}
        \Omega := \{x:P(x)\text{ is true}\} = \{x:x\text{ is a set and }x\notin x\}
    \end{equation*}
    i.e., the set of all sets which do not contain themselves. Now ask the question: does $\Omega$ contain itself, i.e. is $\Omega\in\Omega$? If $\Omega$ did contain itself, then by definition this means that $P(\Omega)$ is true, i.e., $\Omega$ is a set and $\Omega\notin\Omega$. On the other hand, if $\Omega$ did not contain itself, then $P(\Omega)$ would be true, and hence $\Omega\in\Omega$. Thus in either case we have both $\Omega\in\Omega$ and $\Omega\notin\Omega$, which is absurd}{46-47}
    \begin{itemize}
        \item To clarify the last point: Is $\Omega\in\Omega$? Suppose $\Omega\in\Omega$. Then since $\Omega$ contains only sets for which $P(x)$ is true, $P(\Omega)$ must be true. But this implies, by the definition of $P(x)$, that $\Omega\notin\Omega$. Similarly, suppose $\Omega\notin\Omega$. Then since "$\Omega$ is a set and $\Omega\notin\Omega$" is a true statement, $P(\Omega)$ must be true. But this implies, since $\Omega$ contains all sets for which $P(x)$ is true, that $\Omega\in\Omega$. In either case, we have both $\Omega\in\Omega$ and $\Omega\notin\Omega$ (contradictions).
    \end{itemize}
    \item \marginnote{7/23:}The main problem highlighted by Russell's paradox is that Axiom \ref{axm:universalSpecification} allows for the creation of sets that are too "large," i.e., sets that contain themselves, which is somewhat silly.
    \begin{itemize}
        \item This problem can be informally resolved by creating a hierarchy: primitive objects are below primitive sets (which only contain primitive objects), are below second-level sets (which only contain primitive objects and primitive sets), and so on and so forth. Formalizing this notion is complicated and will not be explored further here.
        \item Note that in pure set theory, there are no primitive objects --- only one primitive set (the empty set).
    \end{itemize}
    \item To avoid the complications of Russell's paradox, we create a new axiom.
    \begin{axm}[Regularity]\label{axm:regularity}
        If $A$ is a non-empty set, then there is at least one element $x$ of $A$ which is either not a set or is disjoint from $A$.
    \end{axm}
    \item \emph{Also known as} \textbf{axiom of foundation}.
    \item Axiom \ref{axm:regularity} asserts that \dq{at least one of the elements of $A$ is so low on the hierarchy of objects that it does not contain any of the other elements of $A$}{48} It also asserts that sets may not contain themselves (see Exercise \ref{exr:3.2.2}).
    \item As a less intuitive axiom, one might question whether or not Axiom \ref{axm:regularity} is needed. In fact, it is not necessary for the purposes of doing analysis, as all sets considered in analysis are very low on the hierarchy. However, it is necessary to perform more advanced set theory, so \cite{bib:AnalysisI} included it for the sake of completeness.
\end{itemize}


\subsection*{Exercises}
\begin{enumerate}[ref={\thesection.\arabic*}]
    \item \label{exr:3.2.1}\marginnote{7/22:}Show that the universal specification axiom, Axiom \ref{axm:universalSpecification}, if assumed to be true, would imply Axioms \ref{axm:emptyset}, \ref{axm:singletonPair}, \ref{axm:pairwiseUnion}, \ref{axm:specification}, and \ref{axm:replacement}. (If we assume that all natural numbers are objects, we also obtain Axiom \ref{axm:infinity}.) Thus, this axiom, if permitted, would simplify the foundations of set theory tremendously (and can be viewed as one basis for an intuitive model of set theory known as "naive set theory"). Unfortunately, as we have seen, Axiom \ref{axm:universalSpecification} is "too good to be true!"
    \begin{proof}
        Axiom \ref{axm:emptyset}: Let $P(x)$ be a false statement for all objects $x$. By Axiom \ref{axm:universalSpecification}, there exists a set $\{x:P(x)\text{ is true}\}$, which contains no elements. (Suppose for the sake of contradiction that $y\in\{x:P(x)\text{ is true}\}$ for some object $y$. Then $P(y)$ is true. But $P(y)$ is false by definition, a contradiction. Therefore, $y\notin\{x:P(x)\text{ is true}\}$ for all objects $y$.) Incidentally, that contradiction proof solidifies the symbolic statement of Axiom \ref{axm:emptyset}. Lastly, this set may be denoted $\emptyset$ or $\{\}$.\par
        Axiom \ref{axm:singletonPair}: Let $P(x)$ be the statement $x=a$. By Axiom \ref{axm:universalSpecification}, there exists a set $\{x:P(x)\text{ is true}\}=\{x:x=a\}=\{a\}$ whose element is $a$. For every object $y$, we have $y\in\{a\}$ iff $P(y)$ is true, i.e., iff $y=a$. This set may be called the \textbf{singleton set} whose element is $a$. Now let $P(x)$ be the statement "$x=a$ or $x=b$." By Axiom \ref{axm:universalSpecification}, there exists a set $\{x:P(x)\text{ is true}\}=\{x:x=a\text{ or }x=b\}=\{a,b\}$ whose elements are $a$ and $b$. For every object $y$, we have $y\in\{a,b\}$ iff $P(y)$ is true, i.e., iff $y=a$ or $y=b$. This set may be called the \textbf{pair set} formed by $a$ and $b$.\par
        Axiom \ref{axm:pairwiseUnion}: Let $A,B$ be sets. Let $P(x)$ be the statement "$x\in A$ or $x\in B$." By Axiom \ref{axm:universalSpecification}, there exists a set $\{x:P(x)\text{ is true}\}=\{x:x\in A\text{ or }x\in B\}$. This set may be called the \textbf{union} $A\cup B$ of $A$ and $B$. By Axiom \ref{axm:universalSpecification}, $y\in A\cup B$ iff $P(y)$ is true, i.e., iff $y\in A$ or $y\in B$. Thus, $A\cup B$ is clearly a set whose elements consist of all the elements which belong to $A$ or $B$ or both.\par
        Axiom \ref{axm:replacement}: Let $A$ be a set. Let $P(y)$ be the statement "$P(x,y)$ is true for some $x\in A$," where $P(x,y)$ is a statement pertaining to $x$ and $y$ such that for each $x\in A$, there is at most one $y$ for which $P(x,y)$ is true. By Axiom \ref{axm:universalSpecification}, there exists a set $\{y:P(y)\text{ is true}\}=\{y:P(x,y)\text{ is true for some }x\in A\}$. By Axiom \ref{axm:universalSpecification}, $z\in\{y:P(x,y)\text{ is true for some }x\in A\}$ iff $P(z)$ is true, i.e., iff $P(x,z)$ is true for some $x\in A$.\par
        Axiom \ref{axm:specification}: Implied by the axiom of replacement (see Exercise \ref{exr:3.1.11}).
    \end{proof}
    \item \label{exr:3.2.2}\marginnote{7/23:}Use the axiom of regularity (and the singleton set axiom) to show that if $A$ is a set, then $A\notin A$. Furthermore, show that if $A$ and $B$ are two sets, then either $A\notin B$ or $B\notin A$ (or both).
    \begin{proof}
        % Let $A$ be a set. Then by Axiom \ref{axm:setsAreObjects}, $A$ is an object. Thus, by Axiom \ref{axm:singletonPair}, there exists a set $\{A\}$ whose only element is $A$. Now by Axiom \ref{axm:regularity}, there exists an element of $\{A\}$ (which must be $A$, as referenced above) that is either not a set or is disjoint from $\{A\}$. Since $A$ is a set, we have that $A$ is disjoint from $\{A\}$, i.e., $A\cap\{A\}=\emptyset$. By Axiom \ref{axm:emptyset}, $x\notin A\cap\{A\}$ for all objects $x$. This implies by Definition \ref{dfn:intersection} that $x\notin A$ or $x\notin\{A\}$. But $x\notin\{A\}$ implies that $x\neq A$, so we have either $x\neq A$ or $x\notin A$. Therefore, if $x=A$, then $x\notin A$, implying that $A\notin A$.\par
        Suppose for the sake of contradiction that $A$ is a set and $A\in A$. By Axioms \ref{axm:setsAreObjects} and \ref{axm:singletonPair}, $A\in\{A\}$. Since $A\in A$ and $A\in\{A\}$, Definition \ref{dfn:intersection} tells us that $A\in A\cap\{A\}$. Since there exists an object $x$ (namely $A$) such that $x\in A\cap\{A\}$, by Axiom \ref{axm:emptyset} and Definition \ref{dfn:setEquality}, $A\cap\{A\}\neq\emptyset$. Thus, we have $A$ is a set and $A\cap\{A\}\neq\emptyset$. But by Axiom \ref{axm:regularity}, as the only element of $\{A\}$, $A$ must either not be a set or satisfy $A\cap\{A\}=\emptyset$, a contradiction. Therefore, $A$ is not a set or $A\notin A$. Thus, if $A$ is a set, then $A\notin A$.\par
        Suppose for the sake of contradiction that for two sets $A,B$, $A\in B$ and $B\in A$. By Axioms \ref{axm:setsAreObjects} and \ref{axm:singletonPair}, there exists a set $\{A,B\}$ whose only elements are $A$ and $B$. Since $A\in B$ and $A\in\{A,B\}$, Definition \ref{dfn:intersection}, tells us that $A\in B\cap\{A,B\}$. Since there exists an object $x$ (namely $A$) such that $x\in B\cap\{A,B\}$, by Axiom \ref{axm:emptyset} and Definition \ref{dfn:setEquality}, $B\cap\{A,B\}\neq\emptyset$. By a similar argument, $A\cap\{A,B\}\neq\emptyset$. Thus, we have $A,B$ are sets, $B\cap\{A,B\}\neq\emptyset$, and $A\cap\{A,B\}\neq\emptyset$. But by Axiom \ref{axm:regularity}, an element $x$ of $\{A,B\}$ (namely $A$ or $B$) must either not be a set or satisfy $x\cap\{A,B\}=\emptyset$, a contradiction. Therefore, $A\notin B$ or $B\notin A$.
    \end{proof}
    \item \label{exr:3.2.3}Show (assuming the other axioms of set theory) that the universal specification axiom, Axiom \ref{axm:universalSpecification}, is equivalent to an axiom postulating the existence of a "universal set" $\Omega$ consisting of all objects (i.e., for all objects $x$, we have $x\in\Omega$). In other words, if Axiom \ref{axm:universalSpecification} is true, then a universal set exists, and conversely, if a universal set exists, then Axiom \ref{axm:universalSpecification} is true. (This may explain why Axiom \ref{axm:universalSpecification} is called the axiom of \emph{universal} specification). Note that if a universal set $\Omega$ existed, then we would have $\Omega\in\Omega$ by Axiom \ref{axm:setsAreObjects}, contradicting Exercise \ref{exr:3.2.2}. Thus, the axiom of foundation specifically rules out the axiom of universal specification.
    \begin{proof}
        Suppose Axiom \ref{axm:universalSpecification} is true. Let $P(x)$ be a true statement for all objects $x$. Then there exists a set $\Omega:=\{x:P(x)\text{ is true}\}$, and we have $y\in\Omega$ iff $P(y)$ is true, i.e., iff $y$ is an object, i.e., for all objects $y$. Therefore, a universal set exists. Now suppose that a universal set $\Omega$ exists. By Axiom \ref{axm:specification}, there exists a set $\{x\in\Omega:P(x)\text{ is true}\}$ for some property $P(x)$ pertaining to $x$ (note that this implies that $P(x)$ pertains to all $x$) and $y\in\{x\in\Omega:P(x)\text{ is true}\}$ iff $y\in\Omega$ and $P(y)$ is true. Since $x\in\Omega$ and $y\in\Omega$ are, by the definition of $\Omega$, always true, we have $y\in\{x:P(x)\text{ is true}\}$ iff $P(y)$ is true. Therefore, Axiom \ref{axm:universalSpecification} is true.
    \end{proof}
\end{enumerate}



\section{Functions}
\begin{itemize}
    \item For analysis, we need not just the notion of a set but the notion of a function from one set to another.
    \begin{dfn}[Functions]\label{dfn:functions}
        Let $X,Y$ be sets, and let $P(x,y)$ be a property pertaining to an object $x\in X$ and an object $y\in Y$, such that for every $x\in X$, there is exactly one $y\in Y$ for which $P(x,y)$ is true (this is sometimes known as the \textbf{vertical line test}). Then we define the \textbf{function} $f:X\to Y$ defined by $P$ on the \textbf{domain} $X$ and \textbf{range} $Y$ to be the object which, given any \textbf{input} $x\in X$, assigns an output $f(x)\in Y$, defined to be the unique object $f(x)$ for which $P(x,f(x))$ is true. Thus, for any $x\in X$ and $y\in Y$,
        \begin{equation*}
            y = f(x) \Longleftrightarrow P(x,y)\text{ is true}
        \end{equation*}
    \end{dfn}
    \item \emph{Also known as} \textbf{maps}, \textbf{transformations}, and \textbf{morphisms}.
    \begin{itemize}
        \item Note, however, that a morphism \dq{refers to a more general class of object, which may or may not correspond to actual functions, depending on the context}{49}
    \end{itemize}
    \item Functions obey the axiom of substitution.
    \begin{itemize}
        \item Note that equal inputs imply equal outputs, but unequal inputs do not necessary ensure unequal outputs.
    \end{itemize}
    \item It can be proven that this notion of equality is reflexive, symmetric, and transitive (see Exercise \ref{exr:3.3.1}).
    \item We can now formally define the increment function: Let $X=\N$, $Y=\N$, and $P(x,y)$ be the property that $y=x\pplus$. By Axiom \ref{axm:successorDistinctness}, for each $x\in\N$, there is exactly one $y$ for which $P(x,y)$ is true. Thus, we can define the increment function $f:\N\to\N$ so that $f(x)=x\pplus$ for all $x$. While we cannot define a \emph{decrement} function $g:\N\to\N$ ($0\neq n\pplus$ for any $n\in\N$ by Axiom \ref{axm:0NotSuccessor}), we can define a decrement function $g:\N\setminus\{0\}\to\N$ (by Lemma \ref{lem:backwardsIncrement}).
    \item Informally: Note that while we cannot define a square root function $\sqrt{}:\R\to\R$, we can define the square root function $\sqrt{}:[0,+\infty)\to[0,+\infty)$.
    \item Functions can be defined \textbf{explicitly} or \textbf{implicitly}.
    \begin{itemize}
        \item Explicit definitions: Specify the domain, range, and how one generates the output $f(x)$ from each input.
        \begin{itemize}
            \item For example, the increment function $f$ could be defined explicitly by saying that the domain and range of $f$ are equal to $\N$, and $f(x):=x\pplus$ for all $x\in\N$.
        \end{itemize}
        \item Implicit definitions: Specify what property $P(x,y)$ links the input $x$ with the output $f(x)$.
        \begin{itemize}
            \item For example, the square root function $\sqrt{}$ was defined implicitly by the relation $(\sqrt{x})^2=x$.
            \item Note that implicit definitions are only valid if we know that for every input, there is only one output that obeys the implicit relation.
        \end{itemize}
    \end{itemize}
    \item Often the domain and range are not specified for the sake of brevity.
    \begin{itemize}
        \item For example, we could refer to the increment function $f$ as "the function $f(x):=x\pplus$," "the function $x\mapsto x\pplus$," "the function $x\pplus$," or the extremely abbreviated "$\pplus$."
        \item Note, however, that too much abbreviation can be dangerous, omitting valuable or even necessary information.
    \end{itemize}
    \item Note that while we now use parentheses to clarify the order of operations and enclose the arguments of functions and properties, the usages should be unambiguous from context.
    \begin{itemize}
        \item For example, if $a$ is a number, then $a(b+c)$ denotes $a\times(b+c)$, but if $a$ is a function, then $a(b+c)$ denotes the output of $a$ when the input is $b+c$.
        \item Note that argument are sometimes subscripted --- \dq{a sequence of natural numbers $a_0,a_1,a_2,a_3,\dots$ is, strictly speaking, a function from $\N$ to $\N$, but is denoted by $n\mapsto a_n$ rather than $n\mapsto a(n)$}{51}
    \end{itemize}
    \item Note that functions are not sets and sets are not functions (no $x\in f$ and $A:X\nrightarrow Y$), but we can start with a function $f:X\to Y$ and construct its \textbf{graph} $\{(x,f(x)):x\in X\}$, which describes the function completely (see Section \ref{sse:3.5} for more).
    \item We now define equality for functions.
    \begin{dfn}[Equality of functions]\label{dfn:functionEquality}
        Two functions $f:X\to Y$, $g:X\to Y$ with the same domain and range are said to be equal, $f=g$, if and only if $f(x)=g(x)$ for all $x\in X$. (If $f(x)$ and $g(x)$ agree for some values of $x$, but not others, then we do not consider $f$ and $g$ to be equal.)
    \end{dfn}
    \item Note that functions can be equal over only a certain domain.
    \begin{itemize}
        \item For example, $x\mapsto x$ and $x\mapsto|x|$ are equal if defined only on the positive real axis, and are not equal if defined on $\R$.
    \end{itemize}
    \item There exists the \textbf{empty function} $f:\emptyset\to X$. We need not specify what $f$ does to any input (since there are none), and Definition \ref{dfn:functionEquality} asserts that for each set $X$, there is only one function from $\emptyset$ to $X$.
    \item A fundamental operation of functions is composition.
    \begin{dfn}[Composition]\label{dfn:composition}
        Let $f:X\to Y$ and $g:Y\to Z$ be two functions, such that the range of $f$ is the same set as the domain of $g$. We then define the \textbf{composition} $g\circ f:X\to Z$ of the two functions $g$ and $f$ to be the function defined explicitly by the formula
        \begin{equation*}
            (g\circ f)(x) := g(f(x))
        \end{equation*}
    \end{dfn}
    \item It can be proven that composition obeys the axiom of substitution (see Exercise \ref{exr:3.3.1}).
    \item Composition is not commutative.
    \item However, it is associative.
    \begin{lem}[Composition is associative]
        Let $f:Z\to W$, $g:Y\to Z$, and $h:X\to Y$ be functions. Then $f\circ(g\circ h)=(f\circ g)\circ h$.
        \begin{proof}
            Since $g\circ h$ is a function from $X$ to $Z$, $f\circ(g\circ h)$ is a function from $X$ to $W$. Similarly, $f\circ g$ is a function from $Y\to W$, and hence $(f\circ g)\circ h$ is a function from $X\to W$. Thus, $f\circ(g\circ h)$ and $(f\circ g)\circ h$ have the same domain and range. In order to check that they are equal, we see from Definition \ref{dfn:functionEquality} that we have to verify that $(f\circ(g\circ h))(x)=((f\circ g)\circ h)(x)$ for all $x\in X$. But by Definition \ref{dfn:composition}, we have
            \begin{align*}
                (f\circ(g\circ h))(x) &= f((g\circ h)(x))\\
                &= f(g(h(x)))\\
                &= (f\circ g)(h(x))\\
                &= ((f\circ g)\circ h)(x)
            \end{align*}
            as desired.
        \end{proof}
    \end{lem}
    \item \marginnote{7/24:}We now define several special types of functions.
    \begin{dfn}[One-to-one functions]\label{dfn:injective}
        A function $f$ is \textbf{one-to-one} if different elements map to different elements:
        \begin{equation*}
            x \neq x' \Longrightarrow f(x) \neq f(x')
        \end{equation*}
        Equivalently, a function is one-to-one if
        \begin{equation*}
            f(x) = f(x') \Longrightarrow x = x'
        \end{equation*}
    \end{dfn}
    \item \emph{Also known as} \textbf{injective} (function).
    \item Informally: Note that while the function $f:\Z\to\Z$ defined by $f(n):=n^2$ is not one-to-one, the function $g:\N\to\Z$ defined by $g(n):=n^2$ is one-to-one. Thus, being one-to-one can depend not just on the relation, but on the domain.
    \item \textbf{Two-to-one} (function): A function $f:X\to Y$ such that one can find distinct $x$ and $x'$ in the domain 
    $X$ such that $f(x)=f(x')$.
    \begin{dfn}[Onto functions]\label{dfn:surjective}
        A function $f$ is \textbf{onto} if $f(X)=Y$, i.e., every element in $Y$ comes from applying $f$ to some element in $X$:
        \begin{center}
            For every $y\in Y$, there exists $x\in X$ such that $f(x)=y$
        \end{center}
    \end{dfn}
    \item \emph{Also known as} \textbf{surjective} (function).
    \item Informally: Note that while the function $f:\Z\to\Z$ defined by $f(n):=n^2$ is not onto, if we define the set $A:=\{n^2:n\in\Z\}$, then the function $g:\Z\to A$ defined by $f(n)=n^2$ is onto. Thus, being onto can depend not just on the relation, but on the range.
    \item Injectivity and surjectivity are rather dual to each other (see Exercises \ref{exr:3.3.2}, \ref{exr:3.3.4}, and \ref{exr:3.3.5}).
\end{itemize}


\subsection*{Exercises}
\begin{enumerate}[ref={\thesection.\arabic*}]
    \item \label{exr:3.3.1}\marginnote{7/23:}Show that the definition of equality in Definition \ref{dfn:functionEquality} is reflexive, symmetric, and transitive\footnote{Note that since Definition \ref{dfn:functionEquality} should be an axiom (should axiomatize equality and all that that entails for functions), this part of the exercise is silly (see \cite{bib:TaoErrata}).}. Also verify the substitution property: if $f,\tilde{f}:X\to Y$ and $g,\tilde{g}:Y\to Z$ are functions such that $f=\tilde{f}$ and $g=\tilde{g}$, then $g\circ f=\tilde{g}\circ\tilde{f}$.
    \begin{proof}
        Let $f:X\to Y$ be a function. For any object $f(x)\in Y$, we have $f(x)=f(x)$ by the reflexive axiom of equality (see Section \ref{sss:A.7}). Thus, we have $f(x)=f(x)$ for all $f(x)\in Y$, i.e., for all $x\in X$. Therefore, by Definition \ref{dfn:functionEquality}, we have $f=f$.\par
        Let $f:X\to Y$, $g:X\to Y$ be functions, and let $f=g$. By Definition \ref{dfn:functionEquality}, $f(x)=g(x)$ for all $x\in X$. Since equal objects are of the same type (i.e., follow the reflexive axiom of equality), we have $g(x)=f(x)$ for all $x\in X$. Therefore, by Definition \ref{dfn:functionEquality}, we have $g=f$.\par
        Let $f:X\to Y$, $g:X\to Y$, $h:X\to Y$ be functions, $f=g$, and $g=h$. By Definition \ref{dfn:functionEquality}, $f(x)=g(x)$ for all $x\in X$ and $g(x)=h(x)$ for all $x\in X$. Since equal objects are of the same type (i.e., follow the transitive axiom of equality), we have $f(x)=h(x)$ for all $x\in X$. Therefore, by Definition \ref{dfn:functionEquality}, we have $f=h$.\par
        First, we see that both $g\circ f$ and $\tilde{g}\circ\tilde{f}$ are functions from $X$ to $Z$, i.e., have the same domain and range. To show that $g\circ f=\tilde{g}\circ\tilde{f}$, Definition \ref{dfn:functionEquality} tells us that it will suffice to verify that $(g\circ f)(x)=(\tilde{g}\circ\tilde{f})(x)$ for all $x\in X$. To begin, we see from Definition \ref{dfn:functionEquality} that $f(x)=\tilde{f}(x)$ for all $x\in X$, and that $g(y)=\tilde{g}(y)$ for all $y\in Y$. Since $f(x)\in Y$ for all $f(x)$, we have by Definition \ref{dfn:composition} that
        \begin{align*}
            (g\circ f)(x) &= g(f(x))\\
            &= \tilde{g}(f(x))\\
            &= \tilde{g}(\tilde{f}(x))\\
            &= (\tilde{g}\circ\tilde{f})(x)
        \end{align*}
        as desired.
    \end{proof}
    \item \label{exr:3.3.2}\marginnote{7/24:}Let $f:X\to Y$ and $g:Y\to Z$ be functions. Show that if $f$ and $g$ are both injective, then so is $g\circ f$; similarly, show that if $f$ and $g$ are both surjective, then so is $g\circ f$.
    \begin{proof}
        To prove that $g\circ f$ is injective given the injectivity of $f,g$, Definition \ref{dfn:injective} tells us that we have to verify that $x\neq x' \Longrightarrow (g\circ f)(x)\neq(g\circ f)(x')$ for any distinct elements $x,x'$ of $X$. By Definition \ref{dfn:injective}, we have that if $x$ and $x'$ are two distinct elements of $X$ (such that $x\neq x'$), then $f(x)\neq f(x')$. By Definition \ref{dfn:functions}, we know that $f(x)$ and $f(x')$ are both elements of $Y$. Thus, we have by Definition \ref{dfn:injective} that $g(f(x))\neq g(f(x'))$. Therefore, we have by Definition \ref{dfn:composition} that
        \begin{equation*}
            x\neq x' \Longrightarrow f(x)\neq f(x')
            \Longrightarrow g(f(x))\neq g(f(x'))
            \Longrightarrow (g\circ f)(x)\neq(g\circ f)(x')
        \end{equation*}
        as desired.\par
        By Definition \ref{dfn:composition}, we have $g\circ f:X\to Z$. Thus, to prove that $g\circ f$ is surjective given the surjectivity of $f,g$, Definition \ref{dfn:surjective} tells us that we have to verify that for every $z\in Z$, there exists $x\in X$ such that $(g\circ f)(x)=z$. Let $z$ be any element of $Z$. Then by Definition \ref{dfn:surjective} and the surjectivity of $g$, we have $g(y)=z$ for some $y\in Y$. Similarly, we have $f(x)=y$ for that $y$ and for some $x\in X$. Substituting, we have $g(f(x))=z$ for some $x\in X$. Since $g(f(x))=(g\circ f)(x)$ by Definition \ref{dfn:composition}, we have that for any $z\in Z$, there exists $x\in X$ such that $(g\circ f)(x)=z$.
    \end{proof}
    \setcounter{enumi}{3}
    \item \label{exr:3.3.4}In this section, we give some cancellation laws for composition. Let $f:X\to Y$, $\tilde{f}:X\to Y$, $g:Y\to Z$, and $\tilde{g}:Y\to Z$ be functions. Show that if $g\circ f=g\circ \tilde{f}$ and $g$ is injective, then $f=\tilde{f}$. Is the same statement true if $g$ is not injective? Show that if $g\circ f=\tilde{g}\circ f$ and $f$ is surjective, then $g=\tilde{g}$. Is the same statement true if $f$ is not surjective?
    \begin{proof}
        To prove that $f=\tilde{f}$, Definition \ref{dfn:functionEquality} tells us that it will suffice to show that $f(x)=\tilde{f}(x)$ for all $x\in X$. By Definition \ref{dfn:functionEquality}, $g\circ f=g\circ\tilde{f}$ implies $(g\circ f)(x)=(g\circ\tilde{f})(x)$ for all $x\in X$. Then by Definition \ref{dfn:composition}, $g(f(x))=g(\tilde{f}(x))$ for all $x\in X$. Now by Definition \ref{dfn:injective} and the fact that $g$ is injective, we know that $g(y)=g(y') \Longrightarrow y=y'$. Therefore, since $g(f(x))=g(\tilde{f}(x))$ for all $x\in X$, we have $f(x)=\tilde{f}(x)$ for all $x\in X$. Note that $f$ is not necessarily equal to $\tilde{f}$ if we drop the condition that $g$ is injective (if $g$ is not one-to-one, then we have $g(y)=g(y')$ for some elements $y$ and $y'$ of $Y$ such that $y\neq y'$. Thus, if $f(x)\neq\tilde{f}(x)$, we may still have $g(f(x))=g(\tilde{f}(x))$).\par
        % To prove that $g=\tilde{g}$, we must verify (by Definition \ref{dfn:functionEquality}) that $g(y)=\tilde{g}(y)$ for all $y\in Y$. By Definition \ref{dfn:functionEquality}, $g\circ f=\tilde{g}\circ f$ implies $(g\circ f)(x)=(\tilde{g}\circ f)(x)$ for all $x\in X$. Then by Definition \ref{dfn:composition}, $g(f(x))=\tilde{g}(f(x))$ for all $x\in X$. Now by Definition \ref{dfn:surjective} and the fact that $f$ is surjective, we know that for every $y\in Y$, there exists some $x\in X$ such that $f(x)=y$. Thus, no element $y\in Y$ is not assigned to an $f(x)$. Therefore, we have $g(f(x))=\tilde{g}(f(x))$ for all $f(x)\in Y$, i.e., for all $y\in Y$.
        Suppose for the sake of contradiction that $g\neq\tilde{g}$. Then by Definition \ref{dfn:functionEquality}, either $g$ and $\tilde{g}$ have a different domain or range, or $g(y)\neq\tilde{g}(y)$ for some $y\in Y$. Since $g$ and $\tilde{g}$ have the same domain and range, we must have $g(y)\neq\tilde{g}(y)$ for some $y\in Y$. Now since $f$ is surjective, by Definition \ref{dfn:surjective}, for every $y\in Y$, there is some $x\in X$ such that $y=f(x)$. Thus, $g(f(x))\neq\tilde{g}(f(x))$ for some $x\in X$. Consequently, by Definition \ref{dfn:composition}, $(g\circ f)(x)\neq(\tilde{g}\circ f)(x)$ for some $x\in X$. But this contradicts Definition \ref{dfn:functionEquality}, which, since $g\circ f=\tilde{g}\circ f$, implies that $(g\circ f)(x)=(\tilde{g}\circ f)(x)$ for all $x\in X$. Note that $g$ is not necessarily equal to $\tilde{g}$ if we drop the condition that $f$ is surjective (if $f$ is not surjective, then, as we've only proven the assertion for all $y=f(x)$, we could have some $y\in Y$ not associated with an $f(x)$ for which $g(y)\neq\tilde{g}(y)$).
    \end{proof}
\end{enumerate}




\end{document}