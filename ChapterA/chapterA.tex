\documentclass[../main.tex]{subfiles}

\begin{document}




\chapter{Appendix: The Basics of Mathematical Logic}
\begin{itemize}
    \item \marginnote{6/19:}\textbf{Mathematical logic}: The language one uses to conduct rigorous mathematical proofs.
    \item \dq{A logical argument may sometimes look unwieldy, excessively complicated, or otherwise appear unconvincing. The big advantage of writing logically, however, is that one can be absolutely sure that your conclusion will be correct}{305}
    \item \dq{Being logical is not the only desirable trait in writing, and in fact sometimes it gets in the way; mathematicians for instance often resort to short informal arguments which are not logically rigorous when they wnat to convince other mathematicians of a statement without going through all of the long details}{305}
    \item \dq{Because logic is innate, the laws of logic that you learn should make sense --- if you find yourself having to memorize one of the principles or laws of logic here, without feeling a mental `click' or comprehending why that law should work, then you will probably not be able to use that law of logic correctly and effectively in practice. So, please don't study this appendix the way you might cram before a final --- that is going to be useless. Instead, put away your highlighter pen, and read and understand this appendix rather than merely studying it}{306}
\end{itemize}



\section{Mathematical Statements}
\begin{itemize}
    \item \textbf{Mathematical statement}: A precise statement \dq{concerning various mathematical objects (numbers, vectors, functions, etc.) and relations between them (addition, equality, differentiation, etc.)}{306}
    \begin{itemize}
        \item Mathematical statements are either true or false.
        \item For example, $2+2=4$ is true while $2+2=5$ is false.
    \end{itemize}
    \item \textbf{Ill-formed} (statement): A statement that is neither true nor false (or perhaps not even a statement). \emph{Also known as} \textbf{ill-defined}.
    \begin{itemize}
        \item For example, $0/0=1$.
        \item The antonym is \textbf{well-formed} or \textbf{well-defined} statements.
    \end{itemize}
    \item A logical argument should include only well-formed statements.
    \item \dq{It is important, especially when just learning a subject, to take care in keeping statements well-fromed and precise. Once you have more skill and confidence, of course you can afford once again to speak loosely\footnote{This is very similar to the English class/media arts idea of "you have to know the rules to be able to break them."}, because you will know what you are doing and won't be in as much danger of veering off into nonsense}{307}
    \item The principle of \textbf{proof by contradiction} is \dq{to prove that a statement is true, it suffices to show that it is not false, while to show that a statement is false, it suffices to show that it is not true}{307}
    \item The axioms of logic become more dubious in very non-mathematical situations. One can attempt to apply logic via a mathematical model, but that's science or philosophy. There are other models of logic that attempt to account for these scenarios, but they're beyond the scope of this text.
    \item Note that statements may be true but not useful ($2=2$) or efficient ($4\leq 4$ --- $4=4$ would be better), and they may be false but still be useful ($\pi=22/7$). However, we concern ourselves at present with truth, alone, not usefulness or efficiency as those are to some extent matters of opinion.
    \item \textbf{Expression}: \dq{A sequence of mathematical symbols which produces some mathematical object (a number, matrix, function, set, etc.) as its value}{308-09}
    \begin{itemize}
        \item For example, $2+3*5$ is an expression while $2+3*5=17$ is a statement.
        \item An expression is neither true nor false, but it can be well- or ill-defined.
    \end{itemize}
    \item \textbf{Relation}: A thing that makes statements out of expressions, such as $=$, $<$, $\geq$, $\in$, $\subset$, etc.
    \item \textbf{Property}: A thing that makes statements out of expressions, such as "is prime," "is continuous," "is invertible," etc\footnote{With the introduction of properties, it is worth officially noting that mathematical statements can contain English words.}.
    \item \textbf{Logical connective}: A thing that relates multiple mathematical statements, such as and, or, not, if-then, if-and-only-if, etc.
    \item \textbf{Compound statement}: Two or more mathematical statements joined by logical connectives.
    \item \textbf{Conjunction}: \dq{If $X$ is a statement and $Y$ is a statement, the statement `$X$ and $Y$' is true if $X$ and $Y$ are both true, and is false otherwise}{309}
    \begin{itemize}
        \item Logician's notation: "$X\wedge Y$" or "$X\& Y$."
        \item Note that "$X$ and $Y$" can be reworded "$X$ and also $Y$," or "Both $X$ and $Y$ are true," or even "$X$, but $Y$," or a multitude of other ways and convey the same statement logically, if not expressively.
    \end{itemize}
    \item \textbf{Disjunction}: \dq{If $X$ is a statement and $Y$ is a statement, the statement `$X$ or $Y$' is true if either $X$ or $Y$ is true, or both}{309}
    \begin{itemize}
        \item This is called \textbf{inclusive or}. There is a such thing as \textbf{exclusive or}, but it shows up far less regularly.
        \item An exclusive or may be, "Either $X$ or $Y$ is true, but not both," or "Exactly one of $X$ or $Y$ is true."
        \item To verify a disjunction, it suffices to verify just one case --- this comes in handy when it is significantly easier to verify one case over the other.
    \end{itemize}
    \item \textbf{Negation}: \dq{The statement `$X$ is false'\dots is true if and only if $X$ is false, and is false if and only if $X$ is true}{310}
    \begin{itemize}
        \item Logician's notation: "$\sim X$," "$!X$," or "$\neg X$."
        \item Negations convert "and" into "or": The negation of "Jane has black hair and Jane has blue eyes" is "Jane doesn't have black hair or Jane doesn't have blue eyes." Similarly, the negation of "$x\geq 2$ and $x\leq 6$" is "$x<2$ or $x>6$."
        \item Negations convert "or" into "and": The negation of "John has black hair or brown hair" is "John does not have black hair and does not have brown hair" or, equivalently, "John has neither black nor brown hair." Similarly, the negation of "$x<-1$ or $x>1$" is "$x\geq -1$ and $x\leq 1$."
        \item Negations can produce clearly false statements: the negation of "$x$ is even or odd" is "$x$ is neither even nor odd."
        \item Negations can get unwieldy --- be careful!
    \end{itemize}
    \item Note that "$X$ is true" can be abbreviated to simply "$X$."
    \item \textbf{If and only if}: \dq{If $X$ is a statement, and $Y$ is a statement, we say that "$X$ is true if and only if $Y$ is true," [if] whenever $X$ is true, $Y$ has to be also, and whenever $Y$ is true, $X$ has to be also (i.e., $X$ and $Y$ are `equally true')}{311} \emph{Also known as} \textbf{iff}.
    \begin{itemize}
        \item Logician's notation: "$X\leftrightarrow Y$."
        \item Denotes "logically equivalent statements."
        \item For example, "$x=3$ if and only if $2x=6$" is true.
        \item False statements can be logically equivalent: "$2+2=5$ if and only if $4+4=10$."
        \item Sometimes, it is of interest to show that more than two statements are logically equivalent (See Exercise \ref{exr:A.1.5} and \ref{exr:A.1.6}).
    \end{itemize}
\end{itemize}


\subsection*{Exercises}
\begin{enumerate}[ref={\thesection.\arabic*}]
    \item \marginnote{6/22:}\label{exr:A.1.1}What is the negation of the statement, "either $X$ is true, or $Y$ is true, but not both?"
    \begin{proof}
        $X$ and $Y$ or neither $X$ nor $Y$.
    \end{proof}
    \item \label{exr:A.1.2}What is the negation of the statement, "$X$ is true if and only if $Y$ is true?" There may be multiple ways to phrase this negation.
    \begin{proof}
        $X$ is true if and only if $Y$ is false.
    \end{proof}
    \item \label{exr:A.1.3}Suppose that you have shown that whenever $X$ is true, then $Y$ is true, and whenever $X$ is false, then $Y$ is false. Have you now demonstrated that $X$ and $Y$ are logically equivalent? Explain.
    \begin{proof}
        Yes --- although we have only described the dependence of $Y$ on $X$, the dependence of $X$ on $Y$ is implied (suppose $X$ is false when $Y$ is true --- but since $X$ is false, $Y$ is false, a contradiction; the same holds the other way around).
    \end{proof}
    \item \label{exr:A.1.4}Suppose that you have shown that whenever $X$ is true, then $Y$ is true, and whenever $Y$ is false, then $X$ is false. Have you now demonstrated that $X$ is true if and only if $Y$ is true? Explain.
    \begin{proof}
        No --- $X$ could be false and $Y$ could still be true without leading to any contradictions.
    \end{proof}
    \item \label{exr:A.1.5}Suppose you know that $X$ is true if and only if $Y$ is true, and you know that $Y$ is true if and only if $Z$ is true. Is this enough to show that $X,Y,Z$ are all logically equivalent? Explain.
    \begin{proof}
        Yes --- by the supposition, $X\leftrightarrow Y$ and $Y\leftrightarrow Z$. Since $X$ true implies $Y$ true implies $Z$ true, $X$ false implies $Y$ false implies $Z$ false, $Z$ true implies $Y$ true implies $X$ true, and $Z$ false implies $Y$ false implies $X$ false, $X\leftrightarrow Z$. Thus, $X,Y,Z$ are logically equivalent.
    \end{proof}
    \item \label{exr:A.1.6}Suppose you know that whenever $X$ is true, then $Y$ is true; that whenever $Y$ is true, then $Z$ is true; and whenever $Z$ is true, then $X$ is true. Is this enough to show that $X,Y,Z$ are all logically equivalent? Explain.
    \begin{proof}
        Yes --- clearly $X,Y,Z$ are equally true. If they were not equally false (say if only one or two were false), then we would have a contradiction, as one true implies all the others are true. Thus, they are equally false, too. Therefore, they are logically equivalent.
    \end{proof}
\end{enumerate}



\section{Implication}
\begin{itemize}
    \item \marginnote{6/24:}\textbf{Implication}: \dq{If $X$ is a statement, and $Y$ is a statement, then `if $X$, then $Y$' is the implication from $X$ to $Y$}{312}
    \begin{itemize}
        \item Logician's notation: "$X\Longrightarrow Y$."
        \item Can be reworded "when $X$ is true, $Y$ is true," "$X$ implies $Y$," "$Y$ is true when $X$ is," or "$X$ is true only if $Y$ is true."
        \item \dq{What this statement `if $X$, then $Y$' means depends on whether $X$ is true or false. If $X$ is true, then `if $X$, then $Y$' is true when $Y$ is true, and false when $Y$ is false. If however $X$ is false, then `if $X$, then $Y$' is \emph{always} true, regardless of whether $Y$ is true or false!}{312}
        \item \dq{When $X$ is true, the statement `if $X$, then $Y$' implies that $Y$ is true. But when $X$ is false, the statement `if $X$, then $Y$' offers no information about whether $Y$ is true or not; the statement is true, but \textbf{vacuous}}{312}
    \end{itemize}
    \item \textbf{Vacuous} (statement): A statement that \dq{does not convey any new information}{312}
    \begin{itemize}
        \item Vacuous statements can still be helpful in proofs, though.
    \end{itemize}
    \item The only way to disprove an implication (to show that it's false) is to show that the hypothesis (first statement) is true \emph{and} the conclusion (second statement) is false.
    \item Whereas "if and only if" asserts that $X$ and $Y$ are equally true, an implication only asserts that $Y$ is at least as true as $X$.
    \item Vacuously true implications are often used in a situation where the conclusion and hypothesis are both false, but the implication is true regardless, e.g., "if John had left work at 5pm, then he would be here by now."
    \item They can also be used to facilitate a proof by contradiction: We know that "if John had left work at 5pm, then he would be here by now" and that "John is not here now." Suppose that John left work at 5pm. Then he would be here by now, a contradiction. Thus, John did not leave work at 5pm.
    \item Implications can be true even when there is no causal link between the hypothesis and conclusion\footnote{This is really weird --- I'll need to return to this later.}, e.g., if $1+1=2$, then Washington, D.C. is the capital of the United States is true.
    \begin{itemize}
        \item Using such acausal implications in a logical argument is typically frowned upon, since it will likely cause unneeded confusion.
    \end{itemize}
    \item To prove, "if $X$, then $Y$," assume $X$ is true and then deduce $Y$ from $X$ and other known hypotheses.
    \begin{itemize}
        \item It does not matter whether or not $X$ is true; an implication can be proved true irrespective of whether or not $X$ is true.
        \item It would be incorrect, though, to assume $Y$ and deduce $X$. One cannot assume the conclusion and deduce the hypothesis.
    \end{itemize}
    \item With regard to vacuously true statements, one need not be concerned that some hypotheses in their argument might not be correct, as long as their argument is still structured to give the correct conclusion regardless of whether those hypotheses were true or false (proving $n(n+1)$ is even when $n\in\N$).
    \item \textbf{Converse} (of "if $X$, then $Y$"): The statement "if $Y$, then $X$."
    \begin{itemize}
        \item "$X$ if and only if $Y$" implies that "if $X$, then $Y$" \emph{and} its converse are true.
    \end{itemize}
    \item \textbf{Inverse} (of "if $X$, then $Y$"): The statement "if $X$ is false, then $Y$ is false."
    \begin{itemize}
        \item The inverse of a true implication is not necessarily a true implication.
    \end{itemize}
    \item \textbf{Contrapositive} (of "if $X$, then $Y$"): The statement "if $Y$ is false, then $X$ is false."
    \begin{itemize}
        \item The contrapositive and the original statement are equally true.
    \end{itemize}
    \item \textbf{Proof by contradiction}: \dq{To show that something must be false, assume first that it is true, and show that this implies something which you know to be false (e.g., that a statement is simultaneoulsy true and not true)}{316} \emph{Also known as} \textbf{reductio ad absurdum}.
    \begin{itemize}
        \item Particularly (but not exclusively) useful for proving "negative" statements, e.g., $X$ is false, $a\neq b$, etc.
    \end{itemize}
    \item On logician's notation: general-purpose mathematicians do not often use them because English words are often more readable and don't take up that much more space. That being said, $\Longrightarrow$ is a possible exception.
\end{itemize}



\setcounter{section}{7}
\section{Misc. Notes}
\begin{itemize}
    \item \dq{From a logical point of view, there is no difference between a lemma, proposition, theorem, or corollary --- they are all claims waiting to be proved. However, we use these terms to suggest different levels of importance and difficulty. A lemma is an easily proved claim which is helpful for proving other propositions and theorems, but is usually not particularly interesting in its own right. A proposition is a statement which is interesting in its own right, while a theorem is a more important statement than a proposition which says something definitive on the subject, and often takes more effort to prove than a proposition or lemma. A corollary is a quick consequence of a proposition or theorem that was proven recently.}{25}
\end{itemize}




\end{document}