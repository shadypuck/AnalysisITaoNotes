\documentclass[../main.tex]{subfiles}

\begin{document}




\chapter{Appendix: The Basics of Mathematical Logic}
\setcounter{section}{7}
\section{Misc. Notes}
\begin{itemize}
    \item \dq{From a logical point of view, there is no difference between a lemma, proposition, theorem, or corollary --- they are all claims waiting to be proved. However, we use these terms to suggest different levels of importance and difficulty. A lemma is an easily proved claim which is helpful for proving other propositions and theorems, but is usually not particularly interesting in its own right. A proposition is a statement which is interesting in its own right, while a theorem is a more important statement than a proposition which says something definitive on the subject, and often takes more effort to prove than a proposition or lemma. A corollary is a quick consequence of a proposition or theorem that was proven recently.}{25}
\end{itemize}




\end{document}